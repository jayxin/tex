 % !Mode:: "TeX:UTF-8"
 %% 使用 minted 宏包前先阅读 http://www.latexstudio.net/archives/328
 %% XeLaTeX 方式编译, TeXLive2016. 也可以直接双击 makepdf.bat 进行编译
\documentclass[UTF8, hyperref, fontset = none]{ctexart}
%% 中文使用伪斜体
\xeCJKsetup{
  AutoFakeSlant = true
  }
\xeCJKsetslantfactor{0.17}% 设置中文伪斜体倾斜度数
\usepackage{geometry}
\geometry{left=2.5cm,right=2.5cm,top=2.5cm,bottom=2.5cm}

\usepackage[dvipsnames, svgnames]{xcolor}
\usepackage[minted]{tcolorbox}
\tcbuselibrary{skins, listings, xparse, breakable}

% 字体设置
\ctexset{fontset = none}
% set up fonts
\setCJKmainfont[BoldFont={SimHei}]{SimSun}
\setCJKsansfont{SimHei}
\setCJKmonofont{SimHei}% 中文等宽字体

%%%%%%%%%%%%%%%%%%%%%% 行间代码样式 %%%%%%%%%%%%%%%%%%%%%%%%%%%%%%%%%%%
%%% 其中 minted style=algol, friendly, lovelace, pastie, tango, vs, autumn, fruity, monokai, perldoc, trac, borland, igor, murphy, rrt,vim
%%% 这些是代码的样式
%%%%%%%%%%%% mycode 环境 %%%%%%%%%%%%%%%%%%%%%%
% breakable 可换页
\newtcblisting{mycode}[2]{breakable,drop shadow,listing engine=minted,minted style=trac,
  minted language=#1,minted options={fontsize=\small,linenos,
    numbersep=3mm},
  listing only,
  left=6mm,enhanced,title={#2},
  colframe=blue!50!black,colback=blue!10!white,colbacktitle=blue!5!yellow!10!white,
  fonttitle=\bfseries,coltitle=black,attach boxed title to top center=
  {yshift=-0.25mm-\tcboxedtitleheight/2,yshifttext=2mm-\tcboxedtitleheight/2},
  boxed title style={enhanced,boxrule=0.5mm,
    frame code={ \path[tcb fill frame] ([xshift=-4mm]frame.west)
      -- (frame.north west) -- (frame.north east) -- ([xshift=4mm]frame.east)
      -- (frame.south east) -- (frame.south west) -- cycle; },
    interior code={ \path[tcb fill interior] ([xshift=-2mm]interior.west)
      -- (interior.north west) -- (interior.north east)
      -- ([xshift=2mm]interior.east) -- (interior.south east) -- (interior.south west)
      -- cycle;} },
  overlay={\begin{tcbclipinterior}\fill[red!20!blue!20!white] (frame.south west)
      rectangle ([xshift=5mm]frame.north west);\end{tcbclipinterior}}}
%%%%%%%%%%%%%%%%%%%%%%%%%%%%%%%%%%%%%%%%%%%%%%%%%%%%%%5
%%%%%%%%%%%%%%%%% commandshell %%%%%%%%%%%%%%%%%%%%%
\newtcblisting{commandshell}{colback=black,colupper=white,colframe=yellow!75!black,
	listing engine=minted,minted style=algol,
	minted language=bash}
%%%%%%%%%%%%%%%%%%%%%%%%%%%%%%%%%%%%%%%%%%%%%%%%%%%%%%%%%%%%%%%%%%%%%
%\usemintedstyle{algol} %%全局设置
% 设置 perl 行内代码样式
\usemintedstyle[perl]{algol}
\newmintinline{perl}{showspaces} %% 行内代码样式, showspaces 代表将空格先显式的打印出来
\begin{document}
这是一段 perl 代码 \perlinline/my $foo = $bar;/ 行内代码示例
\section{commandshell环境使用说明}

%\begin{mycode}{language}{title}
%Your codes
%  bla...
%  bla...
%\end{mycode}

\begin{commandshell}
	ls -al
	cd /usr/lib
\end{commandshell}
\section{commandbox环境使用说明}
\DeclareTotalTCBox{\commandbox}{ s v }
{verbatim,colupper=white,colback=black!75!white,colframe=black}
{\IfBooleanTF{#1}{\textcolor{red}{\ttfamily\bfseries > }}{}%
	\lstinline[language=command.com,keywordstyle=\color{blue!35!white}\bfseries]^#2^}

\commandbox*{cd "My Documents"} changes to directory \commandbox{My Documents}.
\commandbox*{dir /A} lists the directory content.
\commandbox*{copy example.txt d:\target} copies \commandbox{example.txt} to
\commandbox{d:\target}.
\commandbox*{texdoc -l minted}
\section{mycode环境使用说明}
\begin{mycode}{java}{My Java}
package win7.xiao;
  
public class duotai {
    
  public static void main(String[] args) {
      // TODO Auto-generated method stub
      /*Cat cat=new Cat();
      cat.cry();
      Dog dog=new Dog();*/
      
      /*Animal an=new Cat();
      an.cry();
      an=new Dog();
      an.cry();*/
      
      Master master=new Master();
      master.feed(new Dog(), new Bone());
      master.feed(new Cat(), new Fish());
    }
    
  }
  //主人类
  
  class Master
  {
    //给动物喂食物,使用多态,方法就可以用一个
    public void feed(Animal an,Food f)
    {
      an.eat();
      f.showName();
    }
  }
  class Food
  {
    String name;
    public void showName()
    {
      
    }
  }
  
  //鱼
  class Fish extends Food
  {
    public void showName()
    {
      System.out.println("fish");
    }
  }
  
  //骨头
  class Bone extends Food
  {
    public void showName()
    {
      System.out.println("骨头");
    }
  }
  //动物类Animal
  class Animal
  {
    String name;
    int age;
    public String getName() {
      return name;
    }
    public void setName(String name) {
      this.name = name;
    }
    public int getAge() {
      return age;
    }
    public void setAge(int age) {
      this.age = age;
    }
    
    //动物会叫
    public void cry()
    {
      System.out.println("不知道怎么叫");
    }
    
    //动物可以吃
    public void eat()
    {
      System.out.println("不知道吃什么");
    }
  }
  class Dog extends Animal
  {
    //狗叫
    public void cry()
    {
      System.out.println("汪汪叫");
    }
    //狗吃
    public void eat()
    {
      System.out.println("狗喜欢吃骨头");
    }
  }
  class Cat extends Animal
  {
    //猫自己叫
    public void cry()
    {
      System.out.println("喵喵叫");
    }
    //猫吃
    public void eat()
    {
      System.out.println("猫喜欢吃鱼");
    }
  }
\end{mycode}
\end{document}