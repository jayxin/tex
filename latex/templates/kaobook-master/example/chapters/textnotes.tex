\renewcommand*{\chapterformat}
{
  \enskip\mbox{\scalebox{4}{\thechapter\autodot}}
}
\renewcommand\chapterlinesformat[3]
{
  \parbox[b]{\textwidth+\marginparsep+\marginparwidth}{\hrulefill#2}\par%
  #3%
  %\parbox[b]{\textwidth+\marginparsep+\marginparwidth}{\hrulefill}%
}
\setchapterpreamble[u]{\margintoc}
\chapter{Sidenotes and Marginnotes}

\section{Sidenotes}

To insert a sidenote, just enter the command \verb|\sidenote{Text of the note}|. You can specify a mark\sidenote[O]{This sidenote has a 
	special mark} with \verb|\sidenote[mark]{Text}|, or you can specify 
an offset and a mark with \verb|\sidenote[offset][mark]{Text}|, in which 
case the mark can be empty. If you want to know more, read the 
documentation of the \verb|snotez| package.

Sidenotes are handled through the \verb|snotez| package, which relies on 
the \verb|marginnote| package. The sidenote counter is never reset, but 
if you want you can reset it at every chapter.

\section{Marginnotes}

\marginnote{\blindtext}

\subsection{Usage}

This command is similar to the previous one: you can use it like 
\verb|\marginnote[offset]{Text}|, where the offset argument can be left 
out.

\subsection{Details}

We load the packages \verb|marginnote|, \verb|marginfix| and 
\verb|placeins|. Since \verb|snotez| uses \verb|marginnote|, what we say 
for marginnotes is also valid for sidenotes. The style of marginnotes 
and captions is the same, and the notes are shifted slightly upwards 
(\verb|\renewcommand{\marginnotevadjust}{-11pt}|) in order to allineate 
them to the bottom of the line of text where the marginnote is issued.

\marginnote[*-1.5]{The offset option can be either a length or a 
	multiple of \texttt{baselineskip}, \eg 
	\texttt{\\marginnote[*-3]{Text}}.}

\section{Footnotes}

Footnotes force the reader to constantly move from one area of the page 
to the other. Arguably marginnotes solve this issue, so you should not 
use footnotes. Nevertheless, for completeness, we provide the standard 
command \verb|\footnote|, just in case you want to put a footnote once 
in a while\footnote{And this is how they look like.}.

\section{Margintoc}

Since we are talking about margins, we introduce here the 
\verb|\margintoc| command, which accepts a parameter for the vertical 
offset, like so: \verb|\margintoc[offset]|. It can be used in any point 
of the document, but we think it makes sense to use it at the beginning 
of chapters or parts. We like to put it in the chapter preamble, with 
this code:

\begin{verbatim}
	\setchapterpreamble[u]{\margintoc}
	\chapter{Sidenotes and Marginnotes}
\end{verbatim}
