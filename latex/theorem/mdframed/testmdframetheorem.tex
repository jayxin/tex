\documentclass[a4paper,10pt]{ctexart}
\usepackage{amsmath,amsthm,mathrsfs,amsfonts,amssymb}
\usepackage{xcolor}
\usepackage{mdframed} 
\theoremstyle{definition}
\mdfdefinestyle{theoremstyle}{%
linecolor=gray!40,linewidth=.5pt,%
backgroundcolor=gray!10,
skipabove=8pt,
skipbelow=5pt,
innerleftmargin=7pt,
innerrightmargin=7pt,
frametitlerule=true,%
frametitlerulewidth=.5pt,
frametitlebackgroundcolor=gray!15,
frametitleaboveskip=0pt,
frametitlebelowskip=0pt,
innertopmargin=.4\baselineskip,
innerbottommargin=.4\baselineskip,
shadow=true,shadowsize=3pt,shadowcolor=black!20,
theoremseparator={.},
}

\mdtheorem[style=theoremstyle]{defi}{$\blacklozenge$定义}[section]
\mdtheorem[style=theoremstyle]{prop}{$\blacklozenge$命题}
\mdtheorem[style=theoremstyle]{theo}{$\blacklozenge$定理}
\mdtheorem[style=theoremstyle]{lem}[defi]{$\blacklozenge$引理}
\mdtheorem[style=theoremstyle]{coro}[defi]{$\blacklozenge$推论}
\mdtheorem[style=theoremstyle]{axiom}[defi]{$\blacklozenge$公理} 
\begin{document}
\section{测试定理}
本文分享的是关于定理定义与推论等的设计方法与设计展示,以最为简单的定理定义为例子进行样式的展示。
\begin{defi}在~$(a,b)$~上给定函数~$f(x)$,$x_0\in(a,b)$,若
\begin{equation}\label{eq:sec3.1-1}
\lim_{x\to x_0}f(x)=f(x_0),
\end{equation}
则称函数~$f(x)$~在~$x_0$~点连续,$x_0$~称为\emph{连续点},否则就称~$x_0$~为\emph{间断点}。
\end{defi}
直观地说,就是当动点~$x$~趋于定点~$x_0$~时,若动点函数值趋于定点的函数值,则函数在~$x_0$~点连续。若~$x_0$~是连续点,则
当自变量在~$x_0$~点有无限小的变化,引起因变量的变化也无限的小。
\begin{theo}[Darboux定理]
\label{theo:riemann}
设~$f(x)\in C^{(1)}[a, b]$。若~$f'(a)<f'(b)$,证明对任意~$\eta$,若~$\eta$~满足
\begin{equation}
 f'(a)<\eta<f'(b),
\end{equation}
则存在~$\xi\in(a, b)$,使得~$f'(\xi)=\eta$。
\end{theo}
\begin{proof}
不妨设~$f(x)$~单调上升。那么对任意~$x_0\in(a,b)$,当~$x\to x_0-0$~时,函数值~$f(x)$~上升,并有上界~$f(x_0)$,所以极限存在,且
\[
  \lim_{x\to x_0-0}f(x)=f(x_0-0)\leq f(x_0)
\]
同理,当~$x\to x_0+0$~时,函数值~$f(x)$~下降,并有下界~$f(x_0)$,所以极限存在,且
\[
  \lim_{x\to x_0+0}f(x)=f(x_0+0)\geq f(x_0)
\]
若~$f(x_0-0)=f(x_0+0)$,则~$x_0$~是函数的连续点;若~$f(x_0-0)\neq f(x_0+0)$,则~$x_0$~是函数的第一类间断点。由于~$x_0$~的
任意性,所以区间上每一点不是连续点就是第一类间断点。
\end{proof}
\begin{prop}
设给定实数~$x_1,x_2$,且~$x_1<x_2$,总可以找到有理数~$q_1,q_2$,使得
\[
  x_1<q_1<q_2<x_2,
\]
因此
\[
  a^{x_1}=\sup_{q\leq x_1}\{a^q\}\leq a^{q_1}<a^{q_2}\leq\sup_{q\leq x_2}\{a^q\}=a^{x_2},
\]
即~$a^x$~在~$\mathbb{R}$~上严格上升。
\end{prop}

\begin{lem}\label{lem:sec3.4-1}
设~$a>1$,$n$~为正整数,则存在实数~$b>1$,使得~$a=b^n$~或者~$\sqrt[n]a=b$。
\end{lem}
\end{document} 