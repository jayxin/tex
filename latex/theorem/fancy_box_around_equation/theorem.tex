 
\documentclass{book}
\usepackage{amssymb}
\usepackage{amsmath}
\usepackage{amsthm}
\usepackage{empheq}
\usepackage{bookman}
\usepackage[framemethod=tikz]{mdframed}

\definecolor{ocre}{RGB}{243,102,25}
\definecolor{mygray}{RGB}{243,243,244}

\newcommand*\mymathbox[1]{%
  \fcolorbox{ocre}{mygray}{\hspace{1em}#1\hspace{1em}}}

\newtheoremstyle{mystyle}
  {\topsep}
  {\topsep}
  {\normalfont}
  {}
  {\sffamily\bfseries}
  {.}
  {.5em}
  {{\color{ocre}\thmname{#1}~\thmnumber{#2}}\thmnote{\,--\,#3}}%
\theoremstyle{mystyle}
\newmdtheoremenv[
  backgroundcolor=mygray,
  linecolor=ocre,
  leftmargin=20pt,
  innerleftmargin=0pt,
  innerrightmargin=0pt,
  ]{theo}{Theorem}[section]

\begin{document}

\chapter{Test chapter}
\section{Test section}

\begin{theo}[Name of the theorem]
In $E=\mathbb{R}^n$ all norms are equivalent.
\begin{align}
a &= b\\
E &= mc^2 + \int_a^a x\, \mathrm{d}x
\end{align}
\end{theo}

\begin{empheq}[box=\mymathbox]{align}
a&=b\\
E&=mc^2 + \int_a^a x\, \mathrm{d}x
\end{empheq}

\end{document}
 