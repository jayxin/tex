
\documentclass{article}

\usepackage{lstcustom}

\pagestyle{plain}

\title{Latex Listings Eclipse Style}
\date{}
\author{Mark Royer}

\begin{document}

\abstract

The following document is an example of using the
\ttfamily{lstcustom.sty} file.  Please look at the following examples
of how to use the style file. The default style is set to eclipse, and
the default language is Java.  These are set at the end of the
\ttfamily{.sty} file.

\section*{Listing Inline}

% So we don't create weird spacing for single words.
\lstset{breaklines=false}

You can use the \ttfamily{\textbackslash listinline\$command\$} to use the custom
eclipse style markup in paragraphs.  The 8 basic types in Java are
\lstinline$boolean$, \lstinline$byte$, \lstinline$char$,
\lstinline$short$, \lstinline$int$, \lstinline$long$,
\lstinline$float$, and \lstinline$double$. These should have been
properly formatted in the previous sentence.

% Make sure to break lines the rest of the document
\lstset{breaklines=true}

\section*{Block Listing Example}

A slightly longer block example is shown in listing \ref{lst:simple}.

\begin{lstlisting}[caption={A simple listing.}, label={lst:simple}]
  /**
    * @param args
    *             Program arguments
    */
  public static void main(String[] args) {
    // Now for the enlightening message.
    System.out.println("Hmm... hello big world!";
    // @TODO Finish this example
    // Just some comment that is probably too long to fit in the space provided....
  }
\end{lstlisting}

\section*{File Listing Example}

An example that includes a \ttfamily{.java} file is shown in listing \ref{lst:file}.

\lstinputlisting[caption={A listing from the file
  \ttfamily{Rectangle.java}}, label={lst:file}]{Rectangle.java}

\end{document}

