% !TeX program = pdflatex
% !TeX TS-program = pdflatex
% !TeX TXS-program:compile = txs:///pdflatex/

% !BIB program = biber
% !BIB TS-program = biber
% !TeX TXS-program:bibliography = txs:///biber




%%%%%%%%%%%%%%%%
%%  PREAMBLE  %%
%%%%%%%%%%%%%%%%


% To ``accept'' all changes in the document:
%\PassOptionsToPackage{final}{changes}
%\PassOptionsToPackage{disable}{todonotes}


\documentclass[11pt, a4paper, twoside]{book}
% Use option ``fleqn'' for left-aligned display formulae

\newcommand{\dissafterdefense}{1}
% See https://www.rsf.uni-bonn.de/dekanat/Formulare/vorgaben-pflichtexemplare-wiwi/:
% Set to 1 for the final version AFTER the defense, because only after the defense is the second page (which lists the date of the defense and the names of the supervisors) to be included!
% The same applies to the ``Tag der Promotion'' note on the title page.
% Also the CV at the end of the disseration is to be included only after the defense.

\newcommand{\disstitle}
	{Essays in Theoretical Microeconomics and\unskip\nobreak\ Empirical\unskip\nobreak\ Macroecoconomics with\unskip\nobreak\ Implications for Social Policy All\unskip\nobreak\ around\unskip\nobreak\ the\unskip\nobreak\ World}
\newcommand{\dissauthor}
	{\v{U}.~R.~B{\`e}ta \c{Z}\^ane-{\AA}l}
\newcommand{\dissbornin}
	{Summacumlaudeville}
\newcommand{\disssubmitdate}
	{July~2023}
\newcommand{\dissdean}
	{Prof.~Dr. J\"urgen von Hagen}
\newcommand{\dissfirstsupervisor}
	{Lou~E. Vu{\"i}-T{\o}n}
\newcommand{\disssecondsupervisor}
	{Prof.~Dr. V\H{a}e-R\'ee Sm\k{a}rt}
\newcommand{\dissyear}
	{2023}
\newcommand{\dissdatedefense}
	{1.~September~\dissyear}

% !TeX program = pdflatex
% !TeX TXS-program:compile = txs:///pdflatex/
% !TeX TS-program = pdflatex
% !BIB program = biber
% !TeX TXS-program:bibliography = txs:///biber




%%%%%%%%%%%%%%%%%%%%%%%%%%%%%%%%%%%%%%%%%
%%  FUNDAMENTAL PACKAGES AND COMMANDS  %%
%%%%%%%%%%%%%%%%%%%%%%%%%%%%%%%%%%%%%%%%%


\usepackage{ifxetex}  % Detect if engine is XeTeX/XeLaTeX

\usepackage{ifthen}

\usepackage[utf8]{inputenc}  % so that umlauts can be input without having to use TeX code

\ifxetex
	\usepackage[euenc]{fontspec}
\else
	\usepackage[LGR, T1]{fontenc}  % LGR needed for sansserif math
\fi

\usepackage[ngerman, american, USenglish]{babel}  % German and US English hyphenation and quotation marks
\selectlanguage{USenglish}
\usepackage[ngerman, USenglish]{isodate}

\usepackage[babel, german=quotes]{csquotes}  % Needed for correct German quotes via BibLaTeX's \mkbibquote{...}
%\usepackage[babel, german=guillemets]{csquotes}  % Needed for correct German quotes via BibLaTeX's \mkbibquote{...}

\usepackage{calc}  % Enables calculations for lengths; provides, e.g., \widthof{text}
\usepackage{fp}  % Enables calculations in LaTeX
% \usepackage[nomessages]{fp}

\usepackage{etoolbox}  % Enables manipulating LaTeX commmands via \preto, \appto, \patchcmd, etc.
\usepackage{xpatch}  % Enables manipulating LaTeX commmands via \xpatchcmd etc.
\usepackage{letltxmacro}
\usepackage{xparse}

%\usepackage{geometry}  % See geometry.pdf to learn the layout options. There are lots.
%\usepackage[mathscr]{eucal}
%\geometry{a4paper, top=20mm, bottom=20mm, right=20mm, left=30mm}
%\geometry{landscape}  % Activate for for rotated page geometry

\usepackage{ragged2e}
% Provides, among others, the \RaggedRight environment, which is \raggedright but with hyphenation enabled.
%\renewcommand{\raggedright}{\RaggedRight}

\usepackage{import}	 % To allow for relative paths in nested \input's (\import's)

% \usepackage{float} % damit die figure da ist wo sie sein soll
\usepackage{placeins}  % Improve placing of floats (figures, tables), provides \FloatBarrier

% DO NOT USE \usepackage{amssymb}!
% The AMS symbols are included in the mathdesign package or the fourier package.
\usepackage{amsmath}
\MakeRobust{\eqref}
	% See https://tex.stackexchange.com/questions/61764/eqref-in-captions-with-mathtools
\renewcommand{\eqref}[1]{(\ref{#1})}  % Necessary to make the font switch from serif to sansserif where required.
\usepackage{amsthm}	% provides \newtheoremstyle
\usepackage{mathtools}
%\mathtoolsset{centercolon}
	% This makes the compilation fail in combination with tikz. See
	% https://tex.stackexchange.com/questions/89467/why-does-pdftex-hang-on-this-file.
% Inspired by https://tex.stackexchange.com/questions/251460/how-to-put-symbols-of-equal-size-on-top-of-each-other
\newcommand{\succeqq}{%
  \mathrel{%
    \vcenter{\offinterlineskip
      \ialign{##\cr$\succ$\cr\noalign{\kern 1pt}$=$\cr}%
    }%
  }%
}
\newcommand{\nsucceqq}{\mathrel{\not\succeqq}}

\usepackage[
	colorlinks=true,
	linkcolor=UBonnBlue,
	citecolor=UBonnBlue,
	filecolor=black,
	urlcolor=UBonnBlue,
	bookmarks=true,
	bookmarksnumbered=true,
	bookmarksopenlevel=2,
	pdfstartview=Fit,
	pdfpagelayout=SinglePage,
	plainpages=false,
	pdfpagelabels=true
]{hyperref}
\urlstyle{same}   % Sets URLs in the text font instead of the typewriter font
\Urlmuskip = 0mu\relax  % Prevent additional whitespace before/after breakable characters in URLs
%\setlength{\Urlmuskip}{0mu}
\newcommand{\email}[1]{\href{mailto:#1}{\nolinkurl{#1}}}
% Change ``([sub]sub)section'' to ``Section'' in \autoref
% and add na
% ==>
\addto\extrasUSenglish{%
	\renewcommand{\chapterautorefname}{Chapter}%  instead of ``chapter''
	\renewcommand{\sectionautorefname}{Section}%  instead of ``section''
	\renewcommand{\subsectionautorefname}{Section}%  instead of ``subsection''
	\renewcommand{\subsubsectionautorefname}{Section}%  instead of ``subsubsection''
}
\addto\extrasamerican{%
	\renewcommand{\chapterautorefname}{Chapter}%  instead of ``chapter''
	\renewcommand{\sectionautorefname}{Section}%  instead of ``section''
	\renewcommand{\subsectionautorefname}{Section}%  instead of ``subsection''
	\renewcommand{\subsubsectionautorefname}{Section}%  instead of ``subsubsection''
}
\addto\extrasngerman{%
	\renewcommand{\subsectionautorefname}{Abschnitt}%  instead of ``Unterabschnitt''
	\renewcommand{\subsubsectionautorefname}{Abschnitt}%  instead of ``Unterunterabschnitt''
}
\newcommand*{\Appendixautorefname}{Appendix}
	% See https://tex.stackexchange.com/questions/207744/no-autoref-name-for-appendix
\newcommand*{\hypothesisautorefname}{Hypothesis}
\newcommand*{\resultautorefname}{Result}
% <==
% Enclose the back references in the bibliography to the pages on which a reference is cited in square brackets (Econometrica style):
% (only applicable if using BibTeX instead of BibLaTeX)
%\let \backrefold \backref
%\renewcommand*{\backref}[1]{[\backrefold{#1}]}

% \usepackage{cleveref}	% Provides \cref{...} etc. for flexible referencing.

\usepackage{verbatim}
\let\ignore=\comment
\let\endignore=\endcomment

%% Command to suppress text:
%% See https://tex.stackexchange.com/questions/97347/selectively-suppress-generation-of-typeset-output
%% ==>
%\makeatletter
%\font\dummyft@=dummy \relax
%\def\suppress{%
%	\begingroup\par
%	\parskip\z@
%	\offinterlineskip
%	\baselineskip=\z@skip
%	\lineskip=\z@skip
%	\lineskiplimit=\maxdimen
%	\dummyft@
%	\count@\sixt@@n
%	\loop\ifnum\count@ >\z@
%	\advance\count@\m@ne
%	\textfont\count@\dummyft@
%	\scriptfont\count@\dummyft@
%	\scriptscriptfont\count@\dummyft@
%	\repeat
%	\let\selectfont\relax
%	\let\mathversion\@gobble
%	\let\getanddefine@fonts\@gobbletwo
%	\tracinglostchars\z@
%	\frenchspacing
%	\hbadness\@M}
%\def\endsuppress{\par\endgroup}
%\makeatother
%% <==




%%%%%%%%%%%%%%%%%%%%%%%%%%%%%%%%%
%%  GRAPHICS-RELATED PACKAGES  %%
%%%%%%%%%%%%%%%%%%%%%%%%%%%%%%%%%


\usepackage{graphicx}
\DeclareGraphicsRule{.tif}{png}{.png}{`convert #1 `dirname #1`/`basename #1 .tif`.png}

\usepackage{epstopdf}  % Has to be loaded after graphic{s,x}

\usepackage[table]{xcolor}
\definecolor{UBonnBlue}   {RGB}{0007,0082,0154}
\definecolor{darkblue}    {rgb}{0.00,0.20,0.40}
\definecolor{darkred}     {rgb}{0.80,0.00,0.00}
\colorlet   {darkred25}   {darkred!25!white}
\definecolor{customgreen} {rgb}{0.15,0.55,0.00}
\definecolor{custompurple}{rgb}{0.15,0.00,0.75}

\usepackage{pgf, pgfarrows, pgfnodes, pgfshade}
\usepackage{pgfplots}

\usepackage{tikz}
\usetikzlibrary{mindmap, trees, patterns}

\usepackage{pdflscape}
% To set single pages in landscape ortientation

\usepackage{afterpage}
% To wrap text around landscape-oriented pages




%%%%%%%%%%%%%%%%%%%%%%%%%%%%%%%%
%%  ADVANCED TEXT FORMATTING  %%
%%%%%%%%%%%%%%%%%%%%%%%%%%%%%%%%


\usepackage[full]{textcomp}  % ``full'' option tequired some packages (e.g., ``newtxtext'')
\usepackage{xfrac}	% Provides \sfrac; loads textcomp (without ``full'' option)

\usepackage{soul}  % Provides a~highlighting command, \hl{...}, and a \caps{...} command

% Allow for fine-grained scaling of font sizes
% ==>
\usepackage{relsize}
\renewcommand\RSpercentTolerance{1}
% Enabling slightly reduced font for CAPS:
\ifxetex
	\renewcommand{\caps}[1]{\textscale{0.96}{\addfontfeature{LetterSpace=5}\MakeUppercase{#1}}}
\else
	\renewcommand{\caps}[1]{\textscale{0.96}{\textls[35]{\MakeUppercase{#1}}}}
\fi
% <==

\frenchspacing	% Prevent excessively large whitespace after periods
\sloppy




%%%%%%%%%%%%%%%%%%%%%%%%%%%%%%%%%%%%
%%  COMMANDS FOR TROUBLESHOOTING  %%
%%%%%%%%%%%%%%%%%%%%%%%%%%%%%%%%%%%%


\usepackage{printlen}  % Enables outputting the current values of lengths

\usepackage[math]{blindtext}
\makeatletter
\def\blindtext@american{}
\renewcommand{\blindmathpaper}{%
	\blindtext
	\blindtext@formula\par
	\blindtext
	\blindtext@formula
	\blindtext
	\blindtext@formula\par
	\blindtext
	\blindtext@formula
	\blindtext
	\blindtext@formula\par
	\blindtext\relax%
}
\makeatother
\setcounter{blindtext}{1}
\setcounter{Blindtext}{1}
% The ``blindtext'' package does not recognize ``USenglish'' as identical to ``american''.
% Fix this -->
\LetLtxMacro{\blindtextblindtext}{\blindtext}
\LetLtxMacro{\blindtextBlindlist}{\Blindlist}
\LetLtxMacro{\blindtextBlindtext}{\Blindtext}
\RenewDocumentCommand{\blindtext}{O{\value{blindtext}}}{%
	\begingroup%
	\iflanguage{USenglish}{\selectlanguage{american}}{}%
	\blindtextblindtext[#1]%
	\endgroup%
}
\RenewDocumentCommand{\Blindtext}{O{\value{blindtext}} O{\value{Blindtext}}}{%
	\begingroup%
	\iflanguage{USenglish}{\selectlanguage{american}}{}%
	\blindtextBlindtext[#1][#2]%
	\endgroup%
}
\RenewDocumentCommand{\Blindlist}{m O{\value{blindlist}}}{%
	\begingroup%
	\iflanguage{USenglish}{\selectlanguage{american}}{}%
	\blindtextBlindlist{#1}[#2]%
	\endgroup%
}
% Based on https://tex.stackexchange.com/questions/299954/styling-blindtext-and-blindtext-aka-renewcommand-with-optional-arguments
% <==

%\usepackage{showframe}
%\renewcommand*\ShowFrameColor{\color{magenta}}
%\usepackage[pagewise]{lineno}
%\addtolength{\linenumbersep}{7.5pt}
%\renewcommand{\linenumberfont}{\sffamily\tiny\color{gray}}
%\linenumbers
% An auxiliary command to display the current font settings -->
\makeatletter
\newcommand{\showfont}{{%
	\color{magenta}
	\textit{Encoding:} \f@encoding{},
	\textit{family:}   \f@family{},
	\textit{series:}   \f@series{},
	\textit{shape:}    \f@shape{},
	\textit{size:}     \f@size{}.
}}
\newcommand{\showfamily}{\f@family{}}
\makeatother
% <--

\makeatletter
\newcommand*{\checkgreekletters}{%
	\@for\@tempa:=%
	alpha,beta,gamma,delta,epsilon,varepsilon,zeta,eta,theta,vartheta,iota,kappa,lambda,mu,nu,xi,%
	omicron,pi,varpi,rho,varrho,sigma,varsigma,tau,upsilon,phi,varphi,chi,psi,omega,digamma,%
	Alpha,Beta,Gamma,Delta,Epsilon,Zeta,Eta,Theta,Iota,Kappa,Lambda,Mu,Nu,Xi,%
	Omicron,Pi,Rho,Sigma,Tau,Upsilon,Phi,Chi,Psi,Omega,Digamma%
	\do{$\csname\@tempa\endcsname,$ }%
}
\makeatother

\usepackage{fonttable}
% Color slot numbers in \xfonttable gray instead of black -->
\makeatletter
\renewcommand*\f@placedecimal[2]{#1\ {\color{gray}\tiny #2}}
\renewcommand*{\f@toct}[1]{\hbox{\color{gray}\rmfamily\'{}\kern-.2em\itshape#1\/\kern.05em}} % octal constant
\renewcommand*{\f@thex}[1]{\hbox{\color{gray}\rmfamily\H{}\ttfamily#1}} % hexadecimal constant
\makeatother
% <--
\hypersetup{pdftitle={\disstitle}, pdfauthor={\dissauthor}}
% !TeX program = pdflatex
% !TeX TXS-program:compile = txs:///pdflatex/
% !TeX TS-program = pdflatex
% !BIB program = biber
% !TeX TXS-program:bibliography = txs:///biber




%%%%%%%%%%%%%%%%%%%%%%%%%%%%%%%%%%%%%%%%
%%  PAGE LAYOUT: ADDITIONAL SETTINGS  %%
%%%%%%%%%%%%%%%%%%%%%%%%%%%%%%%%%%%%%%%%


% Increase text width and text height by 5% to make the text
% flow exactly as in \documentclass[a4paper, 10pt]{article}
% ==>
\addtolength{\textheight}{0.05\textheight}
\addtolength{\textwidth}{0.05\textwidth}
\addtolength{\evensidemargin}{-0.05\textwidth}
% <==

\usepackage[sf, bf, raggedright, explicit]{titlesec}	% Sansserif font for the headings
% Page headers
% ==>
\addtolength{\topmargin}{-14.25pt}
\addtolength{\headsep}{14.25pt}
	% Move header up so that the distance to the body text increases;
	% it should be approximately as large as the largest space above a heading.
\newpagestyle{chapterleftsectionright}[\small\sffamily\selectfont]{
	\sethead%
	[\textbf{\thepage}~~|~~%
	 \ifthechapter{\thechapter~~}{}\ecapitalisewords{\chaptertitle}%
	 \strut%
	]% even-left
	[]% even-center
	[]% even-right
	{}% odd-left
	{}% odd-center
	{\ifx\sectiontitle\empty%
		\ifthechapter{\thechapter~~}{}\ecapitalisewords{\chaptertitle}%
	 \else%
		\ifthesection{\thesection~~}{}\ecapitalisewords{\sectiontitle}%
	 \fi%
	 ~~|~~\textbf{\thepage}%
	 \strut%
	}%	odd-right
}
\newpagestyle{chapterleftappendixsectionright}[\small\sffamily\selectfont]{
	\sethead%
	[\textbf{\thepage}~~|~~%
	 \ifthechapter{\thechapter~~}{}\ecapitalisewords{\chaptertitle}%
	 \strut%
	]% even-left
	[]% even-center
	[]% even-right
	{}% odd-left
	{}% odd-center
	{\ifx\sectiontitle\empty%
		\ifthechapter{\thechapter~~}{}\ecapitalisewords{\chaptertitle}%
	 \else%
		\appendixname~\ifthesection{\thesection~~}{}\ecapitalisewords{\sectiontitle}%
	 \fi%
	 ~~|~~\textbf{\thepage}%
	 \strut%
	}%	odd-right
}
\pagestyle{chapterleftsectionright}
\definecolor{almostwhite}{RGB}{254,254,254}
% Some printers skip completely empty pages. To avoid this, we put a tiny gray period in ``empty'' headers:
\renewpagestyle{empty}[\tiny\selectfont]{
	\sethead%
	[]% even-left
	[]% even-center
	[\textcolor{almostwhite}{.}]% even-right
	{\textcolor{almostwhite}{.}}% odd-left
	{}% odd-center
	{}% odd-right
}
\usepackage{emptypage}  % Suppresses page numbers and headings from appearing on empty pages.
\assignpagestyle{\chapter}{empty} % Make footer empty on starting page of a new chapter
% <==

% Remove indent from the list of figures and list of tables
% ==>
%\usepackage{titletoc}
%\newcommand{\tocdots}{\titlerule*{\hspace{.8pc}.\hspace{-.45pc}}}
%\contentsmargin{2.55em}
%\titlecontents{chapter}
%	[1.5em]
%	{\addvspace{1ex}}
%	{\large\sffamily\bfseries\contentslabel{1.5em}}{\hspace*{-1.5em}\large\sffamily\bfseries}  % The latter ``outdents'' unnumbered sections; large wbold and sans-serif chapter entries in ToC
%	{\large\sffamily\bfseries\titlerule*{}\contentspage[\thecontentspage\hspace{-.12em}]}
%\titlecontents{section}
%	[3.8em] % = 1.5em + 2.3em, 1.5em reserved for chapter numbering
%	{\addvspace{0.25ex}}
%	{\bfseries\contentslabel{2.3em}}{\hspace*{-2.3em}}  % The latter ``outdents'' unnumbered sections, bold section entries in ToC
%	{\titlerule*{}\contentspage}%{\tocdots\contentspage}
%\titlecontents{subsection}
%	[6.8em] % = 1.5em + 2.3em + 3em, 1.5em + 2.3em reserved for chapter and section numbering
%	{}{\contentslabel{3em}}{\hspace*{-3em}}  % The latter ``outdents'' unnumbered subsections
%	{\titlerule*{}\contentspage}%{\tocdots\contentspage}
%\newlength{\loflotleft} \setlength{\loflotleft}{3.5em}
%\titlecontents{figure}
%	[\loflotleft]
%	{}{\contentslabel{\loflotleft}}{}
%	{\titlerule*{}\contentspage}%{\tocdots\contentspage}
%\titlecontents{table}
%	[\loflotleft]
%	{}{\contentslabel{\loflotleft}}{}
%	{\titlerule*{}\contentspage}%{\tocdots\contentspage}
% <==

%% Remove the leader (dots) from the table of contents, see
%% https://tex.stackexchange.com/questions/73789/how-to-remove-the-dots-in-the-tableof-contents
%% ==>
%\makeatletter
%\renewcommand{\@dotsep}{10000} 
%\makeatother
%% <==

\usepackage[titles]{tocloft}
\renewcommand{\cftdotsep}{\cftnodots}  % Remove the leader (dots) from the table of contents
\renewcommand{\cftchapfont}{\large\sffamily\bfseries}
\renewcommand{\cftchappagefont}{\cftchapfont}
\apptocmd{\cftchapfillnum}{\nopagebreak}{}{}  % Prevent page breaks after chapter entry in table of contents
%\renewcommand{\cftsecfont}{\bfseries}  % Make section entries in the table of contents bold
%\renewcommand{\cftsecpagefont}{\cftsecfont}
\setlength{\cftbeforesecskip}{\smallskipamount}
\setlength{\cftfigindent}{0pt}  % Remove indentation from list of figures
\setlength{\cfttabindent}{0pt}  % Remove indentation from list of tables
\setlength{\cftfignumwidth}{3.75em}  % Make labels in LoF wider
\setlength{\cfttabnumwidth}{3.75em}  % Make labels in LoT wider
% Prevent unwanted optical marginal alignment of page numbers by microtype:
\newlength{\cftaux}
\renewcommand{\cftfigafterpnum}{\phantom{\,}\settowidth{\cftaux}{\,}\hspace{-\cftaux}}
\renewcommand{\cfttabafterpnum}{\phantom{\,}\settowidth{\cftaux}{\,}\hspace{-\cftaux}}

\setcounter{secnumdepth}{3}
	% So that also 3rd-level headings are numbered.
	% Note: chapter is level 0, hence \subsubsection is actually on the 4th level.

\newcommand{\TestForPunct}[1]{\iftextterm{#1}{#1}{#1\unskip.}}
	% Adds a period if no punctuation mark is present yet

\titleformat{\chapter}[display]
	{\sffamily\fontseries{medium}\huge}%\sffamily\bfseries\addfontfeature{Numbers=Lining}
	{\scalefont{0.95}\chaptertitlename~\thechapter}% {\thechapter}
	{1.5\bigskipamount}%
	{\sffamily\fontseries{l}\linespread{0.95}\Huge\raggedright%
	 \xcapitalisewords{%
		#1%
	 }%
	}

%\newlength{\appendixlabellength}
\newcommand{\appendixformat}[1][\currentname]{%
	\pagestyle{chapterleftappendixsectionright}%
	\titleformat{\section}[hang]
		{\raggedright\Large\sffamily\bfseries}{\appendixname~\thesection\quad}{0pt}{%
			\ecapitalisewords{%
				#1%
			}%
		}[]%
%	\setlength{\appendixlabellength}{\widthof{\textbf{\appendixname~}}+2.3em}%
%	\titlecontents{section}
%		[\appendixlabellength+1.5em] % = 1.5em + 2.3em, 1.5em reserved for chapter numbering
%		{\addvspace{0.25ex}}
%		{\bfseries\contentslabel[\appendixname~\thecontentslabel]{\appendixlabellength}}{}
%		{\titlerule*{}\contentspage}%{\tocdots\contentspage}
}
\AtBeginEnvironment{appendix}{\appendixformat}
\AtBeginEnvironment{appendices}{\appendixformat}
\AtBeginEnvironment{subappendices}{\appendixformat}
\AtEndEnvironment{appendix}{%
%	\pagestyle{chapterleftsectionright}%
	\thispagestyle{chapterleftappendixsectionright}%
}
\AtEndEnvironment{appendices}{%
%	\pagestyle{chapterleftsectionright}%
	\thispagestyle{chapterleftappendixsectionright}%
}
\AtEndEnvironment{subappendices}{%
%	\pagestyle{chapterleftsectionright}%
	\thispagestyle{chapterleftappendixsectionright}%
}

\titleformat{name=\section}[hang]
	{\raggedright\Large\sffamily\bfseries}{\thesection\quad}{0pt}{%
		\xcapitalisewords{%
			#1%
		}%
	}[]
\titleformat{name=\section, numberless}[hang]
	{\raggedright\Large\sffamily\bfseries}{}{0pt}{%
		\xcapitalisewords{%
			#1%
		}%
	}[]
\titleformat{name=\subsection}[hang]
	{\raggedright\large\sffamily\bfseries}{\thesubsection\quad}{0pt}{%
		\xcapitalisewords{%
			#1%
		}%
	}[]
\titleformat{name=\subsection, numberless}[hang]
	{\raggedright\large\sffamily\bfseries}{}{0pt}{%
		\xcapitalisewords{%
			#1%
		}%
	}[]
\titleformat{name=\subsubsection}[hang]
	{\raggedright\normalsize\sffamily\bfseries}{\thesubsubsection\quad}{0pt}{%
		\xcapitalisewords{%
			#1%
		}%
	}[]
\titleformat{name=\subsubsection, numberless}[hang]
	{\raggedright\normalsize\sffamily\bfseries}{}{0pt}{%
		\xcapitalisewords{%
			#1%
		}%
	}[]
\titleformat{name=\paragraph, numberless}[runin]
	{}{}{0pt}{%
		\textbf{%
			#1\iftextterm{#1}{}{\unskip.}%
		}%
	}[]
\titleformat{name=\subparagraph, numberless}[runin]
	{}{}{0pt}{%
		\textbf{%
			#1\iftextterm{#1}{}{\unskip.}%
		}%
	}[]
\titlespacing{\paragraph}
	{0pt}{0.5\baselineskip plus 0.5\baselineskip}{3\wordsep}[]
\titlespacing{\subparagraph}
	{\parindent}{0pt}{3\wordsep}[]

\AtBeginDocument{%
	\captionsetup{footfont={sf, footnotesize}}
		% Set font in \floatfoot to sans-serif and footnotesize
}

% \marginpar settings for the commenting functions
\setlength{\marginparwidth}{3.75cm}
\setlength{\marginparsep}{0.4cm}
\setlength{\marginparpush}{0.4cm}
% !TeX program = pdflatex
% !TeX TXS-program:compile = txs:///pdflatex/
% !TeX TS-program = pdflatex
% !BIB program = biber
% !TeX TXS-program:bibliography = txs:///biber




%%%%%%%%%%%%%%%%%%%%%%%%%%%%%%%%%%%%%%%%%%%%%%%%%
%%  GRID-BASED TYPESETTING AS FAR AS POSSIBLE  %%
%%%%%%%%%%%%%%%%%%%%%%%%%%%%%%%%%%%%%%%%%%%%%%%%%


\flushbottom

\newlength{\origbaselineskip}
\setlength{\origbaselineskip}{\baselineskip}

\ifdef{\linesperpagedesired}
	{}% If already defined, do nothing.
	{\newcommand{\linesperpagedesired}{42}}
	% Number of text lines per page, see https://en.wikipedia.org/wiki/Gutenberg_Bible

\newcommand{\linesperpagecurrent}{\numexpr (\textheight - \topskip) / \baselineskip + 1 \relax}  % Integer division!
\makeatletter
%\newcommand{\newbaselinestretch}{\strip@pt\dimexpr (\linesperpagecurrent pt - 1pt) / (\linesperpagedesired - 1)}
\newcommand{\newbaselinestretch}{1.1}
\makeatother
\linespread{\newbaselinestretch}
\newlength{\newbaselineskip}
\setlength{\newbaselineskip}{
	\dimexpr (\textheight - \topskip) / (\linesperpagedesired - 1)
}
\setlength{\textheight}{\dimexpr \numexpr \linesperpagedesired - 1 \relax \newbaselineskip + \topskip}  % to prevent too small \textheigtht due to rounding errors
\newlength{\newparindent}
\setlength{\newparindent}{1.15\newbaselineskip}%
\AtBeginDocument{%
	\setlength{\baselineskip}{\newbaselineskip}%
	\setlength{\parindent}{\newparindent}
	\setlength{\lineskiplimit}{0pt}
		% Prevents increased line spacing in response to spacious inline formulas.
}
\newlength{\abovedisplayauxskip}
\newlength{\belowdisplayauxskip}
\setlength{\abovedisplayauxskip}{0pt plus 0.5\baselineskip}
\setlength{\belowdisplayauxskip}{0pt plus 0.5\baselineskip}
\newcommand{\predisplaycmd}{%
	\ifvmode\else\unskip\fi%
	\nopagebreak[2]%
	\vspace{\abovedisplayauxskip}%
}
\makeatletter
\def\@itemize@name{itemize}
\def\@enumerate@name{enumerate}
\def\@description@name{description}
\newcommand{\postdisplaycmd}{%
	\ifx\@currenvir\@itemize@name%
	\else%
		\ifx\@currenvir\@enumerate@name%
		\else%
			\ifx\@currenvir\@description@name%
			\else%
				\vskip\belowdisplayauxskip%
			\fi%
		\fi%
	\fi%
	\noindent%
}
\makeatother
\AtBeginEnvironment {align}      {\predisplaycmd}
\AfterEndEnvironment{align}      {\postdisplaycmd}
\AtBeginEnvironment {align*}     {\predisplaycmd}
\AfterEndEnvironment{align*}     {\postdisplaycmd}
\AtBeginEnvironment {alignat}    {\predisplaycmd}
\AfterEndEnvironment{alignat}    {\postdisplaycmd}
\AtBeginEnvironment {alignat*}   {\predisplaycmd}
\AfterEndEnvironment{alignat*}   {\postdisplaycmd}
\AtBeginEnvironment {displaymath}{\predisplaycmd}
\AfterEndEnvironment{displaymath}{\postdisplaycmd}
\AtBeginEnvironment {eqnarray}   {\predisplaycmd}
\AfterEndEnvironment{eqnarray}   {\postdisplaycmd}
\AtBeginEnvironment {eqnarray*}  {\predisplaycmd}
\AfterEndEnvironment{eqnarray*}  {\postdisplaycmd}
\AtBeginEnvironment {equation}   {\predisplaycmd}
\AfterEndEnvironment{equation}   {\postdisplaycmd}
\AtBeginEnvironment {equation*}  {\predisplaycmd}
\AfterEndEnvironment{equation*}  {\postdisplaycmd}
\AtBeginEnvironment {flalign}    {\predisplaycmd}
\AfterEndEnvironment{flalign}    {\postdisplaycmd}
\AtBeginEnvironment {flalign*}   {\predisplaycmd}
\AfterEndEnvironment{flalign*}   {\postdisplaycmd}
\AtBeginEnvironment {gather}     {\predisplaycmd}
\AfterEndEnvironment{gather}     {\postdisplaycmd}
\AtBeginEnvironment {gather*}    {\predisplaycmd}
\AfterEndEnvironment{gather*}    {\postdisplaycmd}
\AtBeginEnvironment {multiline}  {\predisplaycmd}
\AfterEndEnvironment{multiline}  {\postdisplaycmd}
\AtBeginEnvironment {multiline*} {\predisplaycmd}
\AfterEndEnvironment{multiline*} {\postdisplaycmd}
% \setlength{\topskip}{\newbaselineskip pt}

\setlength{\parskip}{0pt}  % Prevents whitespace from being added between paragraphs
%\setlength{\parskip}{0pt plus 0.0001pt}
% Add whitepsace between paragraphs only in emergency cases




%%%%%%%%%%%%%%%%%%%%%%%%%%%%%%%%%
%%  FORMATTING OF MATHEMATICS  %%
%%%%%%%%%%%%%%%%%%%%%%%%%%%%%%%%%


%\setlength{\mathindent}{\newparindent}  % Only for option ``fleqn''

% Prevent stretching of spaces around operators in inline formulas; see ==>
%  http://tex.stackexchange.com/questions/83746/keeping-the-distance-between-mathematical-symbols-consistent
% \thinmuskip=3.0mu (already without glue)
\medmuskip=1\medmuskip	% Formerly 4.0mu plus 2.0mu minus 4.0mu -> 4.0mu
\thickmuskip=1\thickmuskip	% Formerly 5.0mu plus 5.0mu -> 5.0mu
% <==

% Increase spacing around relational symbols (<, =, >, \le, \ge, \equiv; +, -, \times, etc.) in display formulae ==>
\newcommand{\regularmu}{\thickmuskip= 5mu \medmuskip=4mu}
\newcommand{\thickmu}  {\thickmuskip=10mu \medmuskip=5mu}
\AtBeginEnvironment{align}      {\thickmu}
\AtBeginEnvironment{align*}     {\thickmu}
\AtBeginEnvironment{alignat}    {\thickmu}
\AtBeginEnvironment{alignat*}   {\thickmu}
\AtBeginEnvironment{displaymath}{\thickmu}
\AtBeginEnvironment{eqnarray}   {\thickmu}
\AtBeginEnvironment{eqnarray*}  {\thickmu}
\AtBeginEnvironment{equation}   {\thickmu}
\AtBeginEnvironment{equation*}  {\thickmu}
\AtBeginEnvironment{flalign}    {\thickmu}
\AtBeginEnvironment{flalign*}   {\thickmu}
\AtBeginEnvironment{gather}     {\thickmu}
\AtBeginEnvironment{gather*}    {\thickmu}
\AtBeginEnvironment{multiline}  {\thickmu}
\AtBeginEnvironment{multiline*} {\thickmu}
% <==

% Use the bm (= boldmath) package for better support of setting math in bold ==>
% Prevent the "Too many math fonts used" error:
\newcommand{\bmmax}{0}
\newcommand{\hmmax}{0}
\usepackage{bm}
% <==

% Allow paragraph breaks after a display equation
\makeatletter
\predisplaypenalty=\@medpenalty
\postdisplaypenalty=0
\makeatother




%%%%%%%%%%%%%%%%%%%%%%%%%%%%%
%%  LAYOUT AND SECTIONING  %%
%%%%%%%%%%%%%%%%%%%%%%%%%%%%%

% Disable single lines that start a~paragraph at the end of a~page (widows/Schusterjungen)
% and disable single lines at the end of a~paragraph that start a~new page (orphans/Hurenkinder):
\usepackage[all]{nowidow}

\AtBeginDocument{%
	% Reduce the amount of white space after a period (and enforce this throughout the entire document, because babel's \selectlanguage{...} resets \frenchspacing):
	\let\nonfrenchspacing=\frenchspacing
	\frenchspacing
	\sloppy%  % Prevent overfull hboxes (at the expense of more uneven whitespace)
}

\ifxetex
	\usepackage[protrusion=true, expansion=false]{microtype}
\else
	\usepackage[protrusion=true, expansion=false, kerning=true]{microtype}
\fi

% Reduce by how much an interword space can shrink; see
% http://tex.stackexchange.com/questions/19236/how-to-change-the-interword-spacing,
% http://tex.stackexchange.com/questions/88991/what-do-different-fontdimennum-mean, and
% http://mirrors.ctan.org/macros/latex/contrib/everysel/everysel.pdf:
\AddToHook{selectfont}{%
	\spaceskip = \fontdimen2\font plus \fontdimen3\font minus 0.75\fontdimen4\font%
}

% Let even relatively big floats (long tables, spacious figures) be placed on text pages ==>
\renewcommand{\textfraction}{0.05}
\renewcommand{\topfraction}{0.95}
\renewcommand{\bottomfraction}{0.95}
% <== See http://tex.stackexchange.com/questions/39017/how-to-influence-the-position-of-float-environments-like-figure-and-table-in-lat/39020#39020

% Set margins of block quotes
% ==>
\renewenvironment{quote}%
  {\list{}{\leftmargin=\parindent \rightmargin=\parindent}%
   \linespread{\newbaselinestretch}\item[]\itshape\small}%
  {\endlist}
\renewenvironment{quotation}%
  {\list{}{\leftmargin=\parindent \rightmargin=\parindent
           \listparindent=\parindent \parsep=0pt}%
   \item[]}%
  {\endlist}
% <==

\usepackage{csquotes}

\usepackage{nameref}
\makeatletter
\newcommand*{\currentname}{\@currentlabelname}
\makeatother




%%%%%%%%%%%%%%%%%%%%%%%%%%%%%%%
%%  FORMATTING OF FOOTNOTES  %%
%%%%%%%%%%%%%%%%%%%%%%%%%%%%%%%


\usepackage[multiple, bottom, norule, splitrule, marginal]{footmisc}
% \renewcommand{\multfootsep}{,\,}
% AER-like style: no regular footnote rule, half-page splitrule
% ==>
\setlength{\footnotemargin}{0.85\newparindent}%
\renewcommand{\footnotelayout}{\hspace{0.15\newparindent}}
\renewcommand{\hangfootparindent}{\newparindent}
\preto{\footnote}{\setlength{\parindent}{\newparindent}}
% Add some space around the footnoterule:
\let\oldfootnoterule\footnoterule
\addtolength{\skip\footins}{\bigskipamount}
\AtBeginDocument{%
	\renewcommand{\splitfootnoterule}{{\hrule width 0.5\textwidth}}%
	\renewcommand{\footnoterule}{\oldfootnoterule\medskip}%
}
% <==
%\setlength{\footnotesep}{0.8\baselineskip}

% Chicago Manual of Style (16th edition):
% ``Note reference numbers in text are set as superior (superscript) numbers.
% In the notes themselves, they are normally full size, not raised, and followed by a period.''
% (also REStud and JEEA style)
% ==>
\usepackage{xstring}
\newlength{\textparindent}
\setlength{\textparindent}{\parindent}
\newlength{\templength}
\makeatletter
\let \@makefntextorig \@makefntext
    % Saving the original definition so we can reuse it if necessary.
\newcommand{\@makefntextcustom}[1]{%
	\parindent 2\textparindent%
	\hspace{-\textparindent}%
	\settowidth{\templength}{0}%
	\ifnum\value{footnote}<10 \hspace{\templength}\else\fi%
	\thefootnote.\enskip #1%
}
\renewcommand{\@makefntext}[1]{\@makefntextcustom{#1}}
\makeatother
%\usepackage{scrextend}
%\newlength{\footnoteflmargin}
%\setlength{\footnoteflmargin}{\parindent}
%% \deffootnote[\footnoteflmargin]{0pt}{0pt}{\thefootnotemark.\:\,}
%\deffootnote[\footnoteflmargin]{0pt}{\footnoteflmargin}{}
%\renewcommand{\footnote}[1]{%
%	\footnoteorig{%
%		\ifnum\value{footnote}<10 \phantom{0}\else\fi%
%		\thefootnotemark.\enskip%
%		#1%
%	}%
%}
% <==
\let \thefootnoteorig \thefootnote
\DefineFNsymbols*{star}{%
	{$\mathrm{\star}$}{$\mathrm{\star\star}$}{$\mathrm{\ddagger}$}%
	{$\mathrm{\ddagger\ddagger}$}{\S}{\S\S}{\P}{\P\P}{$\mathrm{\|}$}{$\mathrm{\|\|}$}%
}
\setfnsymbol{star}




%%%%%%%%%%%%%%%%%%%%%%%%%%
%%  FIGURES AND TABLES  %%
%%%%%%%%%%%%%%%%%%%%%%%%%%


\usepackage[singlelinecheck=on]{caption}
\DeclareCaptionLabelSeparator{periodlargespace}{.\:\:}
\captionsetup{
	singlelinecheck=on,
	figureposition=below,
	tableposition=above,
	format = plain,
	labelsep = periodlargespace,
	margin = 0pt,
	font = {sf, small},
	labelfont = {sf, bf, small},
	justification = justified
}

% Referencing ``subfigures'' (i.e., individual panels of which a figure is comprised):
%\usepackage{subfigure}
%\usepackage{subfig}
% Justus says this is the newer and preferable package.
% However, see https://tex.stackexchange.com/questions/13625/subcaption-vs-subfig-best-package-for-referencing-a-subfigure:
\usepackage{subcaption}

% Packages for creating better-looking tables
\usepackage{booktabs}
\setlength{\cmidrulewidth}{.035em}
\setlength{\lightrulewidth}{.035em}
\setlength{\heavyrulewidth}{.09em}
\setlength{\abovetopsep}{-5pt}
\addtolength{\aboverulesep}{1.5pt}	% Make tables a little more spacious
\addtolength{\belowrulesep}{1.5pt}
% \setlength{\belowbottomsep}{-2pt}

\usepackage{tabularx}	% Provides environment tabularx to adjust width of tables
% \usepackage{tabulary}	% For some reason, tabulary doesn't obey the width argument ...
% Emulate the "tabulary" column types:
\newcolumntype{C}{>{\centering\arraybackslash}X}
\newcolumntype{J}{>{\arraybackslash}X}
\newcolumntype{L}{>{\RaggedRight\arraybackslash}X}
\newcolumntype{R}{>{\RaggedLeft\arraybackslash}X}
% \usepackage[flushleft]{threeparttable}	% Provides the tablenotes environment

% Remove superfluous whitespace at beginning and end of table rows -->
\LetLtxMacro{\oldtabular}{\tabular}
\LetLtxMacro{\endoldtabular}{\endtabular}
\RenewDocumentEnvironment{tabular}{O{c} m}{%
	\oldtabular[#1]{@{}#2@{}}%
}{%
	\endoldtabular%
}
\LetLtxMacro{\oldtabularx}{\tabularx}
\LetLtxMacro{\endoldtabularx}{\endtabularx}
\RenewDocumentEnvironment{tabularx}{m O{c} m}{%
	\oldtabularx{#1}[#2]{@{}#3@{}}%
}{%
	\endoldtabularx%
}

% Package for long tables that span multiple pages
\usepackage{xltabular}

\usepackage{siunitx}[=v2]
	% Allows, among others, for alignment of decimal numbers in tables at the decimal point.
\sisetup{
	detect-all,
	round-integer-to-decimal = true,
	group-digits             = true,
	group-minimum-digits     = 5,
	group-separator          = {\kern 1pt},
	table-align-text-pre     = false,
	table-align-text-post    = false,
	input-signs              = + -,
	input-symbols            = {*} {**} {***} \sigstar,
	input-open-uncertainty 	 = ,
	input-close-uncertainty  = ,
	retain-explicit-plus
}
\newcolumntype{T}[1]{@{}S[table-format = #1, table-space-text-pre = {***}, table-space-text-post = {***}]}
% Fix incompatibility of siunitx (v2018-05-17) with FiraSans (v2019-06-06),
% based on https://tex.stackexchange.com/questions/213605/siunitx-does-not-detect-semi-bold-font
% ==>
\ExplSyntaxOn\makeatletter
\newcommand{\thisseries}{\f@series}
\cs_set_protected:Npn \__siunitx_detect_font_weight_text: {%
	\let\origmdseries\mdseries@sf%
	\let\origbfseries\bfseries@sf%
	\let\currentseries\f@series%
	\edef\XcurrentseriesX{/\f@series/}%
		% Store the current fontseries but enclose it in some kind of delimiter,
		% because otherwise one-letter fontseries may erroneously triger \boldmath
		% (for instance, ``m'' is contained in ``semibold'')
	% Use \boldmath for any weight above semibold:
	\tl_if_in:noTF
		{ /sb/ /b/ /bx/ /eb/ /ub/ /bold/ /extrabold/ /ultrabold/ /heavy/ /black/ /demibold/ /semibold/ }
		{ \XcurrentseriesX }
		{% if included in the above list: switch to \mathversion{bold}
			\cs_set:Nn \__siunitx_font_weight: {%
				\boldmath%
				\let\bfseries@sf\currentseries%
					% necessary because siunitx in some way accesses
					% \bfseries@sf from the mweights package
				\fontseries{\bfseries@sf}\selectfont%
			}%
			\let\bfseries@sf\origbfseries%  % restore \bfseries@sf
		}
		{% if not: use \mathversion{normal}
			\cs_set:Nn \__siunitx_font_weight: {%
				\unboldmath
				\let\mdseries@sf\currentseries%
					% necessary because siunitx in some way accesses
					% \bfseries@sf from the mweights package
				\fontseries{\mdseries@sf}\selectfont%
			}%
			\let\mdseries@sf\origmdseries%  % restore \mdseries@sf
		}%
}
\makeatother\ExplSyntaxOff
% <==

% Ability to add footnotes to tables:
% ==>
\usepackage[restart, breakwithin, indentafter]{parnotes}
	% BEWARE: For some reason, \parnotes removes the vertical space before the following section/the indent of the following paragraph if not included in the table itself!
\renewcommand{\parnotevskip}{0pt}
% \renewcommand{\parnoteintercmd}{\\}
\renewcommand{\theparnotemark}{\alph{parnotemark}}
\renewcommand{\parnotefmt}[1]{\footnotesize\noindent\justify #1\par}
%\renewcommand{\parnotefmt}[1]{\footnotesize\rmfamily%
%	\noindent\rule{\linewidth}{1pt}\\%
%	\noindent#1\par%
%	\noindent\rule{\linewidth}{1pt}\\%
%}
%\renewcommand{\parnoteintercmd}{\;$\bullet$\;}
% <==

\usepackage{makecell}

\usepackage{longtable}

\usepackage{multirow}

% Redefine figure environment so that all figures are centered
% ==>
\makeatletter
\let\oldfigure\figure
\def\figure{\@ifnextchar[\figure@i \figure@ii}
\def\figure@i[#1]{\oldfigure[#1]\centering}
\def\figure@ii{\oldfigure\centering}
\makeatother
% <==

\usepackage[font={sf, small}]{floatrow}	% Set the font in tables to sansserif small
\floatsetup[table]{style=Plaintop}
\renewcommand{\floatfootskip}{\smallskipamount}
\newcommand{\tablenotes}[2][Notes:]{%
	\floatfoot*{%
		\setlength{\baselineskip}{11pt}%
		\textit{#1} #2%
	}%
}
\newcommand{\figurenotes}[2][Notes:]{%
	\floatfoot{%
		\setlength{\baselineskip}{11pt}%
		\vspace{-\floatfootskip}%
		\vspace{\medskipamount}%
		\\[-\baselineskip]
		\textit{#1} #2%
	}%
}
% \usepackage{float}	% Allows the inclusion of figures inside a minipage	% Do not use in combination with floatrow.
\usepackage{placeins} % improve placing of floats (figures, tables), provides \FloatBarrier

% Adjust the minimum distance between a float (figure, table) and the body text:
\setlength{\textfloatsep}{1.5\newbaselineskip plus 0.5\newbaselineskip minus 0.0pt}
% originally, \textfloatsep: 20.0pt plus 2.0pt minus 4.0pt

\usepackage{verbatim}

\usepackage{longtable}

%\usepackage{setspace}

%\DeclareCaptionLabelFormat{REStudTable}{\MakeUppercase{\tablename}~#2}
%\DeclareCaptionTextFormat{REStudTableT}{\textit{#1}}
%\captionsetup[table]{labelformat=REStudTable, textformat=REStudTableT}




%%%%%%%%%%%%%%%%%%%%%%%%%%%
%%  FORMATTING OF LISTS  %%
%%%%%%%%%%%%%%%%%%%%%%%%%%%


\usepackage[inline]{enumitem}
% General settings:
\setlist{leftmargin=\parindent, listparindent=\parindent, itemsep=\smallskipamount, parsep=0pt}
%\setlist[1]{topsep=\medskipamount, partopsep=0pt}
% Type-specific settings
%\setlist[enumerate]{labelwidth=\parindent, labelindent=0pt, labelsep=!, align=left}
\setlist[enumerate]{leftmargin=\parindent, labelsep=*}
	% Fine as long as the list does not include more than 9 items.
\setlist[enumerate, 1]{label=(\arabic*), labelindent=-0.5pt}
\setlist[enumerate, 2]{label=\alph*., align=right}
	% Taken from the Chicago Manual of Style (16th ed., Section 6.126)
\setlist[enumerate, 3]{label=\roman*., align=right, widest*=3, labelsep=0.3\parindent}
\setlist[itemize]{labelsep=0.435\parindent}
\setlist[description]{font=\rmfamily\normalsize}




%%%%%%%%%%%%%%%%%%%%%%%%%%%%%%
%%  FORMATTING OF THEOREMS  %%
%%%%%%%%%%%%%%%%%%%%%%%%%%%%%%


\newtheoremstyle{Standard}% name
	{\topsep}    % Space above: Use \topsep to make the space identical to the one around lists
	{\topsep}    % Space below
	{\itshape}   % Body font
	{}           % Indent amount (empty = no indent, \parindent = paragraph indent)
	{\bfseries}  % Theorem head font
	{.}          % Punctuation after theorem head
	{.5em}       % Space after theorem head: " " = normal interword space; \newline = linebreak
	{\thmname{#1}\thmnumber{\:#2}\thmnote{\bfseries\upshape\ (#3)}}
		% Theorem head spec (changed such that also the ``theorem note'' is printed in boldface)

\theoremstyle{Standard}
\newtheorem{theorem}{Theorem}
\newtheorem{corollary}[theorem]{Corollary}
\newtheorem{lemma}[theorem]{Lemma}
\newtheorem{proposition}[theorem]{Proposition}
\newtheorem{hypothesis}{Hypothesis}
\newtheorem{result}{Result}

\theoremstyle{definition}
\newtheorem{definition}[theorem]{Definition}
\newtheorem{example}[theorem]{Example}
\newtheorem{conjecture}[theorem]{Conjecture}

% Make numbering chapter-specific if we are compiling the dissertation template:
\makeatletter
\@ifclassloaded{book}{%
	\numberwithin{theorem}{chapter}%
	%\numberwithin{corollary}{chapter}%
	%\numberwithin{lemma}{chapter}%
	%\numberwithin{proposition}{chapter}%
	\numberwithin{hypothesis}{chapter}%
	\numberwithin{result}{chapter}%
	%\numberwithin{definition}{chapter}%
	%\numberwithin{example}{chapter}%
	%\numberwithin{conjecture}{chapter}%
}
\makeatother



%%%%%%%%%%%%%%%%%%%%%%%%%%%%%%%%%%%%%%%%%%%%%%%%%%%%%%%%%
%%  AUTOMATIC CAPITALIZATION OF HEADINGS AND CAPTIONS  %%
%%%%%%%%%%%%%%%%%%%%%%%%%%%%%%%%%%%%%%%%%%%%%%%%%%%%%%%%%


\usepackage{mfirstuc-english}
% \gMFUnocap{xxx} adds ``xxx'' to the words not to be capitalized
% Do not capitalize prepositions:
\gMFUnocap{about}
\gMFUnocap{at}
\gMFUnocap{against}
\gMFUnocap{around}
\gMFUnocap{between}
\gMFUnocap{by}
\gMFUnocap{from}
\gMFUnocap{on}
\gMFUnocap{over}
\gMFUnocap{per}
\gMFUnocap{to}
\gMFUnocap{versus}
\gMFUnocap{vs.}
\gMFUnocap{vis-\`a-vis}
\gMFUnocap{within}
\gMFUnocap{without}

\ifcase 0
	% 0 for disabling auto-capitalization, 1 for enabling auto-capitalization of headings, 2 for headings + figure/table captions
	% Case 0: Make capitalization commands ineffective
	\renewcommand{\capitalisewords}[1]{#1}
	\renewcommand{\ecapitalisewords}[1]{#1}
	\renewcommand{\xcapitalisewords}[1]{#1}
\or
	% Case 1:
	% Add auto-capitalization to table of contents -->
	\let\SavedContentsline\contentsline
	\renewcommand{\contentsline}[4]{%
		\SavedContentsline{#1}{\capitalisewords{#2}}{#3}{#4}%
	}
	% <--
\else
	% Case 2: Also auto-capitalize figure and table captions:
	% Add auto-capitalization to table of contents -->
	\let\SavedContentsline\contentsline
	\renewcommand{\contentsline}[4]{%
		\SavedContentsline{#1}{\capitalisewords{#2}}{#3}{#4}%
	}
	% <--
	\LetLtxMacro{\SavedCaption}{\caption}
	\RenewDocumentCommand{\caption}{ O{\shortcaption} m }{%
		\def\shortcaption{%
			\xcapitalisewords{%
				% \TestForPunct{%
				#2%
				% }%
			}%
		}%
		\SavedCaption[#1]{%
			\xcapitalisewords{%
				% \TestForPunct{%
				#2%
				% }%
			}%
		}%
	}
\fi




%%%%%%%%%%%%%%%%%%%%%%
%%  APPENDIX STYLE  %%
%%%%%%%%%%%%%%%%%%%%%%


\usepackage[title, titletoc]{appendix}  % allows, e.g., for appendices within chapters
% Patch a bug in appendix.sty,
% see https://github.com/wspr/herries-press/issues/34#issuecomment-935770824:
\makeatletter
\patchcmd\@resets@pp{\xdef}{\def}{}{\fail}
\patchcmd\@resets@ppsub{\xdef}{\def}{}{\fail}
\makeatother
\usepackage{chngcntr}  % to innclude section numbers/letters in the figure/table/equation counters

\AtBeginEnvironment{appendices}{%
	\counterwithin{figure}{section}%
	\counterwithin{table}{section}%
	\counterwithin{equation}{section}%
}
\AtBeginEnvironment{subappendices}{%
	\counterwithin{figure}{section}%
	\counterwithin{table}{section}%
	\counterwithin{equation}{section}%
}

% Revoke the changes at the end of the (sub)appendices environment
% if we are compiling the dissertation template:
\makeatletter
\@ifclassloaded{book}{%
	\AfterEndEnvironment{appendices}{%
		\counterwithout{figure}{section}%
		\counterwithin{figure}{chapter}%
		\counterwithout{table}{section}%
		\counterwithin{table}{chapter}%
		\counterwithout{equation}{section}%
		\counterwithin{equation}{chapter}%
	}%
	\AfterEndEnvironment{subappendices}{%
		\counterwithout{figure}{section}%
		\counterwithin{figure}{chapter}%
		\counterwithout{table}{section}%
		\counterwithin{table}{chapter}%
		\counterwithout{equation}{section}%
		\counterwithin{equation}{chapter}%
	}%
}{}
\makeatother
\subimport{0_0_Preamble/}{Preamble_Fonts_Charter_FiraSans}
% !TeX program = pdflatex
% !TeX TXS-program:compile = txs:///pdflatex/
% !TeX TS-program = pdflatex
% !BIB program = biber
% !TeX TXS-program:bibliography = txs:///biber




%%%%%%%%%%%%%%%%%%%%%%%%%%%%%%%%%%%%%%%%%%%%%%%%%
%%  SANS-SERIF MATH IN SANS-SERIF ENVIRONMENT  %%
%%%%%%%%%%%%%%%%%%%%%%%%%%%%%%%%%%%%%%%%%%%%%%%%%


% See https://tex.stackexchange.com/questions/41497/how-to-typeset-some-text-including-math-content-in-sans-serif
% See https://tex.stackexchange.com/questions/33165/make-mathfont-respect-the-surrounding-family

% Necessary for use of kpfonts
% ==>
\makeatletter
\newif\ifkp@upRm
\newif\ifkp@osm
\newif\ifkp@vosm
\makeatother
% <==

\DeclareMathVersion{normalup}
\DeclareMathVersion{boldup}
\DeclareMathVersion{sans}

%\SetSymbolFont{operators}{sans}{OT1}{jkpss}{m}{n}
%	% From http://mirrors.ctan.org/fonts/kpfonts/latex/kpfonts.sty
\SetSymbolFont{operators}   {sans}{OT1}{mdbch}{m}{n}
\SetSymbolFont{letters}     {sans}{OML}{jkpss}{m}{it}
	% From http://mirrors.ctan.org/fonts/kpfonts/latex/kpfonts.sty
%\SetSymbolFont{letters}     {sans}{OML}{cmbrm}{m}{it}
%\SetSymbolFont{symbols}     {sans}{OMS}{cmbrs}{m}{n}
\SetSymbolFont{symbols}     {sans}{OMS}{jkp}  {m}{n}
	% From http://mirrors.ctan.org/fonts/kpfonts/latex/kpfonts.sty
\DeclareSymbolFont{extrasymbols}  {OMS}{cmbrs}{m}{n}
\SetSymbolFont{extrasymbols}{sans}{OMS}{cmbrs}{m}{n}
	% Some symbols (e.g., \prime) look weird in kpfonts.
	% This provides the option to replace them by symbols from mathdesign-charter.

\SetMathAlphabet{\mathit} {sans}{T1}{\savesffamily}{\savesfmdseries}{it}
\SetMathAlphabet{\mathbf} {sans}{T1}{\savesffamily}{\savesfbfseries}{n}
\SetMathAlphabet{\mathtt} {sans}{OT1}{cmtl}{m}{n}
\SetMathAlphabet{\mathcal}{sans}{OMS}{ntxsy}{m}{n}
	% See https://tex.stackexchange.com/questions/231583/import-mathcal-symbols-from-txfonts
%\SetSymbolFont{largesymbols}{sans}{OMX}{jkpss}{m}{n}
%	% From http://mirrors.ctan.org/fonts/kpfonts/latex/kpfonts.sty
\SetSymbolFont{largesymbols} {sans}{OMX}{mdbch}{m}{n}
	% Using symbols like \int, \left(, etc. from mathdesign-charter because they look better than the ones included in kpfonts

\DeclareMathVersion{sansup}
\SetSymbolFont{letters}  {sansup}{OML}{jkpss}{m}{it}
\SetSymbolFont{symbols}  {sansup}{OMS}{jkp}  {m}{n}

\DeclareMathVersion{boldsans}
%\SetSymbolFont{operators}{boldsans}{OT1}{jkpss}{b}{n}
%	% From http://mirrors.ctan.org/fonts/kpfonts/latex/kpfonts.sty
\SetSymbolFont{operators}{boldsans}{OT1}{mdbch}{bx}{n}
\SetSymbolFont{letters}  {boldsans}{OML}{jkpss}{bx}{it}
	% From http://mirrors.ctan.org/fonts/kpfonts/latex/kpfonts.sty
%\SetSymbolFont{letters}  {boldsans}{OML}{mdbch}{bx}{it}
%\SetSymbolFont{letters}{boldsans}{OML}{cmbrm}{b}{it}
\SetSymbolFont{symbols}  {boldsans}{OMS}{jkp}  {bx}{n}
	% From http://mirrors.ctan.org/fonts/kpfonts/latex/kpfonts.sty
%\SetMathAlphabet{\mathrm}{boldsans}{OT1}{\savesffamily}{\savesfbfseries}{n}
\SetMathAlphabet{\mathit} {boldsans}{T1}{\savesffamily}{\savesfbfseries}{it}
\SetMathAlphabet{\mathtt} {boldsans}{T1}{cmtl}{b}{n}
\SetMathAlphabet{\mathcal}{boldsans}{OMS}{ntxsy}{b}{n}
%\SetSymbolFont{largesymbols}{boldsans}{OMX}{jkpss}{bx}{n}
%	% From http://mirrors.ctan.org/fonts/kpfonts/latex/kpfonts.sty
\SetSymbolFont{largesymbols}{boldsans}{OMX}{mdbch}{bx}{n}
	% Using symbols like \int, \left(, etc. from mathdesign-charter because they look better than the ones included in kpfonts

\DeclareMathVersion{boldsansup}
\SetSymbolFont{letters}{boldsansup}{OML}{jkpss}{bx}{it}
\SetSymbolFont{symbols}{boldsansup}{OMS}{jkp}  {bx}{n}

% Adapted from mdbch.sty
\SetSymbolFont{operators}     {boldup}{OT1}{mdbch}{b}{n}%
\SetSymbolFont{letters}       {boldup}{OML}{mdbch}{b}{n}%{it}%
\SetSymbolFont{lettersupright}{boldup}{OML}{mdbch}{b}{n}%
\SetSymbolFont{symbols}       {boldup}{OMS}{mdbch}{b}{n}%
\SetSymbolFont{largesymbols}  {boldup}{OMX}{mdbch}{b}{n}%

% Using glyphs for math mode from the custom sansserif font
\DeclareSymbolFont{uprightglyphs}{T1}{\savermfamily}{\savermmdseries}{n}
\SetSymbolFont{uprightglyphs}{normal}    {T1}{\savermfamily}{\savermmdseries}{n}
\SetSymbolFont{uprightglyphs}{normalup}  {T1}{\savermfamily}{\savermmdseries}{n}
\SetSymbolFont{uprightglyphs}{bold}      {T1}{\savermfamily}{\savermbfseries}{n}
\SetSymbolFont{uprightglyphs}{boldup}    {T1}{\savermfamily}{\savermbfseries}{n}
\SetSymbolFont{uprightglyphs}{sans}      {T1}{\savesffamily}{\savesfmdseries}{n}
\SetSymbolFont{uprightglyphs}{sansup}    {T1}{\savesffamily}{\savesfmdseries}{n}
\SetSymbolFont{uprightglyphs}{boldsans}  {T1}{\savesffamily}{\savesfbfseries}{n}
\SetSymbolFont{uprightglyphs}{boldsansup}{T1}{\savesffamily}{\savesfbfseries}{n}
\DeclareSymbolFont{italicglyphs} {T1}{\savermfamily}{\savermmdseries}{it}
\SetSymbolFont{italicglyphs} {normal}    {T1}{\savermfamily}{\savermmdseries}{it}
\SetSymbolFont{italicglyphs} {normalup}  {T1}{\savermfamily}{\savermmdseries}{n}
\SetSymbolFont{italicglyphs} {bold}      {T1}{\savermfamily}{\savermbfseries}{it}
\SetSymbolFont{italicglyphs} {boldup}    {T1}{\savermfamily}{\savermbfseries}{n}
\SetSymbolFont{italicglyphs} {sans}      {T1}{\savesffamily}{\savesfmdseries}{it}
\SetSymbolFont{italicglyphs} {sansup}    {T1}{\savesffamily}{\savesfmdseries}{n}
\SetSymbolFont{italicglyphs} {boldsans}  {T1}{\savesffamily}{\savesfbfseries}{it}
\SetSymbolFont{italicglyphs} {boldsansup}{T1}{\savesffamily}{\savesfbfseries}{n}

% Syntax of \DeclareMathSymobl:
% \DeclareMathSymbol {<symbol>} {<type>} {<sym-font>} {<slot>}
% Type              Meaning	            Example
% 0 or \mathord     Ordinary             $\alpha$
% 1 or \mathop      Large operator       $\sum$
% 2 or \mathbin     Binary operation     $\times$
% 3 or \mathrel     Relation             $\leq$
% 4 or \mathopen    Opening              $\langle$
% 5 or \mathclose   Closing              $\rangle$
% 6 or \mathpunct   Punctuation          ;
% 7 or \mathalpha   Alphabet character   A
% Example declaration:
% \DeclareMathSymbol{b}{0}{letters}{`b}

% Digits
\DeclareMathSymbol{0}{\mathalpha}{uprightglyphs}{`0}
\DeclareMathSymbol{1}{\mathalpha}{uprightglyphs}{`1}
\DeclareMathSymbol{2}{\mathalpha}{uprightglyphs}{`2}
\DeclareMathSymbol{3}{\mathalpha}{uprightglyphs}{`3}
\DeclareMathSymbol{4}{\mathalpha}{uprightglyphs}{`4}
\DeclareMathSymbol{5}{\mathalpha}{uprightglyphs}{`5}
\DeclareMathSymbol{6}{\mathalpha}{uprightglyphs}{`6}
\DeclareMathSymbol{7}{\mathalpha}{uprightglyphs}{`7}
\DeclareMathSymbol{8}{\mathalpha}{uprightglyphs}{`8}
\DeclareMathSymbol{9}{\mathalpha}{uprightglyphs}{`9}
% Operators and punctuation
\DeclareMathSymbol{+}{\mathbin}  {operators}    {`+}
	% Not from uprightglyphs due to bad spacing
\DeclareMathSymbol{=}{\mathrel}  {operators}    {`=}
	% Not from uprightglyphs due to bad spacing
\DeclareMathSymbol{.}{\mathord}  {uprightglyphs}{`.}
\DeclareMathSymbol{,}{\mathpunct}{uprightglyphs}{`,}
\DeclareMathSymbol{;}{\mathpunct}{uprightglyphs}{`;}
\DeclareMathSymbol{/}{\mathord}  {uprightglyphs}{`/}
%\DeclareMathSymbol{/}{\mathop}   {uprightglyphs}{`/}
%	% This would icrease the spacing around the division slash slightly
%\DeclareMathSymbol{(}{\mathopen} {uprightglyphs}{`(}
%\DeclareMathSymbol{)}{\mathclose}{uprightglyphs}{`)}
%\DeclareMathSymbol{[}{\mathopen} {uprightglyphs}{`[}
%\DeclareMathSymbol{]}{\mathclose}{uprightglyphs}{`]}
\DeclareMathSymbol{\prime}{\mathord}{extrasymbols}{"30}
	% Use \prime from mathdesign-charter because it looks better than the one in kpfonts
\DeclareMathDelimiter{(}      {\mathopen} {uprightglyphs}{`(} {largesymbols}{"00}
\DeclareMathDelimiter{)}      {\mathclose}{uprightglyphs}{`)} {largesymbols}{"01}
\DeclareMathDelimiter{[}      {\mathopen} {uprightglyphs}{`[} {largesymbols}{"02}
\DeclareMathDelimiter{]}      {\mathclose}{uprightglyphs}{`]} {largesymbols}{"03}
\DeclareMathDelimiter{\lbrace}{\mathopen} {uprightglyphs}{`\{}{largesymbols}{"08}
\DeclareMathDelimiter{\rbrace}{\mathclose}{uprightglyphs}{`\}}{largesymbols}{"09}
% Uppercase Latin characters
\DeclareMathSymbol{A}{\mathalpha}{italicglyphs}{`A}
\DeclareMathSymbol{B}{\mathalpha}{italicglyphs}{`B}
\DeclareMathSymbol{C}{\mathalpha}{italicglyphs}{`C}
\DeclareMathSymbol{D}{\mathalpha}{italicglyphs}{`D}
\DeclareMathSymbol{E}{\mathalpha}{italicglyphs}{`E}
\DeclareMathSymbol{F}{\mathalpha}{italicglyphs}{`F}
\DeclareMathSymbol{G}{\mathalpha}{italicglyphs}{`G}
\DeclareMathSymbol{H}{\mathalpha}{italicglyphs}{`H}
\DeclareMathSymbol{I}{\mathalpha}{italicglyphs}{`I}
\DeclareMathSymbol{J}{\mathalpha}{italicglyphs}{`J}
\DeclareMathSymbol{K}{\mathalpha}{italicglyphs}{`K}
\DeclareMathSymbol{L}{\mathalpha}{italicglyphs}{`L}
\DeclareMathSymbol{M}{\mathalpha}{italicglyphs}{`M}
\DeclareMathSymbol{N}{\mathalpha}{italicglyphs}{`N}
\DeclareMathSymbol{O}{\mathalpha}{italicglyphs}{`O}
\DeclareMathSymbol{P}{\mathalpha}{italicglyphs}{`P}
\DeclareMathSymbol{Q}{\mathalpha}{italicglyphs}{`Q}
\DeclareMathSymbol{R}{\mathalpha}{italicglyphs}{`R}
\DeclareMathSymbol{S}{\mathalpha}{italicglyphs}{`S}
\DeclareMathSymbol{T}{\mathalpha}{italicglyphs}{`T}
\DeclareMathSymbol{U}{\mathalpha}{italicglyphs}{`U}
\DeclareMathSymbol{V}{\mathalpha}{italicglyphs}{`V}
\DeclareMathSymbol{W}{\mathalpha}{italicglyphs}{`W}
\DeclareMathSymbol{X}{\mathalpha}{italicglyphs}{`X}
\DeclareMathSymbol{Y}{\mathalpha}{italicglyphs}{`Y}
\DeclareMathSymbol{Z}{\mathalpha}{italicglyphs}{`Z}
% lowercase Latin characters
\DeclareMathSymbol{a}{\mathalpha}{italicglyphs}{`a}
\DeclareMathSymbol{b}{\mathalpha}{italicglyphs}{`b}
\DeclareMathSymbol{c}{\mathalpha}{italicglyphs}{`c}
\DeclareMathSymbol{d}{\mathalpha}{italicglyphs}{`d}
\DeclareMathSymbol{e}{\mathalpha}{italicglyphs}{`e}
\DeclareMathSymbol{f}{\mathalpha}{italicglyphs}{`f}
\DeclareMathSymbol{g}{\mathalpha}{italicglyphs}{`g}
\DeclareMathSymbol{h}{\mathalpha}{italicglyphs}{`h}
\DeclareMathSymbol{i}{\mathalpha}{italicglyphs}{`i}
\DeclareMathSymbol{\imath}{\mathalpha}{italicglyphs}{"19}
\DeclareMathSymbol{j}{\mathalpha}{italicglyphs}{`j}
\DeclareMathSymbol{\jmath}{\mathalpha}{italicglyphs}{"1A}
\DeclareMathSymbol{k}{\mathalpha}{italicglyphs}{`k}
\DeclareMathSymbol{l}{\mathalpha}{italicglyphs}{`l}
\DeclareMathSymbol{m}{\mathalpha}{italicglyphs}{`m}
\DeclareMathSymbol{n}{\mathalpha}{italicglyphs}{`n}
\DeclareMathSymbol{o}{\mathalpha}{italicglyphs}{`o}
\DeclareMathSymbol{p}{\mathalpha}{italicglyphs}{`p}
\DeclareMathSymbol{q}{\mathalpha}{italicglyphs}{`q}
\DeclareMathSymbol{r}{\mathalpha}{italicglyphs}{`r}
\DeclareMathSymbol{s}{\mathalpha}{italicglyphs}{`s}
\DeclareMathSymbol{t}{\mathalpha}{italicglyphs}{`t}
\DeclareMathSymbol{u}{\mathalpha}{italicglyphs}{`u}
\DeclareMathSymbol{v}{\mathalpha}{italicglyphs}{`v}
\DeclareMathSymbol{w}{\mathalpha}{italicglyphs}{`w}
\DeclareMathSymbol{x}{\mathalpha}{italicglyphs}{`x}
\DeclareMathSymbol{y}{\mathalpha}{italicglyphs}{`y}
\DeclareMathSymbol{z}{\mathalpha}{italicglyphs}{`z}

%% Sansserif Greek letters
%\DeclareSymbolFont{lgrgreek}{LGR}{\savesffamily}{\savesfmdseries}{it}
%\SetSymbolFont{lgrgreek}{sans}    {LGR}{\savesffamily}{\savesfmdseries}{it}
%\SetSymbolFont{lgrgreek}{boldsans}{LGR}{\savesffamily}{\savesfbfseries}{it}

% The following is taken from
% https://tex.stackexchange.com/questions/116389/automatic-upright-math-when-text-is-in-italic/116399#116399
% Filling in ``missing'' Greek glyphs for completeness
% (not really necessary, since they look identical to Latin glyphs and are thus almost never used)
% ==>
\newcommand{\omicron}{o}
\newcommand{\Digamma}{F}
\newcommand{\Alpha}  {A}
\newcommand{\Beta}   {B}
\newcommand{\Epsilon}{E}
\newcommand{\Zeta}   {Z}
\newcommand{\Eta}    {H}
\newcommand{\Iota}   {I}
\newcommand{\Kappa}  {K}
\newcommand{\Mu}     {M}
\newcommand{\Nu}     {N}
\newcommand{\Omicron}{O}
\newcommand{\Rho}    {P}
\newcommand{\Tau}    {T}
\newcommand{\Chi}    {X}
\newcommand{\omicronup}{\mathup{o}}
\newcommand{\Digammaup}{\mathup{F}}
\newcommand{\Alphaup}  {\mathup{A}}
\newcommand{\Betaup}   {\mathup{B}}
\newcommand{\Epsilonup}{\mathup{E}}
\newcommand{\Zetaup}   {\mathup{Z}}
\newcommand{\Etaup}    {\mathup{H}}
\newcommand{\Iotaup}   {\mathup{I}}
\newcommand{\Kappaup}  {\mathup{K}}
\newcommand{\Muup}     {\mathup{M}}
\newcommand{\Nuup}     {\mathup{N}}
\newcommand{\Omicronup}{\mathup{O}}
\newcommand{\Rhoup}    {\mathup{P}}
\newcommand{\Tauup}    {\mathup{T}}
\newcommand{\Chiup}    {\mathup{X}}
% <==

% Save original definitions of the Greek letters
% ==>
\makeatletter
\@for\@tempa:=%
	alpha,beta,gamma,delta,epsilon,varepsilon,zeta,eta,theta,vartheta,iota,kappa,lambda,mu,nu,xi,%
	omicron,pi,varpi,rho,varrho,sigma,varsigma,tau,upsilon,phi,varphi,chi,psi,omega,digamma,%
	Alpha,Beta,Gamma,Delta,Epsilon,Zeta,Eta,Theta,Iota,Kappa,Lambda,Mu,Nu,Xi,%
	Omicron,Pi,Rho,Sigma,Tau,Upsilon,Phi,Chi,Psi,Omega,Digamma%
	\do{%
		\expandafter\let\csname\@tempa orig\expandafter\endcsname\csname\@tempa\endcsname%
		\expandafter\let\csname\@tempa uporig\expandafter\endcsname\csname\@tempa up\endcsname%
	}%
\makeatother
% <==

% LGR-encoded Greek letters
% ==>
\newcommand{\textformath}[1]{%
	\IfInBoldMode%
		\IfInUpMode\textbf{#1}\else\textit{\bfseries #1}\fi\relax%
	\else
		\IfInUpMode\textup{#1}\else\textit{#1}\fi\relax%
	\fi\relax%
}
% The double curly braces in this section are necessary to be able to use Greek letters
% in subscripts and superscripts without having to enclose theme in curly braces;
% for example, $\sigma_\epsilon$ instead of $\sigma_{\epsilon}$.
% Uppercase
\newcommand{\AlphaLGR}   {{\mathord{\textformath{\fontencoding{LGR}\selectfont A}}}}
\newcommand{\BetaLGR}    {{\mathord{\textformath{\fontencoding{LGR}\selectfont B}}}}
\newcommand{\GammaLGR}   {{\mathord{\textformath{\fontencoding{LGR}\selectfont G}}}}
\newcommand{\DeltaLGR}   {{\mathord{\textformath{\fontencoding{LGR}\selectfont D}}}}
\newcommand{\EpsilonLGR} {{\mathord{\textformath{\fontencoding{LGR}\selectfont E}}}}
\newcommand{\ZetaLGR}    {{\mathord{\textformath{\fontencoding{LGR}\selectfont Z}}}}
\newcommand{\EtaLGR}     {{\mathord{\textformath{\fontencoding{LGR}\selectfont H}}}}
\newcommand{\ThetaLGR}   {{\mathord{\textformath{\fontencoding{LGR}\selectfont J}}}}
\newcommand{\IotaLGR}    {{\mathord{\textformath{\fontencoding{LGR}\selectfont I}}}}
\newcommand{\KappaLGR}   {{\mathord{\textformath{\fontencoding{LGR}\selectfont K}}}}
\newcommand{\LambdaLGR}  {{\mathord{\textformath{\fontencoding{LGR}\selectfont L}}}}
\newcommand{\MuLGR}      {{\mathord{\textformath{\fontencoding{LGR}\selectfont M}}}}
\newcommand{\NuLGR}      {{\mathord{\textformath{\fontencoding{LGR}\selectfont N}}}}
\newcommand{\XiLGR}      {{\mathord{\textformath{\fontencoding{LGR}\selectfont X}}}}
\newcommand{\OmicronLGR} {{\mathord{\textformath{\fontencoding{LGR}\selectfont O}}}}
\newcommand{\PiLGR}      {{\mathord{\textformath{\fontencoding{LGR}\selectfont P}}}}
\newcommand{\RhoLGR}     {{\mathord{\textformath{\fontencoding{LGR}\selectfont R}}}}
\newcommand{\SigmaLGR}   {{\mathord{\textformath{\fontencoding{LGR}\selectfont S}}}}
\newcommand{\TauLGR}     {{\mathord{\textformath{\fontencoding{LGR}\selectfont T}}}}
\newcommand{\UpsilonLGR} {{\mathord{\textformath{\fontencoding{LGR}\selectfont U}}}}
\newcommand{\PhiLGR}     {{\mathord{\textformath{\fontencoding{LGR}\selectfont F}}}}
\newcommand{\ChiLGR}     {{\mathord{\textformath{\fontencoding{LGR}\selectfont Q}}}}
\newcommand{\PsiLGR}     {{\mathord{\textformath{\fontencoding{LGR}\selectfont Y}}}}
\newcommand{\OmegaLGR}   {{\mathord{\textformath{\fontencoding{LGR}\selectfont W}}}}
\newcommand{\DigammaLGR} {{\mathord{\textformath{\fontencoding{LGR}\selectfont \char195}}}}
% lowercase
\newcommand{\alphaLGR}   {{\mathord{\textformath{\fontencoding{LGR}\selectfont a}}}}
\newcommand{\betaLGR}    {{\mathord{\textformath{\fontencoding{LGR}\selectfont b}}}}
\newcommand{\gammaLGR}   {{\mathord{\textformath{\fontencoding{LGR}\selectfont g}}}}
\newcommand{\deltaLGR}   {{\mathord{\textformath{\fontencoding{LGR}\selectfont d}}}}
\newcommand{\epsilonLGR} {{\mathord{\textformath{\fontencoding{LGR}\selectfont e}}}}
\newcommand{\zetaLGR}    {{\mathord{\textformath{\fontencoding{LGR}\selectfont z}}}}
\newcommand{\etaLGR}     {{\mathord{\textformath{\fontencoding{LGR}\selectfont h}}}}
\newcommand{\thetaLGR}   {{\mathord{\textformath{\fontencoding{LGR}\selectfont j}}}}
\newcommand{\iotaLGR}    {{\mathord{\textformath{\fontencoding{LGR}\selectfont i}}}}
\newcommand{\kappaLGR}   {{\mathord{\textformath{\fontencoding{LGR}\selectfont k}}}}
\newcommand{\lambdaLGR}  {{\mathord{\textformath{\fontencoding{LGR}\selectfont l}}}}
\newcommand{\muLGR}      {{\mathord{\textformath{\fontencoding{LGR}\selectfont m}}}}
\newcommand{\nuLGR}      {{\mathord{\textformath{\fontencoding{LGR}\selectfont n}}}}
\newcommand{\xiLGR}      {{\mathord{\textformath{\fontencoding{LGR}\selectfont x}}}}
\newcommand{\omicronLGR} {{\mathord{\textformath{\fontencoding{LGR}\selectfont o}}}}
\newcommand{\piLGR}      {{\mathord{\textformath{\fontencoding{LGR}\selectfont p}}}}
\newcommand{\rhoLGR}     {{\mathord{\textformath{\fontencoding{LGR}\selectfont r}}}}
\newcommand{\sigmaLGR}   {{\mathord{\textformath{\fontencoding{LGR}\selectfont s\noboundary}}}}
	% \noboundary prevents sigma from being replaced by the word-end sigma (varsigma),
	% see http://mirrors.ctan.org/macros/latex/contrib/textgreek/textgreek.pdf
\newcommand{\varsigmaLGR}{{\mathord{\textformath{\fontencoding{LGR}\selectfont c}}}}
\newcommand{\tauLGR}     {{\mathord{\textformath{\fontencoding{LGR}\selectfont t}}}}
\newcommand{\upsilonLGR} {{\mathord{\textformath{\fontencoding{LGR}\selectfont u}}}}
\newcommand{\phiLGR}     {{\mathord{\textformath{\fontencoding{LGR}\selectfont f}}}}
\newcommand{\chiLGR}     {{\mathord{\textformath{\fontencoding{LGR}\selectfont q}}}}
\newcommand{\psiLGR}     {{\mathord{\textformath{\fontencoding{LGR}\selectfont y}}}}
\newcommand{\omegaLGR}   {{\mathord{\textformath{\fontencoding{LGR}\selectfont w}}}}
\newcommand{\digammaLGR} {{\mathord{\textformath{\fontencoding{LGR}\selectfont \char147}}}}
% Uppercase, upright
\newcommand{\AlphaupLGR}   {{\mathord{\textup{\fontencoding{LGR}\selectfont A}}}}
\newcommand{\BetaupLGR}    {{\mathord{\textup{\fontencoding{LGR}\selectfont B}}}}
\newcommand{\GammaupLGR}   {{\mathord{\textup{\fontencoding{LGR}\selectfont G}}}}
\newcommand{\DeltaupLGR}   {{\mathord{\textup{\fontencoding{LGR}\selectfont D}}}}
\newcommand{\EpsilonupLGR} {{\mathord{\textup{\fontencoding{LGR}\selectfont E}}}}
\newcommand{\ZetaupLGR}    {{\mathord{\textup{\fontencoding{LGR}\selectfont Z}}}}
\newcommand{\EtaupLGR}     {{\mathord{\textup{\fontencoding{LGR}\selectfont H}}}}
\newcommand{\ThetaupLGR}   {{\mathord{\textup{\fontencoding{LGR}\selectfont J}}}}
\newcommand{\IotaupLGR}    {{\mathord{\textup{\fontencoding{LGR}\selectfont I}}}}
\newcommand{\KappaupLGR}   {{\mathord{\textup{\fontencoding{LGR}\selectfont K}}}}
\newcommand{\LambdaupLGR}  {{\mathord{\textup{\fontencoding{LGR}\selectfont L}}}}
\newcommand{\MuupLGR}      {{\mathord{\textup{\fontencoding{LGR}\selectfont M}}}}
\newcommand{\NuupLGR}      {{\mathord{\textup{\fontencoding{LGR}\selectfont N}}}}
\newcommand{\XiupLGR}      {{\mathord{\textup{\fontencoding{LGR}\selectfont X}}}}
\newcommand{\OmicronupLGR} {{\mathord{\textup{\fontencoding{LGR}\selectfont O}}}}
\newcommand{\PiupLGR}      {{\mathord{\textup{\fontencoding{LGR}\selectfont P}}}}
\newcommand{\RhoupLGR}     {{\mathord{\textup{\fontencoding{LGR}\selectfont R}}}}
\newcommand{\SigmaupLGR}   {{\mathord{\textup{\fontencoding{LGR}\selectfont S}}}}
\newcommand{\TauupLGR}     {{\mathord{\textup{\fontencoding{LGR}\selectfont T}}}}
\newcommand{\UpsilonupLGR} {{\mathord{\textup{\fontencoding{LGR}\selectfont U}}}}
\newcommand{\PhiupLGR}     {{\mathord{\textup{\fontencoding{LGR}\selectfont F}}}}
\newcommand{\ChiupLGR}     {{\mathord{\textup{\fontencoding{LGR}\selectfont Q}}}}
\newcommand{\PsiupLGR}     {{\mathord{\textup{\fontencoding{LGR}\selectfont Y}}}}
\newcommand{\OmegaupLGR}   {{\mathord{\textup{\fontencoding{LGR}\selectfont W}}}}
\newcommand{\DigammaupLGR} {{\mathord{\textup{\fontencoding{LGR}\selectfont \char195}}}}
% lowercase, upright
\newcommand{\alphaupLGR}   {{\mathord{\textup{\fontencoding{LGR}\selectfont a}}}}
\newcommand{\betaupLGR}    {{\mathord{\textup{\fontencoding{LGR}\selectfont b}}}}
\newcommand{\gammaupLGR}   {{\mathord{\textup{\fontencoding{LGR}\selectfont g}}}}
\newcommand{\deltaupLGR}   {{\mathord{\textup{\fontencoding{LGR}\selectfont d}}}}
\newcommand{\epsilonupLGR} {{\mathord{\textup{\fontencoding{LGR}\selectfont e}}}}
\newcommand{\zetaupLGR}    {{\mathord{\textup{\fontencoding{LGR}\selectfont z}}}}
\newcommand{\etaupLGR}     {{\mathord{\textup{\fontencoding{LGR}\selectfont h}}}}
\newcommand{\thetaupLGR}   {{\mathord{\textup{\fontencoding{LGR}\selectfont j}}}}
\newcommand{\iotaupLGR}    {{\mathord{\textup{\fontencoding{LGR}\selectfont i}}}}
\newcommand{\kappaupLGR}   {{\mathord{\textup{\fontencoding{LGR}\selectfont k}}}}
\newcommand{\lambdaupLGR}  {{\mathord{\textup{\fontencoding{LGR}\selectfont l}}}}
\newcommand{\muupLGR}      {{\mathord{\textup{\fontencoding{LGR}\selectfont m}}}}
\newcommand{\nuupLGR}      {{\mathord{\textup{\fontencoding{LGR}\selectfont n}}}}
\newcommand{\xiupLGR}      {{\mathord{\textup{\fontencoding{LGR}\selectfont x}}}}
\newcommand{\omicronupLGR} {{\mathord{\textup{\fontencoding{LGR}\selectfont o}}}}
\newcommand{\piupLGR}      {{\mathord{\textup{\fontencoding{LGR}\selectfont p}}}}
\newcommand{\rhoupLGR}     {{\mathord{\textup{\fontencoding{LGR}\selectfont r}}}}
\newcommand{\sigmaupLGR}   {{\mathord{\textup{\fontencoding{LGR}\selectfont s\noboundary}}}}
	% \noboundary prevents sigma from being replaced by the word-end sigma (varsigma),
	% see http://mirrors.ctan.org/macros/latex/contrib/textgreek/textgreek.pdf
\newcommand{\varsigmaupLGR}{{\mathord{\textup{\fontencoding{LGR}\selectfont c}}}}
\newcommand{\tauupLGR}     {{\mathord{\textup{\fontencoding{LGR}\selectfont t}}}}
\newcommand{\upsilonupLGR} {{\mathord{\textup{\fontencoding{LGR}\selectfont u}}}}
\newcommand{\phiupLGR}     {{\mathord{\textup{\fontencoding{LGR}\selectfont f}}}}
\newcommand{\chiupLGR}     {{\mathord{\textup{\fontencoding{LGR}\selectfont q}}}}
\newcommand{\psiupLGR}     {{\mathord{\textup{\fontencoding{LGR}\selectfont y}}}}
\newcommand{\omegaupLGR}   {{\mathord{\textup{\fontencoding{LGR}\selectfont w}}}}
\newcommand{\digammaupLGR} {{\mathord{\textup{\fontencoding{LGR}\selectfont \char147}}}}
% <==

% Based on description of the TS1 encoding in
% http://ctan.math.illinois.edu/macros/latex/doc/encguide.pdf:
%\let \oldpm    \pm
%\let \oldtimes \times
%\let \olddiv   \div
%\makeatletter
%\newcommand{\pmsf}   {\mathbin{\text{\usefont{TS1}{\sfdefault}{\f@series}{n}\char"B1}}}
%\newcommand{\timessf}{\mathbin{\text{\usefont{TS1}{\sfdefault}{\f@series}{n}\char"D6}}}
%\newcommand{\divsf}  {\mathbin{\text{\usefont{TS1}{\sfdefault}{\f@series}{n}\char"F6}}}
%\makeatother

% Use LGR-encoded Greek letters for \mathversion{sans}
% ==>
\makeatletter

% Save original definition of \varepsilon etc.
\@for\@tempa:=%
	epsilon,theta,pi,rho,sigma,phi%
\do{%
	\expandafter\let\csname var\@tempa orig\expandafter\endcsname\csname var\@tempa\endcsname%
}

\newcommand*{\sansmath}{%
	\@for\@tempa:=%
		alpha,beta,gamma,delta,epsilon,zeta,eta,theta,iota,kappa,lambda,mu,nu,xi,%
		omicron,pi,rho,sigma,varsigma,tau,upsilon,phi,chi,psi,omega,digamma,%
		Alpha,Beta,Gamma,Delta,Epsilon,Zeta,Eta,Theta,Iota,Kappa,Lambda,Mu,Nu,Xi,%
		Omicron,Pi,Rho,Sigma,Tau,Upsilon,Phi,Chi,Psi,Omega,Digamma%
	\do{%
		\expandafter\let\csname\@tempa\expandafter\endcsname\csname\@tempa LGR\endcsname%
		\expandafter\let\csname\@tempa up\expandafter\endcsname\csname\@tempa upLGR\endcsname%
		\expandafter\let\csname up\@tempa\expandafter\endcsname\csname\@tempa upLGR\endcsname%
	}%
	\@for\@tempa:=%
		epsilon,theta,pi,rho,phi%
	\do{%
		\expandafter\let\csname var\@tempa\expandafter\endcsname\csname\@tempa\endcsname%
		\expandafter\let\csname var\@tempa up\expandafter\endcsname\csname\@tempa up\endcsname%
		\expandafter\let\csname upvar\@tempa\expandafter\endcsname\csname up\@tempa\endcsname%
	}%
	%\renewcommand{\pm}{\pmsf}%
	%\renewcommand{\times}{\timessf}%
	%\renewcommand{\div}{\divsf}%
}
% <==

% Switch back to the original Greek letters for \mathversion{normal}, i.e., the serif font
% ==>
\newcommand*{\unsansmath}{%
	\@for\@tempa:=%
		alpha,beta,gamma,delta,epsilon,zeta,eta,theta,iota,kappa,lambda,mu,nu,xi,%
		omicron,pi,rho,sigma,varsigma,tau,upsilon,phi,chi,psi,omega,digamma,%
		Alpha,Beta,Gamma,Delta,Epsilon,Zeta,Eta,Theta,Iota,Kappa,Lambda,Mu,Nu,Xi,%
		Omicron,Pi,Rho,Sigma,Tau,Upsilon,Phi,Chi,Psi,Omega,Digamma%
	\do{%
		\expandafter\let\csname\@tempa\expandafter\endcsname\csname\@tempa orig\endcsname%
		\expandafter\let\csname\@tempa up\expandafter\endcsname\csname\@tempa uporig\endcsname%
		\expandafter\let\csname up\@tempa\expandafter\endcsname\csname\@tempa uporig\endcsname%
	}%
	\@for\@tempa:=%
		epsilon,theta,pi,rho,phi%
	\do{%
		\expandafter\let\csname var\@tempa\expandafter\endcsname\csname var\@tempa orig\endcsname%
		\expandafter\let\csname var\@tempa up\expandafter\endcsname\csname var\@tempa uporig\endcsname%
		\expandafter\let\csname upvar\@tempa\expandafter\endcsname\csname var\@tempa uporig\endcsname%
	}%
	%\renewcommand{\pm}{\oldpm}%
	%\renewcommand{\times}{\oldtimes}%
	%\renewcommand{\div}{\olddiv}%
}
% <==

%% If you would like to use LGR-encoded Greek letters also for the serif font
%% ==>
%\renewcommand*{\unsansmath}{%
%	\@for\@tempa:=%
%		alpha,beta,gamma,delta,epsilon,zeta,eta,theta,iota,kappa,lambda,mu,nu,xi,%
%		omicron,pi,rho,sigma,varsigma,tau,upsilon,phi,chi,psi,omega,digamma,%
%		Alpha,Beta,Gamma,Delta,Epsilon,Zeta,Eta,Theta,Iota,Kappa,Lambda,Mu,Nu,Xi,%
%		Omicron,Pi,Rho,Sigma,Tau,Upsilon,Phi,Chi,Psi,Omega,Digamma%
%		\do{%
%			\expandafter\let\csname\@tempa\expandafter\endcsname\csname\@tempa LGR\endcsname%
%			\expandafter\let\csname\@tempa up\expandafter\endcsname\csname\@tempa upLGR\endcsname%
%			\expandafter\let\csname up\@tempa\expandafter\endcsname\csname\@tempa upLGR\endcsname%
%		}%
%	%\renewcommand{\pm}{\pmsf}%
%	%\renewcommand{\times}{\timessf}%
%	%\renewcommand{\div}{\divsf}%
%}
%% <==

\newcommand*{\upgreekletters}{%
	\@for\@tempa:=%
		alpha,beta,gamma,delta,epsilon,varepsilon,zeta,eta,theta,vartheta,iota,kappa,lambda,mu,nu,xi,%
		omicron,pi,varpi,rho,varrho,sigma,varsigma,tau,upsilon,phi,varphi,chi,psi,omega,digamma,%
		Alpha,Beta,Gamma,Delta,Epsilon,Zeta,Eta,Theta,Iota,Kappa,Lambda,Mu,Nu,Xi,%
		Omicron,Pi,Rho,Sigma,Tau,Upsilon,Phi,Chi,Psi,Omega,Digamma%
	\do{%
		\expandafter\let\csname\@tempa\expandafter\endcsname\csname\@tempa up\endcsname%
	}%
}
\newcommand*{\itgreekletters}{%
	\@for\@tempa:=%
		alpha,beta,gamma,delta,epsilon,varepsilon,zeta,eta,theta,vartheta,iota,kappa,lambda,mu,nu,xi,%
		omicron,pi,varpi,rho,varrho,sigma,varsigma,tau,upsilon,phi,varphi,chi,psi,omega,digamma,%
		Alpha,Beta,Gamma,Delta,Epsilon,Zeta,Eta,Theta,Iota,Kappa,Lambda,Mu,Nu,Xi,%
		Omicron,Pi,Rho,Sigma,Tau,Upsilon,Phi,Chi,Psi,Omega,Digamma%
	\do{%
		\expandafter\let\csname\@tempa\expandafter\endcsname\csname\@tempa orig\endcsname%
	}%
}

\makeatother

%\makeatletter
%	\@for\@tempa:=%
%	%alpha,beta,gamma,delta,epsilon,zeta,eta,theta,iota,kappa,lambda,mu,nu,xi,%
%	%pi,rho,sigma,varsigma,tau,upsilon,phi,chi,psi,omega,digamma,%
%	Gamma,Delta,Theta,Lambda,Xi,Pi,Sigma,Upsilon,Phi,Psi,Omega%
%	\do{\expandafter\let\csname\@tempa\expandafter\endcsname\csname other\@tempa\endcsname}%
%\makeatother

% Fix the \bm command so that it also works properly in the sans mathversions
% ==>
\let \bmorig \bm
\renewcommand{\bm}[1]{%
	\IfInSansMode%
		\textbf{\mathversion{boldsans}\(#1\)}%
	\else%
		\bmorig{#1}%
	\fi\relax%
}
% <==
\renewcommand{\mathbf}[1]{\bm{#1}}
\renewcommand{\boldsymbol}[1]{\bm{#1}}
\newcommand{\mathbfit}[1]{\mathbf{\mathit{#1}}}
\renewcommand{\mathcal}[1]{\mathscr{#1}}

% Apply sansmath etc. automagically
% ==>
\newif\IfInSansMode
\newif\IfInBoldMode
\newif\IfInUpMode
\let \oldsf \sffamily
\renewcommand*{\sffamily}{%
	\oldsf\sansmath\InSansModetrue%
	\IfInBoldMode\mathversion{boldsans}\else\mathversion{sans}\fi\relax%
}
\let \oldbf \bfseries
\renewcommand*{\bfseries}{%
	\oldbf\InBoldModetrue%
	\IfInSansMode\sansmath\mathversion{boldsans}\else\mathversion{bold}\fi\relax%
}
\let \oldmd \mdseries
\renewcommand*{\mdseries}{%
	\oldmd\InBoldModefalse%
	\IfInSansMode\sansmath\mathversion{sans}\else\mathversion{normal}\fi\relax%
}
\let \oldnorm \normalfont
\renewcommand*{\normalfont}{%
	\oldnorm\InSansModefalse\InBoldModefalse\mathversion{normal}%
	\unsansmath%
}
\let \oldrm \rmfamily
\renewcommand*{\rmfamily}{%
	\oldrm\InSansModefalse%
	\IfInBoldMode\mathversion{bold}\else\mathversion{normal}\fi\relax%
	\unsansmath%
}
% <==

% Make \mathnormal obey the currently active \mathversion ==>
\let \mathnormalorig \mathnormal
\renewcommand{\mathnormal}[1]{%
	\IfInSansMode%
		\IfInBoldMode%
			\mathversion{boldsans}%
			{\textbf{\(#1\)}}%
		\else%
			\mathversion{sans}%
			{\textmd{\(#1\)}}%
		\fi\relax%
	\else%
		\mathnormalorig{#1}%
	\fi\relax%
}
% <==

% Adjust \mathrm to the curretly active \mathversion.
% We set it up such that also in sansserif mode, \mathrm activates the serif font.
% ==>
\let \mathrmorig \mathrm
\renewcommand{\mathrm}[1]{%
	\IfInSansMode%
		{\textrm{%
			\IfInBoldMode%
				\mathversion{bold}%
				\(\mathrmorig{#1}\)%
			\else%
				\mathversion{normal}%
				\(\mathrmorig{#1}\)%
			\fi\relax%
		}}%
	\else%
		\mathrmorig{#1}%
	\fi\relax%
}
% <==

% Define \mathup to activate \upshape without switching to the serif font
% (in contrast to \mathrm)
% ==>
\newcommand{\mathup}[1]{%
	\IfInSansMode%
		{\textup{%
			\InUpModetrue%
			\IfInBoldMode%
				\mathversion{boldsansup}%
				\(#1\)%
			\else%
				\mathversion{sansup}%
				\(#1\)%
			\fi\relax%
		}}%
	\else%
		{\upgreekletters\mathrm{#1}\itgreekletters}%
	\fi\relax%
}
\newcommand{\mathbfup}[1]{%
	\IfInSansMode%
		{\mathbf{\mathup{#1}}}%
	\else%
		{\upgreekletters\mathbf{\mathrm{#1}}\itgreekletters}%
	\fi\relax%
}
% <==

%% If you would like to redefine \mathup also for the serif font:
%% ==>
%\renewcommand{\mathup}[1]{%
%	\IfInSansMode%
%		{\textup{%
%			\InUpModetrue%
%			\IfInBoldMode%
%				\mathversion{boldsansup}%
%				\(#1\)%
%			\else%
%				\mathversion{sansup}%
%				\(#1\)%
%			\fi\relax%
%		}}%
%	\else%
%		{\textup{%
%			\InUpModetrue%
%			\IfInBoldMode%
%				\mathversion{boldup}%
%				\(#1\)%
%			\else%
%				\mathversion{normalup}%
%				\(#1\)%
%			\fi\relax%
%		}}%
%	\fi\relax%
%}
%\renewcommand{\mathbfup}[1]{%
%	\IfInSansMode%
%		{\mathbf{\mathup{#1}}}%
%	\else%
%		{\upgreekletters\mathbf{\mathup{#1}}\itgreekletters}%
%	\fi\relax%
%}
%% <==

% Make the LaTeX-defined operators obey sansserif math
% ==>
\let \operatornameorig \operatorname
\renewcommand{\operatorname}[1]{%
	\operatornameorig{\mathup{#1}}%
}
\makeatletter
\@for\@tempa:=%
	arccos,arccot,arccsc,arcsec,arcsin,arctan,arg,cos,cosh,cot,coth,csc,%
	deg,det,dim,exp,gcd,hom,inf,ker,lg,lim,liminf,limsup,ln,log,max,min,%
	Pr,sec,sin,sinh,sup,tan,tanh%
	\do{%
		\expandafter\let\csname\@tempa\endcsname\relax%
	}%
\makeatother
\DeclareMathOperator {\arccos}{\mathup{arccos}}
\DeclareMathOperator {\arccot}{\mathup{arccot}}
\DeclareMathOperator {\arccsc}{\mathup{arccsc}}
\DeclareMathOperator {\arcsec}{\mathup{arcsec}}
\DeclareMathOperator {\arcsin}{\mathup{arcsin}}
\DeclareMathOperator {\arctan}{\mathup{arctan}}
\DeclareMathOperator {\arg}   {\mathup{arg}}
\DeclareMathOperator {\cos}   {\mathup{cos}}
\DeclareMathOperator {\cosh}  {\mathup{cosh}}
\DeclareMathOperator {\cot}   {\mathup{cot}}
\DeclareMathOperator {\coth}  {\mathup{coth}}
\DeclareMathOperator {\csc}   {\mathup{csc}}
\DeclareMathOperator {\deg}   {\mathup{deg}}
\DeclareMathOperator {\det}   {\mathup{det}}
\DeclareMathOperator {\dim}   {\mathup{dim}}
\DeclareMathOperator {\exp}   {\mathup{exp}}
\DeclareMathOperator {\gcd}   {\mathup{gcd}}
\DeclareMathOperator*{\hom}   {\mathup{hom}}
\DeclareMathOperator*{\inf}   {\mathup{inf}}
\DeclareMathOperator {\ker}   {\mathup{ker}}
\DeclareMathOperator {\lg}    {\mathup{lg}}
\DeclareMathOperator*{\lim}   {\mathup{lim}}
\DeclareMathOperator*{\liminf}{\mathup{lim\,inf}}
\DeclareMathOperator*{\limsup}{\mathup{lim\,sup}}
\DeclareMathOperator {\ln}    {\mathup{ln}}
\DeclareMathOperator {\log}   {\mathup{log}}
\DeclareMathOperator*{\max}   {\mathup{max}}
\DeclareMathOperator*{\min}   {\mathup{min}}
\DeclareMathOperator {\Pr}    {\mathup{Pr}}
\DeclareMathOperator {\sec}   {\mathup{sec}}
\DeclareMathOperator {\sin}   {\mathup{sin}}
\DeclareMathOperator {\sinh}  {\mathup{sinh}}
\DeclareMathOperator*{\sup}   {\mathup{sup}}
\DeclareMathOperator {\tan}   {\mathup{tan}}
\DeclareMathOperator {\tanh}  {\mathup{tanh}}
% <==
% !TeX program = pdflatex
% !TeX TXS-program:compile = txs:///pdflatex/
% !TeX TS-program = pdflatex
% !BIB program = biber
% !TeX TXS-program:bibliography = txs:///biber




%%%%%%%%%%%%%%%%%%%%%%%%%%%%%%%%%%%%%%%%%%%%%%%%
%%  CITATION COMMANDS AND BIBLIOGRAPHY STYLE  %%
%%%%%%%%%%%%%%%%%%%%%%%%%%%%%%%%%%%%%%%%%%%%%%%%


% AER/JEL/JEP style

\usepackage[
	authordate, backend = biber, natbib = true, doi = only, isbn = false,
	sorting = ynt, sortcites = true,  % sort the in-text citations by year of publication
	compresspages = true, dashed = false,
	backref = true, backrefstyle = three,
	bibencoding = inputenc, citetracker = true,
	noibid  % See https://tex.stackexchange.com/questions/129487/bug-in-the-parencite-command-of-the-biblatex-chicago-package
]{biblatex-chicago}

\urlstyle{same}  % Since biblatex-chicago sets \urlstyle{rm}
% See https://tex.stackexchange.com/questions/134191/line-breaks-of-long-urls-in-biblatex-bibliography:
\setcounter{biburllcpenalty}{10000}
\setcounter{biburlucpenalty}{10000}
\setcounter{biburlnumpenalty}{10000}
\AtBeginDocument{\biburlbigskip = 0mu\relax}  % To prevent excessive whitespace in URLs and DOIs

\DeclareDelimFormat{nameyeardelim}{\addcomma\space}
\let \citeorig \cite
\renewcommand{\cite}{\citet}
\renewcommand{\citealp}{\citeorig}
\AtBeginDocument{\setlength{\bibhang}{\baselineskip}}
\DeclareFieldFormat[report]{title}{\mkbibquote{#1}}
\AtBeginDocument{%
	\iftoggle{cms@comprange}
		{\DeclareFieldFormat{pages}{\addspace\mbox{\mkcomprange{#1}}}}%
		{\DeclareFieldFormat{pages}{\addspace\mbox{\mknormrange{#1}}}}%
}%
% Make author/editor name(s) bold
\xpatchbibmacro{author}{\usebibmacro{justauthor}}{\mkbibbold{\usebibmacro{justauthor}}}{}{}
\xpatchbibmacro{author}{\usebibmacro{moreauthor}}{\mkbibbold{\usebibmacro{moreauthor}}}{}{}

%\usepackage[backend=biber, natbib=true, bibencoding=inputenc, bibstyle=authoryear, citestyle=authoryear-comp, mincitenames=1, maxcitenames=3, minbibnames=99, maxbibnames=99, uniquename=false, uniquelist=true, backref=true, backrefstyle=three, doi=true, isbn=false, dashed=false, sorting=ynt, sortcites=true, mergedate=true, dateabbrev=false, abbreviate=false, citetracker=true]{biblatex}
% sortcites sorts the in-text citations by year of publication
\DeclareBibliographyAlias{newspaper}{article}

\defbibheading{subbibliography}[\refname]{%
	\section*{#1}%
	\sectionmark{#1}%
	\addcontentsline{toc}{section}{#1}%
}

\renewcommand{\bibfont}{\sffamily\small}
	% Reduce font size for the bibliography, make the font sans-serif.
\setlength{\bibindent}{\parindent}
\setlength{\bibitemsep}{0pt}

\DefineBibliographyStrings{english}{%
	andothers = {et~al\adddot},
	volume = {vol\adddot}
}

% Handling of newspaper articles
\DeclareFieldFormat[newspaper]{journaltitle}{%
	\mkbibemph{#1} \mkbibparens{\thefield{edition}}\addcomma\addspace%
	\iflanguage{ngerman}{%
		\addnbspace\thefield{day}\addperiod\addnbspace\mkbibmonth{\thefield{month}}\addspace\thefield{year}%
	}{% else: default to the USenglish version
		\mkbibmonth{\thefield{month}}\addnbspace\thefield{day}\addcomma\addspace\thefield{year}%
	}%
	\iffieldundef{volume}{\addcolon}{\addcomma}%
}
\DeclareFieldFormat[newspaper]{date}{%
	\thefield{year}%
}

% Adjust formatting of back-references
% ==>
\DefineBibliographyStrings{english}{%
	backrefpage  = {},	% originally ``cit. on p.''
	backrefpages = {}	% originally ``cit. on pp.''
}
\DefineBibliographyStrings{ngerman}{%
	backrefpage  = {},	% originally ``Siehe Seite''
	backrefpages = {}	% originally ``Siehe Seiten''
}
\renewcommand*{\finentrypunct}{}
\renewbibmacro*{pageref}{%
	\addperiod
	\iflistundef{pageref}
	{}
	{\printtext[brackets]{%
			\ifnumgreater{\value{pageref}}{1}
				{\bibstring{backrefpages}}
				{\bibstring{backrefpage}}%
			\printlist[pageref][-\value{listtotal}]{pageref}%
		}%
	}%
}
% <==

% Move notes to the end of an entry by copying the ``note'' information into ``addendum'': -->
\DeclareSourcemap{
	\maps[datatype=bibtex]{
		% Copy values of the Mendeley-created ``annote'' field to the ``note'' field:
		\map[overwrite]{
		 	\step[fieldsource=annote]
		 	\step[fieldset=note, origfieldval, append]
			\step[fieldset=annote, null]
		}
		\map{
			\step[fieldsource=note, final]
			\step[fieldset=addendum, origfieldval, final]
			\step[fieldset=note, null]
		}
	}
}
% \DeclareFieldFormat{addendum}{(#1)} % Enclose addendum/note in parentheses.
% <--

\AtEveryBibitem{%
	\ifentrytype{newspaper}%
		{\clearfield{addendum}}% then
		{\clearfield{month}}% else
}

% Add ``working paper'' as the default value for type of ``techreports'' ==>
% see https://tex.stackexchange.com/questions/212362/how-to-use-declaresourcemap-to-add-default-value-to-a-field
\DeclareSourcemap{
	\maps[datatype=bibtex]{
		\map{% Will overwrite fields without the ``overwrite'' option
			\pertype{techreport}
			\step[fieldset=type, fieldvalue={Working paper}]
		}
	}
}
% <==

% Add “doctoral dissertation” as the default value for type of “phdthesis” ==>
% see https://tex.stackexchange.com/questions/212362/how-to-use-declaresourcemap-to-add-default-value-to-a-field
\DeclareSourcemap{
	\maps[datatype=bibtex]{
		\map{% Will overwrite fields without the ``overwrite'' option
			\pertype{phdthesis}
			\step[fieldset=type, fieldvalue={Doctoral dissertation}]
		}
	}
}
% <==

% Remove superfluous ``The'' from journal names ==>
\DeclareSourcemap{
	\maps[datatype=bibtex]{
		\map{
			\step[fieldsource=journal, match={\regexp{^The\s}}, replace={}]
		}
	}
}
% <==

% Replace (Mendeley's) month strings (``jan'', ``feb'', etc.) by full names ==>
\DeclareSourcemap{
	\maps[datatype=bibtex]{
		\map{
			\step[fieldsource=number, match={\regexp{^jan$}}, replace=\regexp{\\mkbibmonth\{1\}}]
			\step[fieldsource=number, match={\regexp{^feb$}}, replace=\regexp{\\mkbibmonth\{2\}}]
			\step[fieldsource=number, match={\regexp{^mar$}}, replace=\regexp{\\mkbibmonth\{3\}}]
			\step[fieldsource=number, match={\regexp{^apr$}}, replace=\regexp{\\mkbibmonth\{4\}}]
			\step[fieldsource=number, match={\regexp{^may$}}, replace=\regexp{\\mkbibmonth\{5\}}]
			\step[fieldsource=number, match={\regexp{^jun$}}, replace=\regexp{\\mkbibmonth\{6\}}]
			\step[fieldsource=number, match={\regexp{^jul$}}, replace=\regexp{\\mkbibmonth\{7\}}]
			\step[fieldsource=number, match={\regexp{^aug$}}, replace=\regexp{\\mkbibmonth\{8\}}]
			\step[fieldsource=number, match={\regexp{^sep$}}, replace=\regexp{\\mkbibmonth\{9\}}]
			\step[fieldsource=number, match={\regexp{^oct$}}, replace=\regexp{\\mkbibmonth\{10\}}]
			\step[fieldsource=number, match={\regexp{^nov$}}, replace=\regexp{\\mkbibmonth\{11\}}]
			\step[fieldsource=number, match={\regexp{^dec$}}, replace=\regexp{\\mkbibmonth\{12\}}]
		}
	}
}
% <==  %% !TeX program = pdflatex
% !TeX TXS-program:compile = txs:///pdflatex/
% !TeX TS-program = pdflatex
% !BIB program = biber
% !TeX TXS-program:bibliography = txs:///biber




%%%%%%%%%%%%%%%%%%%%%%%%%%%%%%%%%%%%%%%%%%%%%%%
%%  CITATION COMMANDS AND BIBLIOGRAPHY STYLE %%
%%%%%%%%%%%%%%%%%%%%%%%%%%%%%%%%%%%%%%%%%%%%%%%


% For APA:
\usepackage[style=apa, backend=biber, natbib=true, backref=true]{biblatex}

\let\citeorig\cite
\renewcommand{\cite}{\citet}
\renewcommand{\citealp}{\citeorig}

\renewcommand{\bibfont}{\sffamily\small}
	% Reduce font size for the bibliography, make the font sans-serif.
\AtBeginDocument{%
	\setlength{\bibindent}{\parindent}%
}
\setlength{\bibitemsep}{0pt}

\DefineBibliographyStrings{english}{%
	backrefpage = {},	% originally ``cit. on p.''
	backrefpages = {}	% originally ``cit. on pp.''
}
\DefineBibliographyStrings{ngerman}{%
	backrefpage  = {},	% originally ``Siehe Seite''
	backrefpages = {}	% originally ``Siehe Seiten''
}

\renewcommand*{\finentrypunct}{}
\renewbibmacro*{pageref}{%
	\addperiod
	\iflistundef{pageref}
	{}
	{\printtext[brackets]{% NEW
			\ifnumgreater{\value{pageref}}{1}
			{\bibstring{backrefpages}}
			{\bibstring{backrefpage}}%
			\printlist[pageref][-\value{listtotal}]{pageref}%
		}%
	}%
}% NEW

% Link DOIs automatically to https://doi.org/<DOI>, in line with the
% APA's DOI Display Guidelines Update from March 2017:
% https://blog.apastyle.org/apastyle/digital-object-identifier-doi/
% ==>
\DeclareFieldFormat{doi}{%
	\Urlmuskip = 0mu\relax%
	\mathchardef\UrlBigBreakPenalty=1\relax%
	\mathchardef\UrlBreakPenalty=2\relax%
		% See https://tex.stackexchange.com/questions/22854/url-line-breaks-with-biblatex
	\url{https://doi.org/#1}%
}
% <==
\DeclareFieldFormat{url}{%
	\Urlmuskip = 0mu\relax%
	\mathchardef\UrlBigBreakPenalty=1\relax%
	\mathchardef\UrlBreakPenalty=2\relax%
		% See https://tex.stackexchange.com/questions/22854/url-line-breaks-with-biblatex
	\url{#1}%
}
\AtEveryBibitem{\iffieldundef{doi}{}{\clearfield{eprint}}}  % If DOI provided, remove eprint
\AtEveryBibitem{\iffieldundef{doi}{}{\clearfield{url}}}  % If DOI provided, remove URL
% !TeX program = pdflatex
% !TeX TXS-program:compile = txs:///pdflatex/
% !TeX TS-program = pdflatex
% !BIB program = biber
% !TeX TXS-program:bibliography = txs:///biber




%%%%%%%%%%%%%%%%%%%%%%%%%%%
%%  COMMENTING COMMANDS  %%
%%%%%%%%%%%%%%%%%%%%%%%%%%%


\let\comment\undefined  % in case \comment{...} is already defined, e.g., by verbatim

\usepackage[todonotes={textsize=scriptsize}, authormarkuptext=name, authormarkup=none]{changes}

\makeatletter
% Make the ``changes'' boxes solid and less round:
\tikzstyle{notestyleraw} = [
	draw=\@todonotes@currentbordercolor,
	fill=\@todonotes@currentbordercolor,
	text=white,
	line width = 0.35pt,
	text width = \@todonotes@textwidth - 1.5ex - 1pt,
	inner sep = 0.75ex,
	rounded corners = 0pt
]
% Redefine default color (which is used when the author ID is omitted):
\colorlet{authorcolor}{magenta}
\@namedef{Changes@AuthorColor}{magenta}
\colorlet{Changes@Color}{magenta}
% Decrease font size of the ``changes'' boxes and use sans-serif font:
\newcommand{\changesfontsettings}{\sffamily\scriptsize\baselineskip=2.25ex}
\providecommand{\@todonotes@useSizeCommand}{\changesfontsettings}
% Decrease rule thickness slightly:
\xpatchcmd{\@todonotes@drawLineToLeftMargin} {connectstyle}{connectstyle, line width=0.6pt}{}{}
\xpatchcmd{\@todonotes@drawLineToRightMargin}{connectstyle}{connectstyle, line width=0.6pt}{}{}
\makeatother

\definecolor{electricultramarine}{rgb}{0.25, 0.0, 1.0}
\definecolor{alizarin}{rgb}{0.82, 0.1, 0.26}
\definecolor{dartmouthgreen}{rgb}{0.05, 0.5, 0.06}
\definecolor{goldenpoppy}{HTML}{FCC200}
\definecolor{internationalorange}{rgb}{1.0, 0.31, 0.0}
\definechangesauthor[name={Holger}, color=alizarin]           {HG}
\definechangesauthor[name={Lou~E.}, color=internationalorange]{LV}
\definechangesauthor[name={U.~R.},  color=electricultramarine]{UR}

% Make the \todo lines solid (instead of 70% opacity) and introduce a ``tickmark''
\LetLtxMacro{\todoorig}{\todo}
\makeatletter
\renewcommand{\todo}[2][]{%
	\todoorig[color=Changes@Color, bordercolor=Changes@Color, #1, tickmarkheight=0.1cm, linecolor=authorcolor]{#2}%
}
\makeatother

\LetLtxMacro{\commentorig}{\comment}
% Workaround for bug in the changes package (as of 2021-03-22, see https://gitlab.com/ekleinod/changes/-/issues/97): -->
\newsavebox{\commentbox}
\newlength{\commentboxwidth}
\makeatletter
\setlength{\commentboxwidth}{\@todonotes@textwidth - 1.5ex - 1pt}
\newcommand{\commentpatched}[2][]{%
	\savebox{\commentbox}[\commentboxwidth][t]{%
		\parbox{\commentboxwidth}{%
			\changesfontsettings\color{white}\RaggedRight%
			\hrule\smallskip%
			#2%
		}%
	}%
	\commentorig[#1]{%
		\strut\newline%
		\usebox{\commentbox}%
	}%
}
% <--
\makeatother
\renewcommand{\comment}[2][]{%
	\commentpatched[#1]{#2}%
}

% Redefine \added, \deleted, \replaced, and \highlight from https://ctan.org/tex-archive/macros/latex/contrib/changes
% ==>

\let \added \undefined
\let \deleted \undefinded
\let \replaced \undefinded
\let \highlight \undefinded
\setdeletedmarkup{}

\makeatletter
\newcommand{\lineabovecomment}[1]{%
	\setbox0=\hbox{#1\unskip}%
	\ifdim\wd0=0pt%
		% empty
	\else%
		\hrule\smallskip\strut%
	\fi%
}
\newcommand{\insertchangesadded}{%
	\smallskip\hrule\smallskip%
	\strut\textit{\mbox{\changesaddedname}}\\[\smallskipamount]
	\lineabovecomment{\Changes@added@comment}%
}%  % NEW
\newcommand{\insertchangesdeleted}[1]{%
	\smallskip\hrule\smallskip%
	\strut\textit{\mbox{\changesdeletedname:}} #1\\[\smallskipamount]
	\lineabovecomment{\Changes@deleted@comment}%
}%  % NEW
\newcommand{\insertchangesreplaced}[1]{%
	\smallskip\hrule\smallskip%
	\strut\textit{\mbox{\changesreplacedname:}} #1\\[\smallskipamount]
	\lineabovecomment{\Changes@replaced@comment}%
}%  % NEW
\newcommand{\insertchangeshighlight}{%
	\strut\lineabovecomment{\Changes@highlight@comment}%
}
\Changes@set@commandname{added}
\expandafter\newcommand\expandafter{\csname\Changes@commandname\endcsname}[2][\@empty]{%
	\setkeys{Changes@added}{#1}%
	\Changes@output%
		{added}%
		{\Changes@added@id}%
		{#2}%
		{}%
		{\insertchangesadded\Changes@added@comment}%
		{\changesaddedname}%
		{#2}%
}
\Changes@set@commandname{deleted}
\expandafter\newcommand\expandafter{\csname\Changes@commandname\endcsname}[2][\@empty]{%
	\setkeys{Changes@deleted}{#1}%
	\Changes@output%
		{deleted}%
		{\Changes@deleted@id}%
		{}%
		{}% NEW
		{\insertchangesdeleted{#2}\Changes@deleted@comment}%  % NEW
		{\changesdeletedname}%
		{#2}%
}
\Changes@set@commandname{replaced}
\expandafter\newcommand\expandafter{\csname\Changes@commandname\endcsname}[3][\@empty]{%
	\setkeys{Changes@replaced}{#1}%
	\Changes@output%
		{replaced}%
		{\Changes@replaced@id}%
		{#2}%
		{}  % NEW
		{\insertchangesreplaced{#3}\Changes@replaced@comment}%  % NEW
		{\changesreplacedname}%
		{#2}%
}
\Changes@set@commandname{highlight}
\expandafter\newcommand\expandafter{\csname\Changes@commandname\endcsname}[2][\@empty]{%
	\setkeys{Changes@highlight}{#1}%
	\Changes@output%
		{highlight}%
		{\Changes@highlight@id}%
		{#2}%
		{}%
		{\insertchangeshighlight\Changes@highlight@comment}%
		{\changeshighlightname}%
		{#2}%
}
\makeatother

% <==

\newcommand{\coloruline}[2]{%
	\newcommand{\tempuline}{%
		\bgroup\markoverwith{\textcolor{#1}{\rule[-1ex]{1.5pt}{0.25ex}}}%
		\ULon%
	}%
	\tempuline{#2}%
}
\sethighlightmarkup{%
	\colorlet{authorcoloraux}{authorcolor!33}%
	%\ifthenelse{\isColored}{\sethlcolor{authorcoloraux}}{}%
	\IfIsColored{%
		\sethlcolor{authorcoloraux}%
		%\makebox[0pt][l]{\coloruline{authorcolor}{\phantom{#1}}}
		\hl{#1\strut}%
	}{}%
}

\usepackage{pdfcomment}
\newenvironment{holgeradded}{%
	\color{alizarin}%
	\begin{pdfsidelinecomment}[color = alizarin, caption = inline, linebegin = {/None}, lineend = {/None}, linewidth = 2bp, linesep = 1cm]{Holger}\ignorespaces}%
	{\end{pdfsidelinecomment}%
}
% To ``accept'' all insertions in the document:
% \renewenvironment{holgeradded}{}{}
% !TeX program = pdflatex
% !TeX TXS-program:compile = txs:///pdflatex/
% !TeX TS-program = pdflatex
% !BIB program = biber
% !TeX TXS-program:bibliography = txs:///biber




%%%%%%%%%%%%%%%%%%%%%%%%%%%%%%%%%%%
%%  OTHER PACKAGES AND COMMANDS  %%
%%%%%%%%%%%%%%%%%%%%%%%%%%%%%%%%%%%


% Some math-related definitions

%\newcommand*{\coloneqq}{\mathrel{%
%	\mathrel{%
%		\raisebox{0.18ex}{\scalebox{0.85}{\ensuremath{:}}\hspace{-0.2pt}%
%	}%
%	=%
%}}
% Provided by the mathtools package

\newcommand{\Corr}{\operatorname{Corr}}
\newcommand{\Cov} {\operatorname{Cov}}
\newcommand{\E}   {\operatorname{E}}
\newcommand{\Var} {\operatorname{Var}}

\newcommand{\dd}  {\mathup{d}}  % Differential d
\newcommand{\e}   {\mathup{e}}  % Euler's e

\newcommand{\sigstar}{\raisebox{0.66ex}{\scalebox{0.95}{$\star$}}}

% Balanced/unbalanced sliders for text:
\newcommand{\bal}{\mbox{\caps{BAL}}\xspace}
\newcommand{\unbal}{\mbox{\caps{UNBAL}}\xspace}
\newcommand{\balA}[1][1]{\mbox{\caps{BAL}$^{\mathup{I}}_{#1:#1}$}\xspace}
\newcommand{\balB}[1][1]{\mbox{\caps{BAL}$^{\mathup{II}}_{#1:#1}$}\xspace}
\newcommand{\unbalA}[1][n]{\mbox{\caps{UNBAL}$^{\mathup{I}}_{1:#1}$}\xspace}
\newcommand{\unbalB}[1][n]{\mbox{\caps{UNBAL}$^{\mathup{II}}_{#1:1}$}\xspace}

% Balanced/unbalanced slider choice sets for math:
\newcommand{\CS}[1][C]{{\mathbf{#1}}}
\newcommand{\CbalA}[1][1]{\CS^{\mathup{BAL,\,I}}_{#1:#1}}
\newcommand{\CbalB}[1][1]{\CS^{\mathup{BAL,\,II}}_{#1:#1}}
\newcommand{\CunbalA}[1][n]{\CS^{\mathup{UNBAL,\,I}}_{1:#1}}
\newcommand{\CunbalB}[1][n]{\CS^{\mathup{UNBAL,\,II}}_{#1:1}}
\newcommand{\cse}[1][c]{{\mathbf{#1}}}
\newcommand{\cbalA}[1][1]{\cse^{\mathup{BAL,\,I}}_{#1:#1}}
\newcommand{\cbalB}[1][1]{\cse^{\mathup{BAL,\,II}}_{#1:#1}}
\newcommand{\cunbalA}[1][n]{\cse^{\mathup{UNBAL,\,I}}_{1:#1}}
\newcommand{\cunbalB}[1][n]{\cse^{\mathup{UNBAL,\,II}}_{#1:1}}

% Balanced/unbalanced choice lists for text:
\newcommand{\balCL}[1][1]{\mbox{\caps{BAL}$_{\mathup{CL}}$}\xspace}
\newcommand{\unbalCLA}[1][1]{\mbox{\caps{UNBAL}$^{\mathup{I}}_{\mathup{CL}}$}\xspace}
\newcommand{\unbalCLB}[1][1]{\mbox{\caps{UNBAL}$^{\mathup{II}}_{\mathup{CL}}$}\xspace}

% Balanced/unbalanced choice-list choice sets for math:
\newcommand{\CbalCL}{{\CS}^{\textup{BAL}}_{\textup{CL}}}
\newcommand{\CunbalCLA}{{\CS}^{\textup{UNBAL,\,I}}_{\textup{CL}}}
\newcommand{\CunbalCLB}{{\CS}^{\textup{UNBAL,\,II}}_{\textup{CL}}}
\newcommand{\cbalCL}{{\cse}^{\textup{BAL}}_{\textup{CL}}}
\newcommand{\cunbalCLA}{{\cse}^{\textup{UNBAL,\,I}}_{\textup{CL}}}
\newcommand{\cunbalCLB}{{\cse}^{\textup{UNBAL,\,II}}_{\textup{CL}}}




%%%%%%%%%%%%%%%%%%%%%%%%%%%%%%%%%%%%
%%  ILLUSTRATE THE BASELINE GRID  %%
%%%%%%%%%%%%%%%%%%%%%%%%%%%%%%%%%%%%


\usetikzlibrary{calc}
\newcommand{\displaybaselinegrid}{%
	\begin{tikzpicture}[remember picture, overlay, x = 1mm, y = \baselineskip]
		\draw[orange] ($(current page.north west) + (0mm, -1in-\topmargin-\headheight-\headsep)$) -- ($(current page.north east) + (0, -1in-\topmargin-\headheight-\headsep)$);
		\foreach \i in {-5,...,-1,0,1,2,...,\linesperpagedesired}{
			\draw[magenta!50] ($(current page.north west) + (0mm, -1in-\voffset-\topmargin-\headheight-\headsep-\topskip) + (0, -\i-2)$) -- ($(current page.north east) + (0mm, -1in-\voffset-\topmargin-\headheight-\headsep-\topskip) + (0, -\i-2)$);
		}
		\draw[orange] ($(current page.north west) + (0mm, -1in-\topmargin-\headheight-\headsep-\textheight)$) -- ($(current page.north east) + (0mm, -1in-\topmargin-\headheight-\headsep-\textheight)$);
	\end{tikzpicture}%
}




%%%%%%%%%%%%%%%%%%%%%%%%%%%%%
%%  LOAD THE .BIB FILE(S)  %%
%%%%%%%%%%%%%%%%%%%%%%%%%%%%%


%% If desired, set up deviations from the default bibliography settings:
%\ExecuteBibliographyOptions{
%	compresspages = false,
%	uniquename = true, uniquelist = true, maxbibnames = 99
%}
%% % Full author list on first citation, ``et al.'' only from second citation onwards:
%% See https://tex.stackexchange.com/questions/478750/biblatex-first-citation-all-names-after-that-et-al:
%\AtEveryCitekey{%
%  \ifciteseen
%    {}
%    {\defcounter{maxnames}{6}}%
%}

\bibliography{Library.bib}
%\addbibresource{.../filename.bib}  % Use to add further .bib files for the individual papers




%%%%%%%%%%%%
%%  BODY  %%
%%%%%%%%%%%%


\begin{document}


% Save original definitions of the referencing commands
\makeatletter
\let \labelorig \label
\let \label@in@display@orig \label@in@display%
\makeatother
\let \reforig \ref
\let \eqreforig \eqref
\let \autoreforig \autoref
\let \hyperreforig \hyperref
\let \pagereforig \pageref


\begin{refsection}

% !TeX TXS-program:compile = txs:///pdflatex/
% !TeX TS-program = pdflatex
% !TeX root = Diss-Template.tex




%%%%%%%%%%%%%%%%%%
%%  COVER PAGE  %%
%%%%%%%%%%%%%%%%%%


\frontmatter  % roman page numbering

\pdfbookmark[0]{Front Page}{frontpage}

\noindent%
\begin{center}
\begin{minipage}[t][0.99\textheight][t]{0.9\textwidth}
\title{\sffamily\bfseries\linespread{0.975}\LARGE\scalefont{1.11}%
	~\\[-10ex]
	\disstitle%
}
\author{\large\selectlanguage{ngerman}%
	\\[10ex]%	
	% \textsf{% Sans-serif font
		{\Large Inauguraldissertation}%
	% }
	\\[3ex]%
	zur Erlangung des Grades eines Doktors \\%
	der Wirtschaftswissenschaften \\[1.55ex]%
	durch \\[1.5ex]%
	die Rechts- und Staatswissenschaftliche Fakultät der \\%
	Rheinischen Friedrich-Wilhelms-Universität Bonn \\[3ex]%
	vorgelegt von \\[3ex]%
	\textbf{% Bold font
		{\Large \dissauthor}
	}
	\\[3ex]%
	aus \dissbornin
}
% Include date only after defense: -->
\ifnum \dissafterdefense=1%
	\date{%
		\vspace{15ex}%
		{\Large\dissyear}%
		\vfill%
	}%
\else%
	\date{%
		\vfill%
	}%
\fi
% <--

\maketitle
\end{minipage}
\end{center}

\thispagestyle{empty}




%%%%%%%%%%%%%%
%%  PAGE 2  %%
%%%%%%%%%%%%%%


% THIS PAGE IS NOT TO BE INCLUDED WHEN HANDING IN THE THESIS PRIOR TO THE DEFENSE.
% ONLY INCLUDE IT IN THE VERSION THAT IS TO BE PUBLISHED AFTER THE SUCCESSFUL DEFENSE!

\newpage
\thispagestyle{empty}

\ifnum \dissafterdefense=1
	\mbox{\hspace{0pt}}	% Needed for the \vfill to work
	\vfill
	
	\noindent\selectlanguage{ngerman}
	\begin{tabular}{@{}l@{\quad}l@{}}
		Dekan:                          & \dissdean             \\[\medskipamount]
		Erstreferent:                   & \dissfirstsupervisor  \\
		Zweitreferent:                  & \disssecondsupervisor \\[\medskipamount]
		Tag der m\"undlichen Pr\"ufung: & \dissdatedefense
	\end{tabular}
\else
	~
\fi




%%%%%%%%%%%%%%%%%%%%%%%%%%%%%%%%%%%%%%%%
%%  PAGE 3 AND AFTER: TOC, LOF, ETC.  %%
%%%%%%%%%%%%%%%%%%%%%%%%%%%%%%%%%%%%%%%%


\cleardoublepage

\selectlanguage{american}

\chapter{Acknowledgements}

%I would like to thank Holger Gerhardt for TeXnical assistance.

I blame all of you. Writing this book has been an exercise in sustained suffering. The casual reader may, perhaps, exempt herself from excessive guilt, but for those of you who have played the larger role in prolonging my agonies with your encouragement and support, well \dots\ you know who you are, and you owe me.%
\todo[color = magenta, bordercolor = magenta]{
	\textbf{Commenting is on!} \\
	To~switch it off, activate
	\mbox{\texttt{\textbackslash PassOptionsToPackage}}\\
	\mbox{\texttt{\{final\}\{changes\}}} and \\
	\mbox{\texttt{\textbackslash PassOptionsToPackage}}\\
	\mbox{\texttt{\{disable\}\{todonotes\}}} \\
	in the master file.%
}

\RaggedLeft
---Brendan Pietsch, assistant professor of religious studies \\
at Nazarbayev University in Astana, Kazakhstan

\medskip

\textit{Source:\:\url{https://www.timeshighereducation.com/blog/best-academic-acknowledgements-ever}}

\justifying

\bigskip

\noindent%
Regarding the formal requirements for submitting your dissertation to the Department of Economics of the University of Bonn, please see
\begin{itemize}
	\item \url{https://www.rsf.uni-bonn.de/dekanat/Formulare}.
\end{itemize}  In~particular, please consult
\begin{itemize}
	\item \url{https://www.rsf.uni-bonn.de/dekanat/Formulare/promotionsordnung-wiwi-vom-25.-april-2005} (\S\,10 lists the documents that need to be handed in when submitting the dissertation) and
	\item \url{https://www.rsf.uni-bonn.de/dekanat/Formulare/eidesstattliche-erklaerung}.
\end{itemize}
The leaflet
\begin{itemize}
	\item \url{https://www.rsf.uni-bonn.de/dekanat/Formulare/vorgaben-pflichtexemplare-wiwi}
\end{itemize}
describes the details to be included in the front matter of your dissertation and the details regarding the inclusion of the \caps{CV} at the end of the document.
\\
\\
\\
\dissauthor \\
Bonn, \disssubmitdate

\cleardoublepage

% Disable optical marginal alignment for the table of contents etc.
% so that the page numbers are aligned properly:
\microtypesetup{protrusion = false}

\pdfbookmark[0]{\contentsname}{toc}
\tableofcontents

\cleardoublepage

\phantomsection
\listoffigures
\addcontentsline{toc}{chapter}{\listfigurename}

\phantomsection
\listoftables
\addcontentsline{toc}{chapter}{\listtablename}

% Re-enable optical marginal alignment:
\microtypesetup{protrusion = true}

\cleardoublepage

\mainmatter  % Arabic page numbering
	% use \input here, because \subimport prevents generation of the ToC

% Any citation occurring in a figure or table caption has to be included here!
% Otherwise it will not appear correctly in the list of figures/tables.
% ==>
\nocite{Gerhardt2017}
% <==

\end{refsection}


%%%%%%%%%%%%%%%%%%%%%%%%%%%%%%  INTRODUCTION  %%%%%%%%%%%%%%%%%%%%%%%%%%%%%%


\begin{refsection}

\chapter*{Introduction}
\label{diss:intro}
\addcontentsline{toc}{chapter}{Introduction}
\chaptermark{Introduction}

\graphicspath{{0_Introduction_Dissertation/}}
% !TeX program = pdflatex
% !BIB program = biber



\renewcommand{\blindmarkup}[1]{\emph{#1}}
\blindmathfalse


\section{Introduction}
\label{sec:introduction}

\begin{quote}
``Most people can save a~few dollars a~day or even \$10 a~day,'' she said. ``That’s doable. But if you say, `Can you save \$300 a~month or a~couple of thousand dollars a year?' people will say, `Whoa.' Avoiding that `whoa,' which is the hesitancy that can derail planning, is what consultants like Ms.~Davidson are trying to do.'' \\
\upshape
\mbox{}\hfill---\textit{\citefield{Sullivan2016}{journaltitle}}, \citefield{Sullivan2016}[month]{month}~\citefield{Sullivan2016}{day}, \citefield{Sullivan2016}{year}
\end{quote}

%Textheight: \arabic{textheight}
%
%Baselineskip: \arabic{baselineskip}
%
%Linesperpagecurrent: \arabic{linesperpagecurrent}
%
%Linesperpagedesired: \arabic{linesperpagedesired}
%
%Baselinestretch: \arabic{baselinestretch}
%
%\printlength{\baselineskip} \printlength{\textheight} \printlength{\topskip}

This template uses the \href{https://en.wikipedia.org/wiki/Bitstream_Charter}{Charter} typeface for the body text. Charter is a~serif type\-face and was designed in 1987 by \href{https://en.wikipedia.org/wiki/Matthew_Carter}{Matthew Carter}. By contrast, all headings, tables, and captions are set in a~\highlight{sans-serif typeface}. The sans-serif typeface used in this document is \href{https://en.wikipedia.org/wiki/Fira_Sans}{Fira Sans}, designed by \href{https://en.wikipedia.org/wiki/Erik_Spiekermann}{Erik Spiekermann} and collaborators.

The math settings are adjusted in the preamble to the effect that mathematical formulas are automatically typeset in the same font as the surrounding text. That is, math in a~serif environment will be set in a~serif font, while math in a~sans-serif environment will use the sans-serif font. This is an~aesthetic choice that may not please everyone given that a~sans-serif font may be used in mathematical formulas to express a~particular meaning. These cases are, however, very rare.

Let us cite \replaced[id=HG]{a~couple of}{some} publications: \cite{Andersen2008, Andreoni2012, Balakrishnan2016, Lisi1995}. With the options set for BibLaTeX in the preamble, citations in the body text are \deleted[id=LV]{automatically }sorted chronologically---irrespective of the order of the ``citekeys'' in your input. In the list of references, entries are sorted alphabetically by author surname.\added[id=UR]{ Let's cite} \cite{Andersen2008} once more.

\Blindtext[3]

Some\added[id=HG, comment={We already included several references above.}]{ additional} references: See \cite{Sims2003, Gabaix2014} for models of ``rational inattention'' or ``goal-driven attention.'' See \cite{Bordalo2012, Bordalo2013, Koszegi2013, Taubinsky2014, Bushong2016} for models of ``stimulus-driven attention.''%
\comment[id=UR]{Check whether there are more recent publications!}

\blindmathtrue

\Blindtext[3]

In \autoref{sec:Methods}, we describe the \highlight[id=LV, comment={Italics?}]{design} of our study\deleted[id=HG, comment={Too wordy.}]{in detail}. We present the data analysis and our results in \autoref{sec:Results}. In \autoref{sec:Discussion}, we discuss the plausibility of potential alternative explanations. \autoref{sec:Conclusion} \replaced[id=LV, comment={Let's use the present tense throughout.}]{concludes}{will conclude}.

\begin{refcontext}[sorting=nyt]
	% Sort BIBLIOGRAPHY by alphabet (while CITATIONS are sorted by year)
	\phantomsection
	\printbibliography[heading=subbibliography]
\end{refcontext}

\end{refsection}


%%%%%%%%%%%%%%%%%%%%%%%%%%%%%%  CHAPTER 1  %%%%%%%%%%%%%%%%%%%%%%%%%%%%%%


\newcommand{\currentchapter}{Ch1}

% Redefine the referencing commands to be chapter-specific in order to
% avoid erroneous cross-references from one chapter to another ==>
\makeatletter
\renewcommand{\label}[1]{\labelorig{\currentchapter:#1}}
% Based on https://tex.stackexchange.com/a/135995:
\renewcommand{\label@in@display}[1]{\label@in@display@orig{\currentchapter:#1}}
\makeatother
\renewcommand{\ref}[1]{\reforig{\currentchapter:#1}}
\renewcommand{\eqref}[1]{(\ref{#1})}
\renewcommand{\autoref}[1]{\autoreforig{\currentchapter:#1}}
\renewcommand{\hyperref}[2][]{\hyperreforig[\currentchapter:#1]{#2}}
\renewcommand{\pageref}[1]{\pagereforig{\currentchapter:#1}}
% <==

\begin{refsection}

% Save old footnote definition
\makeatletter
\renewcommand{\@makefntext}[1]{\@makefntextorig{#1}}
\makeatother
\chapter
	[My Job Market Paper]
	{My Job Market Paper%
	 \setcounter{footnote}{1}\textsuperscript{\Large\fnsymbol{footnote}}%
	}
\label{diss:JMP}
% Use footnote symbols instead of numbers
\renewcommand{\thefootnote}{\fnsymbol{footnote}}
\footnotetext[1]{\:\protect% !TeX TS-program = pdflatex
% !BIB program = biber
This footnote can be used for acknowledgments. This is where you can express your gratitude to referees, editors, and colleagues for their valuable feedback and suggestions that helped improve your manuscript. Financial support by third parties can also be mentioned here.%}
% Switch back to the original footnote definition
\renewcommand{\thefootnote}{\thefootnoteorig}
\makeatletter
\renewcommand{\@makefntext}[1]{\@makefntextcustom{#1}}
\renewcommand{\@makefnmark}{\@makefnmarkorig}
\makeatother
\setcounter{footnote}{0}

\graphicspath{{1_Example_Content/}}
% \subimport enables execution of \input commands in subdocuments without having to adjust paths
% Do not use \subimport*{...} (i.e., the asterisk version), since this will lead to
% font issues.
\subimport{1_Example_Content/1_Introduction/}{Introduction}
\subimport{1_Example_Content/2_Methods/}{Design}
\subimport{1_Example_Content/2_Methods/}{Predictions_dissertation}
\subimport{1_Example_Content/3_Results/}{Results}
\subimport{1_Example_Content/3_Results/}{Structural_Estimation}
\subimport{1_Example_Content/4_Discussion/}{Discussion}
\subimport{1_Example_Content/5_Conclusion/}{Conclusion}

\begin{subappendices}
	\label{sec:appendix}
	\FloatBarrier
	\subimport{1_Example_Content/9_Appendix/}{Attention}
	\FloatBarrier
	\newpage
	\subimport{1_Example_Content/9_Appendix/}{Additional_Figures}
	\FloatBarrier
	\subimport{1_Example_Content/9_Appendix/}{siunitx_Examples}
	\clearpage
		% \clearpage must be issued INSIDE the subappendices environment
		% for the pagestyle to be applied correctly
\end{subappendices}

\begin{refcontext}[sorting=nyt]
	% Sort BIBLIOGRAPHY by alphabet (while CITATIONS are sorted by year)
	\phantomsection
	\printbibliography[heading=subbibliography]
\end{refcontext}

\end{refsection}


%%%%%%%%%%%%%%%%%%%%%%%%%%%%%%  CHAPTER 2  %%%%%%%%%%%%%%%%%%%%%%%%%%%%%%


\renewcommand{\currentchapter}{Ch2}

\begin{refsection}

% Save old footnote definition
\makeatletter
\renewcommand{\@makefntext}[1]{\@makefntextorig{#1}}
\makeatother
\chapter
	[My Second Paper Has a~Long Title That\nolinebreak\  Spans Two Lines]
	{My Second Paper Has a\nolinebreak\ Long Title That\nolinebreak\  Spans Two Lines%
	 \setcounter{footnote}{1}\textsuperscript{\Large\fnsymbol{footnote}}%
	}
\label{diss:MySecondPaper}
% Use footnote symbols instead of numbers
\renewcommand{\thefootnote}{\fnsymbol{footnote}}
\footnotetext[1]{\:\protect% !TeX TS-program = pdflatex
% !BIB program = biber
This footnote can be used for acknowledgments. This is where you can express your gratitude to referees, editors, and colleagues for their valuable feedback and suggestions that helped improve your manuscript. Financial support by third parties can also be mentioned here.%}
% Switch back to the original footnote definition
\renewcommand{\thefootnote}{\thefootnoteorig}
\makeatletter
\renewcommand{\@makefntext}[1]{\@makefntextcustom{#1}}
\renewcommand{\@makefnmark}{\@makefnmarkorig}
\makeatother
\setcounter{footnote}{0}
\vspace{-\bigskipamount}\textsf{\textit{Joint with Adam Smith, Janet Smith, and Jeremiah Smith}} \bigskip

\graphicspath{{1_Example_Content/}}
\subimport{1_Example_Content/1_Introduction/}{Introduction}
\subimport{1_Example_Content/2_Methods/}{Design}
\subimport{1_Example_Content/2_Methods/}{Predictions_dissertation}
\subimport{1_Example_Content/3_Results/}{Results}
\subimport{1_Example_Content/3_Results/}{Structural_Estimation}
\subimport{1_Example_Content/4_Discussion/}{Discussion}
\subimport{1_Example_Content/5_Conclusion/}{Conclusion}

\begin{subappendices}
	\label{sec:appendix}
	\FloatBarrier
	\subimport{1_Example_Content/9_Appendix/}{Attention}
	\FloatBarrier
	\newpage
	\subimport{1_Example_Content/9_Appendix/}{Additional_Figures}
	\FloatBarrier
	\subimport{1_Example_Content/9_Appendix/}{siunitx_Examples}
	\clearpage
		% \clearpage must be issued INSIDE the subappendices environment
		% for the pagestyle to be applied correctly
\end{subappendices}

\begin{refcontext}[sorting=nyt]
	% Sort BIBLIOGRAPHY by alphabet (while CITATIONS are sorted by year)
	\phantomsection
	\printbibliography[heading=subbibliography]
\end{refcontext}

\end{refsection}


%%%%%%%%%%%%%%%%%%%%%%%%%%%%%%  CHAPTER 3  %%%%%%%%%%%%%%%%%%%%%%%%%%%%%%


\renewcommand{\currentchapter}{Ch3}

%\begin{comment}

\begin{refsection}

\chapter{Math Tests}
\label{diss:MathTests}

{%
	\setlength{\parindent}{0cm}
	%\setlength{\mathindent}{0cm}  % Only for ``fleqn'' option

	%%%%%%%%%%%%%%%%%%%%%%%%%%
%%  TESTING MATH FONTS  %%
%%%%%%%%%%%%%%%%%%%%%%%%%%

\newcommand{\dit}{\mathit{d}}
\newcommand{\dup}{\mathup{d}}

\def\test#1{#1}

\def\testnums{%
  \test 0 \test 1 \test 2 \test 3 \test 4 \test 5 \test 6 \test 7
  \test 8 \test 9 }
\def\testupperi{%
  \test A \test B \test C \test D \test E \test F \test G \test H
  \test I \test J \test K \test L \test M }
\def\testupperii{%
  \test N \test O \test P \test Q \test R \test S \test T \test U
  \test V \test W \test X \test Y \test Z }
\def\testupper{%
  \testupperi\testupperii}

\def\testloweri{%
  \test a \test b \test c \test d \test e \test f \test g \test h
  \test i \test j \test k \test l \test m }
\def\testlowerii{%
  \test n \test o \test p \test q \test r \test s \test t \test u
  \test v \test w \test x \test y \test z }
\def\testlower{%
  \testloweri\testlowerii}

\def\testupgreeki{%
  \test A \test B \test\Gamma \test\Delta \test E \test Z \test H
  \test\Theta \test I \test K \test\Lambda \test M }
\def\testupgreekii{%
  \test N \test\Xi \test O \test\Pi \test P \test\Sigma \test T
  \test\Upsilon \test\Phi \test X \test\Psi \test\Omega }
\def\testupgreek{%
  \testupgreeki\testupgreekii}

\def\testlowgreeki{%
  \test\alpha \test\beta \test\gamma \test\delta \test\epsilon
  \test\zeta \test\eta \test\theta \test\iota \test\kappa \test\lambda
  \test\mu }
\def\testlowgreekii{%
  \test\nu \test\xi \test o \test\pi \test\rho \test\sigma \test\tau
  \test\upsilon \test\phi \test\chi \test\psi \test\omega }
\def\testlowgreekiii{%
  \test\varepsilon \test\vartheta \test\varpi \test\varrho
  \test\varsigma \test\varphi}
\def\testlowgreek{%
  \testlowgreeki\testlowgreekii\testlowgreekiii}

	\renewcommand{\showfamily}{{\color{magenta}%
		Serif%
	}}
	{\rmfamily\mdseries%
	 \section{Math Test \showfamily}

\subsection{Overview \showfamily}

{\parindent 0pt
Default: $a \alpha \alphaup b \beta G \Gamma \upGamma \epsilon \varepsilon \theta \vartheta P \Pi \Sigma \sigma$; $\sigma_\epsilon, c^\alpha$

mathnormal: $\mathnormal{a \alpha \alphaup b \beta G \Gamma \upGamma \epsilon \varepsilon \theta \vartheta P \Pi \Sigma \sigma}$

mathrm: $\mathrm{a \alpha \alphaup b \beta G \Gamma \upGamma \epsilon \varepsilon \theta \vartheta P \Pi \Sigma \sigma}$

mathup: $\mathup{a \alpha \alphaup b \beta G \Gamma \upGamma \epsilon \varepsilon \theta \vartheta P \Pi \Sigma \sigma}$

mathit: $\mathit{a \alpha \alphaup b \beta G \Gamma \upGamma \epsilon \varepsilon \theta \vartheta P \Pi \Sigma \sigma}$

mathbf: $\mathbf{a \alphaup b \beta G \Gamma \upGamma \epsilon \varepsilon \theta \vartheta P \Pi \Sigma \sigma}$

mathbfit: $\mathbfit{a \alpha b \beta G \Gamma \upGamma \epsilon \varepsilon \theta \vartheta P \Pi \Sigma \sigma}$

mathbfup: $\mathbfup{a \alpha b \beta G \Gamma \upGamma \epsilon \varepsilon \theta \vartheta P \Pi \Sigma \sigma}$

\bigskip

{\bfseries
Default: $a \alpha \alphaup b \beta G \Gamma \upGamma \epsilon \varepsilon \theta \vartheta P \Pi \Sigma \sigma$; $\sigma_\epsilon, c^\alpha$

mathnormal: $\mathnormal{a \alpha \alphaup b \beta G \Gamma \upGamma \epsilon \varepsilon \theta \vartheta P \Pi \Sigma \sigma}$

mathrm: $\mathrm{a \alpha \alphaup b \beta G \Gamma \upGamma \epsilon \varepsilon \theta \vartheta P \Pi \Sigma \sigma}$

mathup: $\mathup{a \alpha \alphaup b \beta G \Gamma \upGamma \epsilon \varepsilon \theta \vartheta P \Pi \Sigma \sigma}$

mathit: $\mathit{a \alpha \alphaup b \beta G \Gamma \upGamma \epsilon \varepsilon \theta \vartheta P \Pi \Sigma \sigma}$

mathbf: $\mathbf{a \alpha \alphaup b \beta G \Gamma \upGamma \epsilon \varepsilon \theta \vartheta P \Pi \Sigma \sigma}$

mathbfit: $\mathbfit{a \alpha \alphaup b \beta G \Gamma \upGamma \epsilon \varepsilon \theta \vartheta P \Pi \Sigma \sigma}$

mathbfup: $\mathbfup{a \alpha \alphaup b \beta G \Gamma \upGamma \epsilon \varepsilon \theta \vartheta P \Pi \Sigma \sigma}$
}

\bigskip

{\sffamily\mdseries
Default: $a \alpha \alphaup b \beta G \Gamma \upGamma \epsilon \varepsilon \theta \vartheta P \Pi \Sigma \sigma$; $\sigma_\epsilon, c^\alpha$

mathnormal: $\mathnormal{a \alpha \alphaup b \beta G \Gamma \upGamma \epsilon \varepsilon \theta \vartheta P \Pi \Sigma \sigma}$

mathrm: $\mathrm{a \alpha \alphaup b \beta G \Gamma \upGamma \epsilon \varepsilon \theta \vartheta P \Pi \Sigma \sigma}$

mathup: $\mathup{a \alpha \alphaup b \beta G \Gamma \upGamma \epsilon \varepsilon \theta \vartheta P \Pi \Sigma \sigma}$

mathit: $\mathit{a \alpha \alphaup b \beta G \Gamma \upGamma \epsilon \varepsilon \theta \vartheta P \Pi \Sigma \sigma}$

mathbf: $\mathbf{a \alpha \alphaup b \beta G \Gamma \upGamma \epsilon \varepsilon \theta \vartheta P \Pi \Sigma \sigma}$

mathbfit: $\mathbfit{a \alpha \alphaup b \beta G \Gamma \upGamma \epsilon \varepsilon \theta \vartheta P \Pi \Sigma \sigma}$

mathbfup: $\mathbfup{a \alpha \alphaup b \beta G \Gamma \upGamma \epsilon \varepsilon \theta \vartheta P \Pi \Sigma \sigma}$
}

\bigskip

{\sffamily\bfseries

Default: $a \alpha \alphaup b \beta G \Gamma \upGamma \epsilon \varepsilon \theta \vartheta P \Pi \Sigma \sigma$; $\sigma_\epsilon, c^\alpha$

mathnormal: $\mathnormal{a \alpha \alphaup b \beta G \Gamma \upGamma \epsilon \varepsilon \theta \vartheta P \Pi \Sigma \sigma}$

mathrm: $\mathrm{a \alpha \alphaup b \beta G \Gamma \upGamma \epsilon \varepsilon \theta \vartheta P \Pi \Sigma \sigma}$

mathup: $\mathup{a \alpha \alphaup b \beta G \Gamma \upGamma \epsilon \varepsilon \theta \vartheta P \Pi \Sigma \sigma}$

mathit: $\mathit{a \alpha \alphaup b \beta G \Gamma \upGamma \epsilon \varepsilon \theta \vartheta P \Pi \Sigma \sigma}$

mathbf: $\mathbf{a \alpha \alphaup b \beta G \Gamma \upGamma \epsilon \varepsilon \theta \vartheta P \Pi \Sigma \sigma}$

mathbfit: $\mathbfit{a \alpha \alphaup b \beta G \Gamma \upGamma \epsilon \varepsilon \theta \vartheta P \Pi \Sigma \sigma}$

mathbfup: $\mathbfup{a \alpha \alphaup b \beta G \Gamma \upGamma \epsilon \varepsilon \theta \vartheta P \Pi \Sigma \sigma}$
}
}


\subsection{Formulas \showfamily}

\noindent%
\checkgreekletters

\noindent%
{\boldmath\checkgreekletters}

\noindent%
{\sffamily\selectfont \checkgreekletters}

\noindent%
{\sffamily\bfseries\selectfont \checkgreekletters}

\noindent%
{\sffamily $\alpha a > 0, \beta b + (3 \times 27), \Gamma G = 7 < 8, \lambda$}

\noindent%
$\alpha a > 0, \beta b + (3 \times 27), \Gamma G = 7 < 8, \lambda$

$\lim_{\nu \to \infty} v(\nu) = \max_{s \in S} \{s \pm 3 \gamma + y - 1\} = 4 \times 7$

$\hat{\beta} = (X'X)^{-1}X'y$

$$\lim_{N \to \infty} \sum_{i=0}^{N} x^i = \min_{x \in \mathbb{R}} S(x)$$

$$\int_{-\infty}^{\infty} x\,f(x)\,\mathup{d}x = \left( \frac{27}{2} \right)$$

Disambiguation: $0$~O~$O$, $1$~l~I~$|$~$l$~$I$~$/$, $i$~$j$, $rn$~$m$, $\theta$~$\Theta$, $\phi$~$\psi$, --~$-$

Latin vs. Greek: $a$~$\alpha$, $d$~$\delta$, $e$~$\epsilon$, $i$~$\iota$, $k$~$\kappa$, $n$~$\eta$, $o$~$\sigma$, $p$~$\rho$, \textit{\ss} $\beta$, $u$~$\upsilon$, $v$~$\nu$, $w$~$\omega$, $x$~$\chi$, $y$~$\gamma$, $A$~$\Delta$~$\Lambda$, $O$~$\Theta$~$\Omega$, $T$~$\Gamma$, $Y$~$\Upsilon$.

\noindent%
{\bfseries%
$\alpha a > 0, \beta b + (3 \times 27), \Gamma G = 7 < 8, \lambda$

$\lim_{\nu \to \infty} v(\nu) = \max_{s \in S} \{s \pm 3 \gamma + y - 1\} = 4 \times 7$

$\hat{\beta} = (X'X)^{-1}X'y$

$$\lim_{N \to \infty} \sum_{i=0}^{N} x^i = \min_{x \in \mathbb{R}} S(x)$$

$$\int_{-\infty}^{\infty} x\,f(x)\,\mathup{d}x = \left( \frac{27}{2} \right)$$

Disambiguation: $0$~O~$O$, $1$~l~I~$|$~$l$~$I$~$/$, $i$~$j$, $rn$~$m$, $\theta$~$\Theta$, $\phi$~$\psi$, --~$-$

Latin vs. Greek: $a$~$\alpha$, $d$~$\delta$, $e$~$\epsilon$, $i$~$\iota$, $k$~$\kappa$, $n$~$\eta$, $o$~$\sigma$, $p$~$\rho$, \textit{\ss} $\beta$, $u$~$\upsilon$, $v$~$\nu$, $w$~$\omega$, $x$~$\chi$, $y$~$\gamma$, $A$~$\Delta$~$\Lambda$, $O$~$\Theta$~$\Omega$, $T$~$\Gamma$, $Y$~$\Upsilon$.
}

\noindent%
{\sffamily%
$\alpha a > 0, \beta b + (3 \times 27), \Gamma G = 7 < 8, \lambda$

$\lim_{\nu \to \infty} v(\nu) = \max_{s \in S} \{s \pm 3 \gamma + y - 1\} = 4 \times 7$

$\hat{\beta} = (X'X)^{-1}X'y$

$$\lim_{N \to \infty} \sum_{i=0}^{N} x^i = \min_{x \in \mathbb{R}} S(x)$$

$$\int_{-\infty}^{\infty} x\,f(x)\,\mathup{d}x = \left( \frac{27}{2} \right)$$

Disambiguation: $0$~O~$O$, $1$~l~I~$|$~$l$~$I$~$/$, $i$~$j$, $rn$~$m$, $\theta$~$\Theta$, $\phi$~$\psi$, --~$-$

Latin vs. Greek: $a$~$\alpha$, $d$~$\delta$, $e$~$\epsilon$, $i$~$\iota$, $k$~$\kappa$, $n$~$\eta$, $o$~$\sigma$, $p$~$\rho$, \textit{\ss} $\beta$, $u$~$\upsilon$, $v$~$\nu$, $w$~$\omega$, $x$~$\chi$, $y$~$\gamma$, $A$~$\Delta$~$\Lambda$, $O$~$\Theta$~$\Omega$, $T$~$\Gamma$, $Y$~$\Upsilon$.
}

\noindent%
{\sffamily\bfseries%
$\alpha a > 0, \beta b + (3 \times 27), \Gamma G = 7 < 8, \lambda$

$\lim_{\nu \to \infty} v(\nu) = \max_{s \in S} \{s \pm 3 \gamma + y - 1\} = 4 \times 7$

$\hat{\beta} = (X'X)^{-1}X'y$

$$\lim_{N \to \infty} \sum_{i=0}^{N} x^i = \min_{x \in \mathbb{R}} S(x)$$

$$\int_{-\infty}^{\infty} x\,f(x)\,\mathup{d}x = \left( \frac{27}{2} \right)$$

Disambiguation: $0$~O~$O$, $1$~l~I~$|$~$l$~$I$~$/$, $i$~$j$, $rn$~$m$, $\theta$~$\Theta$, $\phi$~$\psi$, --~$-$

Latin vs. Greek: $a$~$\alpha$, $d$~$\delta$, $e$~$\epsilon$, $i$~$\iota$, $k$~$\kappa$, $n$~$\eta$, $o$~$\sigma$, $p$~$\rho$, \textit{\ss} $\beta$, $u$~$\upsilon$, $v$~$\nu$, $w$~$\omega$, $x$~$\chi$, $y$~$\gamma$, $A$~$\Delta$~$\Lambda$, $O$~$\Theta$~$\Omega$, $T$~$\Gamma$, $Y$~$\Upsilon$.
}


\subsection{Math Alphabets \showfamily}

%\sffamily\selectfont

Default
\def\test#1{#1,}
\begin{eqnarray*}
  && {\testnums}\\
  && {\testupper}\\
  && {\testlower}\\
  && {\testupgreek}\\
  && {\testlowgreek}
\end{eqnarray*}%

Math Normal (\texttt{\string\mathnormal})
\def\test#1{\mathnormal{#1},}
\begin{eqnarray*}
  && {\testnums}\\
  && {\testupper}\\
  && {\testlower}\\
  && {\testupgreek}\\
  && {\testlowgreek}
\end{eqnarray*}%

Math Italic (\texttt{\string\mathit})
\def\test#1{\mathit{#1},}
\begin{eqnarray*}
  && {\testnums}\\
  && {\testupper}\\
  && {\testlower}\\
  && {\testupgreek}\\
  && {\testlowgreek}
\end{eqnarray*}%

Math Roman (\texttt{\string\mathrm})
\def\test#1{\mathrm{#1},}
\begin{eqnarray*}
  && {\testnums}\\
  && {\testupper}\\
  && {\testlower}\\
  && {\testupgreek}\\
  && {\testlowgreek}
\end{eqnarray*}%

%Math Italic Bold (\texttt{\string\mathbm})
%\def\test#1{\mathbm{#1},}
%\begin{eqnarray*}
%  && {\testnums}\\
%  && {\testupper}\\
%  && {\testlower}\\
%  && {\testupgreek}\\
%  && {\testlowgreek}
%\end{eqnarray*}%

Math Bold (\texttt{\string\mathbf})
\def\test#1{\mathbf{#1},}
\begin{eqnarray*}
  && {\testnums}\\
  && {\testupper}\\
  && {\testlower}\\
  && {\testupgreek}\\
  && {\testlowgreek}
\end{eqnarray*}%

Caligraphic (\texttt{\string\mathcal})
\def\test#1{\mathcal{#1},}
\begin{eqnarray*}
  && {\testupper}
\end{eqnarray*}%

Script (\texttt{\string\mathscr})
\def\test#1{\mathscr{#1},}
\begin{eqnarray*}
  && {\testupper}
\end{eqnarray*}%

Fraktur (\texttt{\string\mathfrak})
\def\test#1{\mathfrak{#1},}
\begin{eqnarray*}
  && {\testupper}\\
  && {\testlower}
\end{eqnarray*}%

Blackboard Bold (\texttt{\string\mathbb})
\def\test#1{\mathbb{#1},}
\begin{eqnarray*}
  && {\testupper}
\end{eqnarray*}%

\subsection{Character Sidebearings \showfamily}

Default
\def\test#1{|#1|+{}}
\begin{eqnarray*}
  && {\testupperi}\\
  && {\testupperii}\\
  && {\testloweri}\\
  && {\testlowerii}\\
  && {\testupgreeki}\\
  && {\testupgreekii}\\
  && {\testlowgreeki}\\
  && {\testlowgreekii}\\
  && {\testlowgreekiii}
\end{eqnarray*}%

Math Roman (\texttt{\string\mathrm})
\def\test#1{|\mathrm{#1}|+{}}%
\begin{eqnarray*}
  && {\testupperi}\\
  && {\testupperii}\\
  && {\testloweri}\\
  && {\testlowerii}\\
  && {\testupgreeki}\\
  && {\testupgreekii}
\end{eqnarray*}%

%Math Italic Bold (\texttt{\string\mathbm})
%\def\test#1{|\mathbm{#1}|+{}}%
%\begin{eqnarray*}
%  && {\testupperi}\\
%  && {\testupperii}\\
%  && {\testloweri}\\
%  && {\testlowerii}\\
%  && {\testupgreeki}\\
%  && {\testupgreekii}\\
%  && {\testlowgreeki}\\
%  && {\testlowgreekii}\\
%  && {\testlowgreekiii}
%\end{eqnarray*}%

Math Bold (\texttt{\string\mathbf})
\def\test#1{|\mathbf{#1}|+{}}%
\begin{eqnarray*}
  && {\testupperi}\\
  && {\testupperii}\\
  && {\testloweri}\\
  && {\testlowerii}\\
  && {\testupgreeki}\\
  && {\testupgreekii}
\end{eqnarray*}%

Math Calligraphic (\texttt{\string\mathcal})
\def\test#1{|\mathcal{#1}|+{}}%
\begin{eqnarray*}
  && {\testupperi}\\
  && {\testupperii}
\end{eqnarray*}%


\subsection{Superscript Positioning \showfamily}

Default
\def\test#1{#1^{2}+{}}%
\begin{eqnarray*}
  && {\testupperi}\\
  && {\testupperii}\\
  && {\testloweri}\\
  && {\testlowerii}\\
  && {\testupgreeki}\\
  && {\testupgreekii}\\
  && {\testlowgreeki}\\
  && {\testlowgreekii}\\
  && {\testlowgreekiii}
\end{eqnarray*}%

Math Roman (\texttt{\string\mathrm})
\def\test#1{\mathrm{#1}^{2}+{}}%
\begin{eqnarray*}
  && {\testupperi}\\
  && {\testupperii}\\
  && {\testloweri}\\
  && {\testlowerii}\\
  && {\testupgreeki}\\
  && {\testupgreekii}
\end{eqnarray*}%

%Math Italic Bold (\texttt{\string\mathbm})
%\def\test#1{\mathbm{#1}^{2}+{}}%
%\begin{eqnarray*}
%  && {\testupperi}\\
%  && {\testupperii}\\
%  && {\testloweri}\\
%  && {\testlowerii}\\
%  && {\testupgreeki}\\
%  && {\testupgreekii}\\
%  && {\testlowgreeki}\\
%  && {\testlowgreekii}\\
%  && {\testlowgreekiii}
%\end{eqnarray*}%

Math Bold (\texttt{\string\mathbf})
\def\test#1{\mathbf{#1}^{2}+{}}%
\begin{eqnarray*}
  && {\testupperi}\\
  && {\testupperii}\\
  && {\testloweri}\\
  && {\testlowerii}\\
  && {\testupgreeki}\\
  && {\testupgreekii}
\end{eqnarray*}

Math Calligraphic (\texttt{\string\mathcal})
\def\test#1{\mathcal{#1}^{2}+{}}%
\begin{eqnarray*}
  && {\testupperi}\\
  && {\testupperii}
\end{eqnarray*}%


\subsection{Subscript Positioning \showfamily}

Default
\def\test#1{\mathnormal{#1}_{i}+{}}%
\begin{eqnarray*}
  && {\testupperi}\\
  && {\testupperii}\\
  && {\testloweri}\\
  && {\testlowerii}\\
  && {\testupgreeki}\\
  && {\testupgreekii}\\
  && {\testlowgreeki}\\
  && {\testlowgreekii}\\
  && {\testlowgreekiii}
\end{eqnarray*}%

Math Roman (\texttt{\string\mathrm})
\def\test#1{\mathrm{#1}_{i}+{}}%
\begin{eqnarray*}
  && {\testupperi}\\
  && {\testupperii}\\
  && {\testloweri}\\
  && {\testlowerii}\\
  && {\testupgreeki}\\
  && {\testupgreekii}
\end{eqnarray*}%

%Math Bold Italic (\texttt{\string\mathbm})
%\def\test#1{\mathbm{#1}_{i}+{}}%
%\begin{eqnarray*}
%  && {\testupperi}\\
%  && {\testupperii}\\
%  && {\testloweri}\\
%  && {\testlowerii}\\
%  && {\testupgreeki}\\
%  && {\testupgreekii}\\
%  && {\testlowgreeki}\\
%  && {\testlowgreekii}\\
%  && {\testlowgreekiii}
%\end{eqnarray*}

Math Bold (\texttt{\string\mathbf})
\def\test#1{\mathbf{#1}_{i}+{}}%
\begin{eqnarray*}
  && {\testupperi}\\
  && {\testupperii}\\
  && {\testloweri}\\
  && {\testlowerii}\\
  && {\testupgreeki}\\
  && {\testupgreekii}
\end{eqnarray*}%

Math Calligraphic (\texttt{\string\mathcal})
\def\test#1{\mathcal{#1}_{i}+{}}%
\begin{eqnarray*}
  && {\testupperi}\\
  && {\testupperii}
\end{eqnarray*}%


\subsection{Accent Positioning \showfamily}

Default
\def\test#1{\hat{#1}+{}}%
\begin{eqnarray*}
  && {\testnums}\\
  && {\testupperi}\\
  && {\testupperii}\\
  && {\testloweri}\\
  && {\testlowerii}\\
  && {\testupgreeki}\\
  && {\testupgreekii}\\
  && {\testlowgreeki}\\
  && {\testlowgreekii}\\
  && {\testlowgreekiii}
\end{eqnarray*}%

Math Italic (\texttt{\string\mathit})
\def\test#1{\hat{\mathit{#1}}+{}}%
\begin{eqnarray*}
  && {\testnums}\\
  && {\testupperi}\\
  && {\testupperii}\\
  && {\testloweri} \test\ell \test\wp \test\imath \test\jmath \tilde{i} \\
  && {\testlowerii}\\
  && {\testupgreeki}\\
  && {\testupgreekii}\\
  && {\testlowgreeki}\\
  && {\testlowgreekii}\\
  && {\testlowgreekiii}
\end{eqnarray*}%

Math Roman (\texttt{\string\mathrm})
\def\test#1{\hat{\mathrm{#1}}+{}}%
\begin{eqnarray*}
  && {\testnums}\\
  && {\testupperi}\\
  && {\testupperii}\\
  && {\testloweri}\\
  && {\testlowerii}\\
  && {\testupgreeki}\\
  && {\testupgreekii}
\end{eqnarray*}%

%Math Italic Bold (\texttt{\string\mathbm})
%\def\test#1{\hat{\mathbm{#1}}+{}}%
%\begin{eqnarray*}
%  && {\testnums}\\
%  && {\testupperi}\\
%  && {\testupperii}\\
%  && {\testloweri}\\
%  && {\testlowerii}\\
%  && {\testupgreeki}\\
%  && {\testupgreekii}\\
%  && {\testlowgreeki}\\
%  && {\testlowgreekii}\\
%  && {\testlowgreekiii}
%\end{eqnarray*}%

Math Bold (\texttt{\string\mathbf})
\def\test#1{\hat{\mathbf{#1}}+{}}%
\begin{eqnarray*}
  && {\testnums}\\
  && {\testupperi}\\
  && {\testupperii}\\
  && {\testloweri}\\
  && {\testlowerii}\\
  && {\testupgreeki}\\
  && {\testupgreekii}
\end{eqnarray*}

Math Calligraphic (\texttt{\string\mathcal})
\def\test#1{\hat{\mathcal{#1}}+{}}%
\begin{eqnarray*}
  && {\testupperi}\\
  && {\testupperii}
\end{eqnarray*}%


\subsection{Differentials \showfamily}

\begin{eqnarray*}
\gdef\test#1{\dit #1+{}}%
  && {\testupperi}\\
  && {\testupperii}\\
  && {\testloweri}\\
  && {\testlowerii}\\
  && {\testupgreeki}\\
  && {\testupgreekii}\\
  && {\testlowgreeki}\\
  && {\testlowgreekii}\\
  && {\testlowgreekiii}\\
\gdef\test#1{\dit \mathrm{#1}+{}}%
  && {\testupgreeki}\\
  && {\testupgreekii}
\end{eqnarray*}%

\begin{eqnarray*}
\gdef\test#1{\dup #1+{}}%
  && {\testupperi}\\
  && {\testupperii}\\
  && {\testloweri}\\
  && {\testlowerii}\\
  && {\testupgreeki}\\
  && {\testupgreekii}\\
  && {\testlowgreeki}\\
  && {\testlowgreekii}\\
  && {\testlowgreekiii}\\
\gdef\test#1{\dup \mathrm{#1}+{}}%
  && {\testupgreeki}\\
  && {\testupgreekii}
\end{eqnarray*}%

\begin{eqnarray*}
\gdef\test#1{\partial #1+{}}%
  && {\testupperi}\\
  && {\testupperii}\\
  && {\testloweri}\\
  && {\testlowerii}\\
  && {\testupgreeki}\\
  && {\testupgreekii}\\
  && {\testlowgreeki}\\
  && {\testlowgreekii}\\
  && {\testlowgreekiii}\\
\gdef\test#1{\partial \mathrm{#1}+{}}%
  && {\testupgreeki}\\
  && {\testupgreekii}
\end{eqnarray*}%


\subsection{Slash Kerning \showfamily}

\def\test#1{1/#1+{}}
\begin{eqnarray*}
  && {\testupperi}\\
  && {\testupperii}\\
  && {\testloweri}\\
  && {\testlowerii}\\
  && {\testupgreeki}\\
  && {\testupgreekii}\\
  && {\testlowgreeki}\\
  && {\testlowgreekii}\\
  && {\testlowgreekiii}
\end{eqnarray*}

\def\test#1{#1/2+{}}
\begin{eqnarray*}
  && {\testupperi}\\
  && {\testupperii}\\
  && {\testloweri}\\
  && {\testlowerii}\\
  && {\testupgreeki}\\
  && {\testupgreekii}\\
  && {\testlowgreeki}\\
  && {\testlowgreekii}\\
  && {\testlowgreekiii}
\end{eqnarray*}


\subsection{(Big) Operators \showfamily}

\def\testop#1{#1_{i=1}^{n} x^{n} \quad}
$
	\testop\sum
	\testop\prod
	\testop\coprod
	\testop\int
	\testop\oint
$

\noindent%
$
	\testop\bigotimes
	\testop\bigoplus
	\testop\bigodot
	\testop\bigwedge
	\testop\bigvee
	\testop\biguplus
	\testop\bigcup
	\testop\bigcap
	\testop\bigsqcup
	% \testop\bigsqcap
$

\begin{displaymath}
  \testop\sum
  \testop\prod
  \testop\coprod
  \testop\int
  \testop\oint
\end{displaymath}
\begin{displaymath}
  \testop\bigotimes
  \testop\bigoplus
  \testop\bigodot
  \testop\bigwedge
  \testop\bigvee
  \testop\biguplus
  \testop\bigcup
  \testop\bigcap
  \testop\bigsqcup
% \testop\bigsqcap
\end{displaymath}


\subsection{Radicals \showfamily}

\begin{displaymath}
  \sqrt{x+y} \qquad \sqrt{x^{2}+y^{2}} \qquad
  \sqrt{x_{i}^{2}+y_{j}^{2}} \qquad
  \sqrt{\left(\frac{\cos x}{2}\right)} \qquad
  \sqrt{\left(\frac{\sin x}{2}\right)}
\end{displaymath}

\begingroup
\delimitershortfall-1pt
\begin{displaymath}
  \sqrt{\sqrt{\sqrt{\sqrt{\sqrt{\sqrt{\sqrt{x+y}}}}}}}
\end{displaymath}
\endgroup % \delimitershortfall


\subsection{Over- and Underbraces \showfamily}

\begin{displaymath}
  \overbrace{x} \quad
  \overbrace{x+y} \quad
  \overbrace{x^{2}+y^{2}} \quad
  \overbrace{x_{i}^{2}+y_{j}^{2}} \quad
  \underbrace{x} \quad
  \underbrace{x+y} \quad
  \underbrace{x_{i}+y_{j}} \quad
  \underbrace{x_{i}^{2}+y_{j}^{2}} \quad
\end{displaymath}


\subsection{Normal and Wide Accents \showfamily}

\begin{displaymath}
  \dot{x} \quad
  \ddot{x} \quad
  \vec{x} \quad
  \bar{x} \quad
  \overline{x} \quad
  \overline{xx} \quad
  \tilde{x} \quad
  \widetilde{x} \quad
  \widetilde{xx} \quad
  \widetilde{xxx} \quad
  \hat{x} \quad
  \widehat{x} \quad
  \widehat{xx} \quad
  \widehat{xxx} \quad
\end{displaymath}

\begin{displaymath}
  \hat{x} \quad
  \check{x} \quad
  \tilde{x} \quad
  \acute{x} \quad
  \grave{x} \quad
  \dot{x} \quad
  \ddot{x} \quad
  \breve{x} \quad
  \bar{x} \quad
  \vec{x} \quad
\end{displaymath}


\subsection{Long Arrows \showfamily}

\begin{displaymath}
  \leftarrow \mathrel{-} \rightarrow \quad
  \leftrightarrow \quad
  \longleftarrow  \quad
  \longrightarrow \quad
  \longleftrightarrow \quad
  \Leftarrow = \Rightarrow \quad
  \Leftrightarrow \quad
  \Longleftarrow  \quad
  \Longrightarrow \quad
  \Longleftrightarrow \quad
\end{displaymath}


\subsection{Left and Right Delimiters \showfamily}

\def\testdelim#1#2{ - #1 f #2 - }
\begin{displaymath}
  \testdelim()
  \testdelim[]
  \testdelim\lfloor\rfloor
  \testdelim\lceil\rceil
  \testdelim\langle\rangle
  \testdelim\{\}
\end{displaymath}

Using {\tt\string\left} and {\tt\string\right}.
\def\testdelim#1#2{ - \left#1 f \right#2 - }
\begin{displaymath}
  \testdelim()
  \testdelim[]
  \testdelim\lfloor\rfloor
  \testdelim\lceil\rceil
  \testdelim\langle\rangle
  \testdelim\{\}
% \testdelim\lgroup\rgroup
% \testdelim\lmoustache\rmoustache
\end{displaymath}
\begin{displaymath}
  \testdelim)(
  \testdelim][
  \testdelim//
  \testdelim\backslash\backslash
  \testdelim/\backslash
  \testdelim\backslash/
\end{displaymath}


\subsection{Big-g-g Delimiters \showfamily}

\def\testdelim#1#2{%
  - \left#1\left#1\left#1\left#1\left#1\left#1\left#1\left#1 -
  \right#2\right#2\right#2\right#2\right#2\right#2\right#2\right#2 -}

\begingroup
\delimitershortfall-1pt
\begin{displaymath}
  \testdelim\lfloor\rfloor
  \qquad
  \testdelim()
\end{displaymath}
\begin{displaymath}
  \testdelim\lceil\rceil
  \qquad
  \testdelim\{\}
\end{displaymath}
\begin{displaymath}
  \testdelim[]
  \qquad
  \testdelim\lgroup\rgroup
\end{displaymath}
\begin{displaymath}
  \testdelim\langle\rangle
  \qquad
  \testdelim\lmoustache\rmoustache
\end{displaymath}
\begin{displaymath}
  \testdelim\uparrow\downarrow \quad
  \testdelim\Uparrow\Downarrow \quad
\end{displaymath}
\endgroup % \delimitershortfall

\def\X#1{$x #1 y$ &\tt\string#1}
\def\Y#1{$\big#1$ &\tt\string#1}
\def\Z#1{$x #1 y$}
\def\W#1#2{$#1{#2}$ &\tt\string#1\string{#2\string}}


\subsection{Binary Operators \showfamily}

\begin{tabular}{*8l}
\X\pm           &\X\cap         &\X\diamond             &\X\oplus     \\
\X\mp           &\X\cup         &\X\bigtriangleup       &\X\ominus    \\
\X\times        &\X\uplus       &\X\bigtriangledown     &\X\otimes    \\
\X\div          &\X\sqcap       &\X\triangleleft        &\X\oslash    \\
\X\ast          &\X\sqcup       &\X\triangleright       &\X\odot      \\
\X\star         &\X\vee         &\X\lhd                 &\X\bigcirc   \\
\X\circ         &\X\wedge       &\X\rhd                 &\X\dagger    \\
\X\bullet       &\X\setminus    &\X\unlhd               &\X\ddagger   \\
\X\cdot         &\X\wr          &\X\unrhd               &\X\S         \\
\X+             &\X-            &\X\amalg               &\X\P
\end{tabular}


\subsection{Relations \showfamily}

\begin{tabular}{*8l}
\X\leq          &\X\geq         &\X\equiv       &\X\models      \\
\X\prec         &\X\succ        &\X\sim         &\X\perp        \\
\X\preceq       &\X\succeq      &\X\simeq       &\X\mid         \\
\X\ll           &\X\gg          &\X\asymp       &\X\parallel    \\
\X\subset       &\X\supset      &\X\approx      &\X\bowtie      \\
\X\subseteq     &\X\supseteq    &\X\cong        &\X\Join        \\
\X\sqsubset     &\X\sqsupset    &\X\neq         &\X\smile       \\
\X\sqsubseteq   &\X\sqsupseteq  &\X\doteq       &\X\frown       \\
\X\in           &\X\ni          &\X\propto      &\X=            \\
\X\vdash        &\X\dashv       &\X<            &\X>            \\
\X:
\end{tabular}


\subsection{Punctuation \showfamily}

\begin{tabular}{*{5}{lp{3.2em}}}
\X,     &\X;    &\X\colon       &\X\ldotp       &\X\cdotp
\end{tabular}


\subsection{Arrows \showfamily}

\begin{tabular}{*6l}
\X\leftarrow            &\X\longleftarrow       &\X\uparrow     \\
\X\Leftarrow            &\X\Longleftarrow       &\X\Uparrow     \\
\X\rightarrow           &\X\longrightarrow      &\X\downarrow   \\
\X\Rightarrow           &\X\Longrightarrow      &\X\Downarrow   \\
\X\leftrightarrow       &\X\longleftrightarrow  &\X\updownarrow \\
\X\Leftrightarrow       &\X\Longleftrightarrow  &\X\Updownarrow \\
\X\mapsto               &\X\longmapsto          &\X\nearrow     \\
\X\hookleftarrow        &\X\hookrightarrow      &\X\searrow     \\
\X\leftharpoonup        &\X\rightharpoonup      &\X\swarrow     \\
\X\leftharpoondown      &\X\rightharpoondown    &\X\nwarrow     \\
\X\rightleftharpoons    &\X\leadsto
\end{tabular}


\subsection{Miscellaneous Symbols \showfamily}

\begin{tabular}{*8l}
\X\ldots        &\X\cdots       &\X\vdots       &\X\ddots       \\
\X\aleph        &\X\prime       &\X\forall      &\X\infty       \\
\X\hbar         &\X\emptyset    &\X\exists      &\X\Box         \\
\X\imath        &\X\nabla       &\X\neg         &\X\Diamond     \\
\X\jmath        &\X\surd        &\X\flat        &\X\triangle    \\
\X\ell          &\X\top         &\X\natural     &\X\clubsuit    \\
\X\wp           &\X\bot         &\X\sharp       &\X\diamondsuit \\
\X\Re           &\X\|           &\X\backslash   &\X\heartsuit   \\
\X\Im           &\X\angle       &\X\partial     &\X\spadesuit   \\
\X\mho          &\X.            &\X|            &\X!
\end{tabular}


\subsection{Variable-Sized Operators \showfamily}

\begin{tabular}{*6l}
\X\sum          &\X\bigcap      &\X\bigodot     \\
\X\prod         &\X\bigcup      &\X\bigotimes   \\
\X\coprod       &\X\bigsqcup    &\X\bigoplus    \\
\X\int          &\X\bigvee      &\X\biguplus    \\
\X\oint         &\X\bigwedge
\end{tabular}


\subsection{Log-Like Operators \showfamily}

\begin{tabular}{*8l}
\Z\arccos &\Z\cos  &\Z\csc &\Z\exp &
           \Z\ker    &\Z\limsup &\Z\min &\Z\sinh \\
\Z\arcsin &\Z\cosh &\Z\deg &\Z\gcd &
           \Z\lg     &\Z\ln     &\Z\Pr  &\Z\sup  \\
\Z\arctan &\Z\cot  &\Z\det &\Z\hom &
           \Z\lim    &\Z\log    &\Z\sec &\Z\tan  \\
\Z\arg    &\Z\coth &\Z\dim &\Z\inf &
           \Z\liminf &\Z\max    &\Z\sin &\Z\tanh
\end{tabular}


\subsection{Delimiters \showfamily}

\begin{tabular}{*8l}
\X(             &\X)            &\X\uparrow     &\X\Uparrow     \\
\X[             &\X]            &\X\downarrow   &\X\Downarrow   \\
\X\{            &\X\}           &\X\updownarrow &\X\Updownarrow \\
\X\lfloor       &\X\rfloor      &\X\lceil       &\X\rceil       \\
\X\langle       &\X\rangle      &\X/            &\X\backslash   \\
\X|             &\X\|
\end{tabular}


\subsection{Large Delimiters \showfamily}

\begin{tabular}{*8l}
\Y\rmoustache&  \Y\lmoustache&  \Y\rgroup&      \Y\lgroup\\[5pt]
\Y\arrowvert&   \Y\Arrowvert&   \Y\bracevert
\end{tabular}


\subsection{Math Mode Accents \showfamily}

\begin{tabular}{*{10}l}
\W\hat{a}     &\W\acute{a}  &\W\bar{a}    &\W\dot{a}    &\W\breve{a}\\
\W\check{a}   &\W\grave{a}  &\W\vec{a}    &\W\ddot{a}   &\W\tilde{a}\\
\end{tabular}


\subsection{Miscellaneous Constructions \showfamily}

\begin{tabular}{*4l}
\W\widetilde{abc}       &\W\widehat{abc}                        \\
\W\overleftarrow{abc}   &\W\overrightarrow{abc}                 \\
\W\overline{abc}        &\W\underline{abc}                      \\
\W\overbrace{abc}       &\W\underbrace{abc}                     \\[5pt]
\W\sqrt{abc}            &$\sqrt[n]{abc}$&\verb|\sqrt[n]{abc}|   \\
$f'$&\verb|f'|          &$\frac{abc}{xyz}$&\verb|\frac{abc}{xyz}|
\end{tabular}


\subsection{AMS Delimiters \showfamily}

\begin{tabular}{*8l}
\X\ulcorner&\X\urcorner&\X\llcorner&\X\lrcorner
\end{tabular}


\subsection{AMS Arrows \showfamily}

\begin{tabular}{*8l}
\X\dashrightarrow       &\X\dashleftarrow
        \\ \X\leftleftarrows      &\X\leftrightarrows     \\
\X\Lleftarrow           &\X\twoheadleftarrow
        \\ \X\leftarrowtail       &\X\looparrowleft       \\
\X\leftrightharpoons    &\X\curvearrowleft
        \\ \X\circlearrowleft     &\X\Lsh                 \\
\X\upuparrows           &\X\upharpoonleft
        \\ \X\downharpoonleft     &\X\multimap            \\
\X\leftrightsquigarrow  &\X\rightrightarrows
        \\ \X\rightleftarrows     &\X\rightrightarrows    \\
\X\rightleftarrows      &\X\twoheadrightarrow
        \\ \X\rightarrowtail      &\X\looparrowright      \\
\X\rightleftharpoons    &\X\curvearrowright
        \\ \X\circlearrowright    &\X\Rsh                 \\
\X\downdownarrows       &\X\upharpoonright
        \\ \X\downharpoonright    &\X\rightsquigarrow
\end{tabular}


\subsection{AMS Negated Arrows \showfamily}

\begin{tabular}{*8l}
\X\nleftarrow   &\X\nrightarrow \\ \X\nLeftarrow  &\X\nRightarrow \\
\X\nleftrightarrow&\X\nLeftrightarrow
\end{tabular}


\subsection{AMS Greek \showfamily}

\begin{tabular}{*4l}
\X\digamma      &\X\varkappa
\end{tabular}


\subsection{AMS Hebrew \showfamily}

\begin{tabular}{*6l}
\X\beth &\X\daleth      &\X\gimel
\end{tabular}


\subsection{AMS Miscellaneous \showfamily}

\begin{tabular}{*8l}
\X\hbar         &\X\hslash      \\ \X\vartriangle &\X\triangledown      \\
\X\square       &\X\lozenge     \\ \X\circledS    &\X\angle             \\
\X\measuredangle&\X\nexists     \\ \X\mho         &\X\Finv$^u$          \\
\X\Game$^u$     &\X\Bbbk$^u$    \\ \X\backprime   &\X\varnothing        \\
\X\blacktriangle&\X\blacktriangledown \\ \X\blacksquare&\X\blacklozenge  \\
\X\bigstar      &\X\sphericalangle     \\ \X\complement  &\X\eth       \\
\X\diagup$^u$   &\X\diagdown$^u$
\end{tabular}

$^u$ Not defined in {\tt amssymb.sty}, define using the
\verb|\newsymbol|  command.


\subsection{AMS Binary Operators \showfamily}

\begin{tabular}{*8l}
\X\dotplus      &\X\smallsetminus \\ \X\Cap        &\X\Cup               \\
\X\barwedge     &\X\veebar      \\ \X\doublebarwedge&\X\boxminus        \\
\X\boxtimes     &\X\boxdot      \\ \X\boxplus     &\X\divideontimes     \\
\X\ltimes       &\X\rtimes      \\ \X\leftthreetimes&\X\rightthreetimes \\
\X\curlywedge   &\X\curlyvee    \\ \X\circleddash &\X\circledast        \\
\X\circledcirc  &\X\centerdot   \\ \X\intercal
\end{tabular}


\subsection{AMS Relations \showfamily}

\begin{tabular}{*2l}
\X\leqslant    \\\X\lesssim    \\
\X\approxeq    \\\X\lll        \\
\X\lesseqgtr   \\\X\doteqdot   \\
\X\fallingdotseq\\\X\backsimeq  \\
\X\Subset      \\\X\preccurlyeq\\
\X\precsim     \\\X\vartriangleleft\\
\X\vDash      \\\X\smallsmile \\
\X\bumpeq      \\\X\geqq       \\
\X\eqslantgtr  \\\X\gtrapprox  \\
\X\ggg         \\\X\gtreqless  \\
\X\eqcirc      \\\X\triangleq  \\
\X\thickapprox \\\X\Supset     \\
\X\succcurlyeq \\\X\succsim    \\
\X\vartriangleright\\\X\Vdash      \\
\X\shortparallel\\\X\pitchfork  \\
\X\blacktriangleleft \\\X\backepsilon\\
\X\because
\end{tabular}


\subsection{AMS Negated Relations \showfamily}

\begin{tabular}{*8l}
\X\nless        &\X\nleq        \\ \X\nleqslant   &\X\nleqq       \\
\X\lneq         &\X\lneqq       \\ \X\lvertneqq   &\X\lnsim       \\
\X\lnapprox     &\X\nprec       \\ \X\npreceq     &\X\precnsim    \\
\X\precnapprox  &\X\nsim        \\ \X\nshortmid   &\X\nmid        \\
\X\nvdash       &\X\nvDash      \\ \X\ntriangleleft&\X\ntrianglelefteq\\
\X\nsubseteq    &\X\subsetneq   \\ \X\varsubsetneq&\X\subsetneqq  \\
\X\varsubsetneqq&\X\ngtr        \\ \X\ngeq        &\X\ngeqslant   \\
\X\ngeqq        &\X\gneq        \\ \X\gneqq       &\X\gvertneqq   \\
\X\gnsim        &\X\gnapprox    \\ \X\nsucc       &\X\nsucceq     \\
\X\nsucceqq     &\X\succnsim    \\ \X\succnapprox &\X\ncong       \\
\X\nshortparallel&\X\nparallel  \\ \X\nvDash      &\X\nVDash      \\
\X\ntriangleright&\X\ntrianglerighteq \\ \X\nsupseteq&\X\nsupseteqq\\
\X\supsetneq    &\X\varsupsetneq \\ \X\supsetneqq  &\X\varsupsetneqq
\end{tabular}%
	 \subsection{Math ``Torture'' Test \showfamily}
	 Most of the following examples are taken from \textit{The \TeX book} \citep[][see \url{https://ctan.org/pkg/texbook}]{Knuth1984} and were adapted for \LaTeX\ from Karl Berry's torture test for plain \TeX\ math fonts.

\noindent $x + y - z$, \quad $x + y * z$, \quad $z * y / z$, \quad 
$(x+y)(x-y) = x^2 - y^2$, 

\noindent $x \times y \cdot z = [x\, y\, z]$, \quad $x\circ y \bullet z$, \quad
$x\cup y \cap z$, \quad $x\sqcup y \sqcap z$, \quad

\noindent $x \vee y \wedge z$, \quad $x\pm y\mp z$, \quad
$x=y/z$, \;\; $x:=y$, \;\; $x\le y \ne z$, \;\; $x \sim y \simeq z$
$x \equiv y \nequiv z$, \;\; $x\subset y \subseteq z$

\noindent $\sin2\theta=2\sin\theta\cos\theta$, \quad
$\hbox{O}(n\log n\log n)$, \quad
$\Pr(X>x)=\exp(-x/\mu)$,

\noindent $\bigl(x\in A(n)\bigm|x\in B(n)\bigr)$, \quad
$\bigcup_n X_n\bigm\|\bigcap_n Y_n$

% page 178

\noindent In-text matrices $\binom{1\,1}{0\,1}$ and $\bigl(\genfrac{}{}{0pt}{}{a}{1}\genfrac{}{}{0pt}{}{b}{m}\genfrac{}{}{0pt}{}{c}{n}\bigr)$.

% page 142

$$a_0+\frac1{\displaystyle a_1 +
	{\strut \frac1{\displaystyle a_2 +
			{\strut \frac1{\displaystyle a_3 +
					{\strut \frac1{\displaystyle a_4}}}}}}}$$

% page 143

$$\binom{p}{2}x^2y^{p-2} - \frac1{1 - x}\frac{1}{1 - x^2}
=
\frac{a+1}{b}\bigg/\frac{c+1}{d}.$$

%% page 145

$$\sqrt{1+\sqrt{1+\sqrt{1+\sqrt{1+\sqrt{1+x}}}}}$$

$$\sqrt[n]{1+\sqrt[k]{1+\sqrt[5]{1+\sqrt[4]{1+\sqrt[3]{1+x}}}}}$$

%% page 147

$$\left(\frac{\partial^2}{\partial x^2} + \frac{\partial^2}{\partial y^2}\right)
\bigl|\varphi(x+\mathup{i}y)\bigr|^2=0$$

%% page 149

% $$\pi(n)=\sum_{m=2}^n\left\lfloor\biggl(\sum_{k=1}^{m-1}\bigl
% \lfloor(m/k)\big/\lceil m/k\rceil\bigr\rfloor\biggr)^{-1}\right\rfloor.$$

$$\pi(n)=\sum_{m=2}^n\left\lfloor\Biggl(\sum_{k=1}^{m-1}\bigl
\lfloor(m/k)\big/\lceil m/k\rceil\bigr\rfloor\Biggr)^{-1}\right\rfloor.$$

% page 168

$$\int_0^\infty \frac{t - \mathup{i} b}{t^2 + b^2}e^{\mathup{i}at}\,\mathup{d}t=e^{ab}E_1(ab), \quad
a,b > 0.$$

% page 176

$$\mathbf{A} \coloneqq \begin{pmatrix}x-\lambda&1&0\\
0&x-\lambda&1\\
0&0&x-\lambda\end{pmatrix}.$$

$$\left\lgroup\begin{matrix}a&b&c\\ d&e&f\\\end{matrix}\right\rgroup
\left\lgroup\begin{matrix}u&x\cr v&y\cr w&z\end{matrix}\right\rgroup$$

% page 177

$$\mathbf{A} = \begin{pmatrix}a_{11}&a_{12}&\ldots&a_{1n}\\
a_{21}&a_{22}&\ldots&a_{2n}\\
\vdots&\vdots&\ddots&\vdots\\
a_{m1}&a_{m2}&\ldots&a_{mn}\end{pmatrix}$$

$$\mathbf{M}=\bordermatrix{&C&I&C'\cr
	C&1&0&0\cr I&b&1-b&0\cr C'&0&a&1-a}$$

%% page 186

$$\sum_{n=0}^\infty a_nz^n\quad\hbox{converges if}\quad
|z|<\Bigl(\limsup_{n\to\infty}\root n\of{|a_n|}\,\Bigr)^{-1}.$$

$$\frac{f(x+\mathup{\Delta} x)-f(x)}{\mathup{\Delta} x}\to f'(x)
\qquad \hbox{as $\mathup{\Delta} x\to0$.}$$

$$\|u_i\|=1,\qquad u_i\cdot u_j=0\quad\hbox{if $i\ne j$.}$$

%% page 191

$$\hbox{The confluent image of}\quad
\begin{Bmatrix}\hbox{an arc}\hfill\\\hbox{a circle}\hfill\\
\hbox{a fan}\hfill\\\end{Bmatrix}
\quad\hbox{is}\quad
\begin{Bmatrix}\hbox{an arc}\hfill\\
\hbox{an arc or a circle}\hfill\\
\hbox{a fan or an arc}\hfill\end{Bmatrix}.$$

%% page 191

\begin{align*}
T(n)\le T(2^{\lceil\lg n\rceil})
&\le c(3^{\lceil\lg n\rceil}-2^{\lceil\lg n\rceil})\\
&<3c\cdot3^{\lg n}\\
&=3c\,n^{\lg3}.
\end{align*}

%\begin{align*}
%\left\{%
%\begin{gathered}\alpha&=f(z)\\ \beta&=f(z^2)\\ \gamma&=f(z^3)
%\end{gathered}
%\right\}
%\qquad
%\left\{%
%\begin{gathered}
%x&=\alpha^2-\beta\\ y&=2\gamma
%\end{gathered}
%\right\}%
%\end{align*}

%$$\left\{
%\begin{align}
%\alpha&=f(z)\cr \beta&=f(z^2)\cr \gamma&=f(z^3)\\
%%\end{align}
%\right\}
%\qquad
%\left\{
%%\begin{align}
%x&=\alpha^2-\beta\cr y&=2\gamma\\
%\end{align}
%\right\}.$$
%%% page 192

\begin{align*}
\begin{aligned}
(x+y)(x-y)&=x^2-xy+yx-y^2\\
&=x^2-y^2\\
(x+y)^2&=x^2+2xy+y^2.
\end{aligned}
\end{align*}

%% page 192

\begin{align*}
\begin{aligned}
\left( \int\limits_{-\infty}^\infty \mathup{e}^{-x^2}\,\mathup{d}x \right)^2
&=\int_{-\infty}^\infty\int_{-\infty}^\infty \mathup{e}^{-(x^2+y^2)}\,\mathup{d}x\,\mathup{d}y\\
&=\int_0^{2\piup}\int_0^\infty \mathup{e}^{-r^2}\,\mathup{d}r\,\mathup{d}\theta\\
&=\int_0^{2\piup}\biggl(\mathup{e}^{-\frac{r^2}{2}}\biggl|_{r=0}^{r=\infty}\,\biggr)\,\mathup{d}\theta\\
&=\piup.
\end{aligned}
\end{align*}


%% page 197

$$\prod_{k\ge0}\frac{1}{(1-q^kz)}=
\sum_{n\ge0}z^n\bigg/\!\!\prod_{1\le k\le n}(1-q^k).$$

$$\sum_{\substack{\scriptstyle 0< i\le m\\\scriptstyle0<j\le n}}p(i,j) \,\ne
%
% $$\sum_{i=1}^p \sum_{j=1}^q \sum_{k=1}^r a_{ij} b_{jk} c_{ki}$$
%
\sum_{i=1}^p \sum_{j=1}^q \sum_{k=1}^r a_{ij} b_{jk} c_{ki} \,\ne
%
\sum_{\substack{\scriptstyle 1\le i\le p \\ \scriptstyle 1\le j\le q\\
		\scriptstyle 1\le k\le r}} a_{ij} b_{jk} c_{ki}$$

$$\max_{1\le n\le m}\log_2P_n \quad \hbox{and} \quad
\lim_{x\to0}\frac{\sin x}{x}=1$$
Inline math:
$\max_{1\le n\le m}\log_2P_n \quad \hbox{and} \quad
\lim_{x\to0}\frac{\sin x}{x}=1$
$$p_1(n)=\lim_{m\to\infty}\sum_{\nu=0}^\infty\bigl(1-\cos^{2m}(\nu!^n\piup/n)\bigr)$$
Inline math:
$p_1(n)=\lim_{m\to\infty}\sum_{\nu=0}^\infty\bigl(1-\cos^{2m}(\nu!^n\piup/n)\bigr)$%
 	}

	\renewcommand{\showfamily}{{\color{magenta}%
		Serif Bold%
	}}
	{\rmfamily\bfseries%
	 \section{Math Test \showfamily}

\subsection{Overview \showfamily}

{\parindent 0pt
Default: $a \alpha \alphaup b \beta G \Gamma \upGamma \epsilon \varepsilon \theta \vartheta P \Pi \Sigma \sigma$; $\sigma_\epsilon, c^\alpha$

mathnormal: $\mathnormal{a \alpha \alphaup b \beta G \Gamma \upGamma \epsilon \varepsilon \theta \vartheta P \Pi \Sigma \sigma}$

mathrm: $\mathrm{a \alpha \alphaup b \beta G \Gamma \upGamma \epsilon \varepsilon \theta \vartheta P \Pi \Sigma \sigma}$

mathup: $\mathup{a \alpha \alphaup b \beta G \Gamma \upGamma \epsilon \varepsilon \theta \vartheta P \Pi \Sigma \sigma}$

mathit: $\mathit{a \alpha \alphaup b \beta G \Gamma \upGamma \epsilon \varepsilon \theta \vartheta P \Pi \Sigma \sigma}$

mathbf: $\mathbf{a \alphaup b \beta G \Gamma \upGamma \epsilon \varepsilon \theta \vartheta P \Pi \Sigma \sigma}$

mathbfit: $\mathbfit{a \alpha b \beta G \Gamma \upGamma \epsilon \varepsilon \theta \vartheta P \Pi \Sigma \sigma}$

mathbfup: $\mathbfup{a \alpha b \beta G \Gamma \upGamma \epsilon \varepsilon \theta \vartheta P \Pi \Sigma \sigma}$

\bigskip

{\bfseries
Default: $a \alpha \alphaup b \beta G \Gamma \upGamma \epsilon \varepsilon \theta \vartheta P \Pi \Sigma \sigma$; $\sigma_\epsilon, c^\alpha$

mathnormal: $\mathnormal{a \alpha \alphaup b \beta G \Gamma \upGamma \epsilon \varepsilon \theta \vartheta P \Pi \Sigma \sigma}$

mathrm: $\mathrm{a \alpha \alphaup b \beta G \Gamma \upGamma \epsilon \varepsilon \theta \vartheta P \Pi \Sigma \sigma}$

mathup: $\mathup{a \alpha \alphaup b \beta G \Gamma \upGamma \epsilon \varepsilon \theta \vartheta P \Pi \Sigma \sigma}$

mathit: $\mathit{a \alpha \alphaup b \beta G \Gamma \upGamma \epsilon \varepsilon \theta \vartheta P \Pi \Sigma \sigma}$

mathbf: $\mathbf{a \alpha \alphaup b \beta G \Gamma \upGamma \epsilon \varepsilon \theta \vartheta P \Pi \Sigma \sigma}$

mathbfit: $\mathbfit{a \alpha \alphaup b \beta G \Gamma \upGamma \epsilon \varepsilon \theta \vartheta P \Pi \Sigma \sigma}$

mathbfup: $\mathbfup{a \alpha \alphaup b \beta G \Gamma \upGamma \epsilon \varepsilon \theta \vartheta P \Pi \Sigma \sigma}$
}

\bigskip

{\sffamily\mdseries
Default: $a \alpha \alphaup b \beta G \Gamma \upGamma \epsilon \varepsilon \theta \vartheta P \Pi \Sigma \sigma$; $\sigma_\epsilon, c^\alpha$

mathnormal: $\mathnormal{a \alpha \alphaup b \beta G \Gamma \upGamma \epsilon \varepsilon \theta \vartheta P \Pi \Sigma \sigma}$

mathrm: $\mathrm{a \alpha \alphaup b \beta G \Gamma \upGamma \epsilon \varepsilon \theta \vartheta P \Pi \Sigma \sigma}$

mathup: $\mathup{a \alpha \alphaup b \beta G \Gamma \upGamma \epsilon \varepsilon \theta \vartheta P \Pi \Sigma \sigma}$

mathit: $\mathit{a \alpha \alphaup b \beta G \Gamma \upGamma \epsilon \varepsilon \theta \vartheta P \Pi \Sigma \sigma}$

mathbf: $\mathbf{a \alpha \alphaup b \beta G \Gamma \upGamma \epsilon \varepsilon \theta \vartheta P \Pi \Sigma \sigma}$

mathbfit: $\mathbfit{a \alpha \alphaup b \beta G \Gamma \upGamma \epsilon \varepsilon \theta \vartheta P \Pi \Sigma \sigma}$

mathbfup: $\mathbfup{a \alpha \alphaup b \beta G \Gamma \upGamma \epsilon \varepsilon \theta \vartheta P \Pi \Sigma \sigma}$
}

\bigskip

{\sffamily\bfseries

Default: $a \alpha \alphaup b \beta G \Gamma \upGamma \epsilon \varepsilon \theta \vartheta P \Pi \Sigma \sigma$; $\sigma_\epsilon, c^\alpha$

mathnormal: $\mathnormal{a \alpha \alphaup b \beta G \Gamma \upGamma \epsilon \varepsilon \theta \vartheta P \Pi \Sigma \sigma}$

mathrm: $\mathrm{a \alpha \alphaup b \beta G \Gamma \upGamma \epsilon \varepsilon \theta \vartheta P \Pi \Sigma \sigma}$

mathup: $\mathup{a \alpha \alphaup b \beta G \Gamma \upGamma \epsilon \varepsilon \theta \vartheta P \Pi \Sigma \sigma}$

mathit: $\mathit{a \alpha \alphaup b \beta G \Gamma \upGamma \epsilon \varepsilon \theta \vartheta P \Pi \Sigma \sigma}$

mathbf: $\mathbf{a \alpha \alphaup b \beta G \Gamma \upGamma \epsilon \varepsilon \theta \vartheta P \Pi \Sigma \sigma}$

mathbfit: $\mathbfit{a \alpha \alphaup b \beta G \Gamma \upGamma \epsilon \varepsilon \theta \vartheta P \Pi \Sigma \sigma}$

mathbfup: $\mathbfup{a \alpha \alphaup b \beta G \Gamma \upGamma \epsilon \varepsilon \theta \vartheta P \Pi \Sigma \sigma}$
}
}


\subsection{Formulas \showfamily}

\noindent%
\checkgreekletters

\noindent%
{\boldmath\checkgreekletters}

\noindent%
{\sffamily\selectfont \checkgreekletters}

\noindent%
{\sffamily\bfseries\selectfont \checkgreekletters}

\noindent%
{\sffamily $\alpha a > 0, \beta b + (3 \times 27), \Gamma G = 7 < 8, \lambda$}

\noindent%
$\alpha a > 0, \beta b + (3 \times 27), \Gamma G = 7 < 8, \lambda$

$\lim_{\nu \to \infty} v(\nu) = \max_{s \in S} \{s \pm 3 \gamma + y - 1\} = 4 \times 7$

$\hat{\beta} = (X'X)^{-1}X'y$

$$\lim_{N \to \infty} \sum_{i=0}^{N} x^i = \min_{x \in \mathbb{R}} S(x)$$

$$\int_{-\infty}^{\infty} x\,f(x)\,\mathup{d}x = \left( \frac{27}{2} \right)$$

Disambiguation: $0$~O~$O$, $1$~l~I~$|$~$l$~$I$~$/$, $i$~$j$, $rn$~$m$, $\theta$~$\Theta$, $\phi$~$\psi$, --~$-$

Latin vs. Greek: $a$~$\alpha$, $d$~$\delta$, $e$~$\epsilon$, $i$~$\iota$, $k$~$\kappa$, $n$~$\eta$, $o$~$\sigma$, $p$~$\rho$, \textit{\ss} $\beta$, $u$~$\upsilon$, $v$~$\nu$, $w$~$\omega$, $x$~$\chi$, $y$~$\gamma$, $A$~$\Delta$~$\Lambda$, $O$~$\Theta$~$\Omega$, $T$~$\Gamma$, $Y$~$\Upsilon$.

\noindent%
{\bfseries%
$\alpha a > 0, \beta b + (3 \times 27), \Gamma G = 7 < 8, \lambda$

$\lim_{\nu \to \infty} v(\nu) = \max_{s \in S} \{s \pm 3 \gamma + y - 1\} = 4 \times 7$

$\hat{\beta} = (X'X)^{-1}X'y$

$$\lim_{N \to \infty} \sum_{i=0}^{N} x^i = \min_{x \in \mathbb{R}} S(x)$$

$$\int_{-\infty}^{\infty} x\,f(x)\,\mathup{d}x = \left( \frac{27}{2} \right)$$

Disambiguation: $0$~O~$O$, $1$~l~I~$|$~$l$~$I$~$/$, $i$~$j$, $rn$~$m$, $\theta$~$\Theta$, $\phi$~$\psi$, --~$-$

Latin vs. Greek: $a$~$\alpha$, $d$~$\delta$, $e$~$\epsilon$, $i$~$\iota$, $k$~$\kappa$, $n$~$\eta$, $o$~$\sigma$, $p$~$\rho$, \textit{\ss} $\beta$, $u$~$\upsilon$, $v$~$\nu$, $w$~$\omega$, $x$~$\chi$, $y$~$\gamma$, $A$~$\Delta$~$\Lambda$, $O$~$\Theta$~$\Omega$, $T$~$\Gamma$, $Y$~$\Upsilon$.
}

\noindent%
{\sffamily%
$\alpha a > 0, \beta b + (3 \times 27), \Gamma G = 7 < 8, \lambda$

$\lim_{\nu \to \infty} v(\nu) = \max_{s \in S} \{s \pm 3 \gamma + y - 1\} = 4 \times 7$

$\hat{\beta} = (X'X)^{-1}X'y$

$$\lim_{N \to \infty} \sum_{i=0}^{N} x^i = \min_{x \in \mathbb{R}} S(x)$$

$$\int_{-\infty}^{\infty} x\,f(x)\,\mathup{d}x = \left( \frac{27}{2} \right)$$

Disambiguation: $0$~O~$O$, $1$~l~I~$|$~$l$~$I$~$/$, $i$~$j$, $rn$~$m$, $\theta$~$\Theta$, $\phi$~$\psi$, --~$-$

Latin vs. Greek: $a$~$\alpha$, $d$~$\delta$, $e$~$\epsilon$, $i$~$\iota$, $k$~$\kappa$, $n$~$\eta$, $o$~$\sigma$, $p$~$\rho$, \textit{\ss} $\beta$, $u$~$\upsilon$, $v$~$\nu$, $w$~$\omega$, $x$~$\chi$, $y$~$\gamma$, $A$~$\Delta$~$\Lambda$, $O$~$\Theta$~$\Omega$, $T$~$\Gamma$, $Y$~$\Upsilon$.
}

\noindent%
{\sffamily\bfseries%
$\alpha a > 0, \beta b + (3 \times 27), \Gamma G = 7 < 8, \lambda$

$\lim_{\nu \to \infty} v(\nu) = \max_{s \in S} \{s \pm 3 \gamma + y - 1\} = 4 \times 7$

$\hat{\beta} = (X'X)^{-1}X'y$

$$\lim_{N \to \infty} \sum_{i=0}^{N} x^i = \min_{x \in \mathbb{R}} S(x)$$

$$\int_{-\infty}^{\infty} x\,f(x)\,\mathup{d}x = \left( \frac{27}{2} \right)$$

Disambiguation: $0$~O~$O$, $1$~l~I~$|$~$l$~$I$~$/$, $i$~$j$, $rn$~$m$, $\theta$~$\Theta$, $\phi$~$\psi$, --~$-$

Latin vs. Greek: $a$~$\alpha$, $d$~$\delta$, $e$~$\epsilon$, $i$~$\iota$, $k$~$\kappa$, $n$~$\eta$, $o$~$\sigma$, $p$~$\rho$, \textit{\ss} $\beta$, $u$~$\upsilon$, $v$~$\nu$, $w$~$\omega$, $x$~$\chi$, $y$~$\gamma$, $A$~$\Delta$~$\Lambda$, $O$~$\Theta$~$\Omega$, $T$~$\Gamma$, $Y$~$\Upsilon$.
}


\subsection{Math Alphabets \showfamily}

%\sffamily\selectfont

Default
\def\test#1{#1,}
\begin{eqnarray*}
  && {\testnums}\\
  && {\testupper}\\
  && {\testlower}\\
  && {\testupgreek}\\
  && {\testlowgreek}
\end{eqnarray*}%

Math Normal (\texttt{\string\mathnormal})
\def\test#1{\mathnormal{#1},}
\begin{eqnarray*}
  && {\testnums}\\
  && {\testupper}\\
  && {\testlower}\\
  && {\testupgreek}\\
  && {\testlowgreek}
\end{eqnarray*}%

Math Italic (\texttt{\string\mathit})
\def\test#1{\mathit{#1},}
\begin{eqnarray*}
  && {\testnums}\\
  && {\testupper}\\
  && {\testlower}\\
  && {\testupgreek}\\
  && {\testlowgreek}
\end{eqnarray*}%

Math Roman (\texttt{\string\mathrm})
\def\test#1{\mathrm{#1},}
\begin{eqnarray*}
  && {\testnums}\\
  && {\testupper}\\
  && {\testlower}\\
  && {\testupgreek}\\
  && {\testlowgreek}
\end{eqnarray*}%

%Math Italic Bold (\texttt{\string\mathbm})
%\def\test#1{\mathbm{#1},}
%\begin{eqnarray*}
%  && {\testnums}\\
%  && {\testupper}\\
%  && {\testlower}\\
%  && {\testupgreek}\\
%  && {\testlowgreek}
%\end{eqnarray*}%

Math Bold (\texttt{\string\mathbf})
\def\test#1{\mathbf{#1},}
\begin{eqnarray*}
  && {\testnums}\\
  && {\testupper}\\
  && {\testlower}\\
  && {\testupgreek}\\
  && {\testlowgreek}
\end{eqnarray*}%

Caligraphic (\texttt{\string\mathcal})
\def\test#1{\mathcal{#1},}
\begin{eqnarray*}
  && {\testupper}
\end{eqnarray*}%

Script (\texttt{\string\mathscr})
\def\test#1{\mathscr{#1},}
\begin{eqnarray*}
  && {\testupper}
\end{eqnarray*}%

Fraktur (\texttt{\string\mathfrak})
\def\test#1{\mathfrak{#1},}
\begin{eqnarray*}
  && {\testupper}\\
  && {\testlower}
\end{eqnarray*}%

Blackboard Bold (\texttt{\string\mathbb})
\def\test#1{\mathbb{#1},}
\begin{eqnarray*}
  && {\testupper}
\end{eqnarray*}%

\subsection{Character Sidebearings \showfamily}

Default
\def\test#1{|#1|+{}}
\begin{eqnarray*}
  && {\testupperi}\\
  && {\testupperii}\\
  && {\testloweri}\\
  && {\testlowerii}\\
  && {\testupgreeki}\\
  && {\testupgreekii}\\
  && {\testlowgreeki}\\
  && {\testlowgreekii}\\
  && {\testlowgreekiii}
\end{eqnarray*}%

Math Roman (\texttt{\string\mathrm})
\def\test#1{|\mathrm{#1}|+{}}%
\begin{eqnarray*}
  && {\testupperi}\\
  && {\testupperii}\\
  && {\testloweri}\\
  && {\testlowerii}\\
  && {\testupgreeki}\\
  && {\testupgreekii}
\end{eqnarray*}%

%Math Italic Bold (\texttt{\string\mathbm})
%\def\test#1{|\mathbm{#1}|+{}}%
%\begin{eqnarray*}
%  && {\testupperi}\\
%  && {\testupperii}\\
%  && {\testloweri}\\
%  && {\testlowerii}\\
%  && {\testupgreeki}\\
%  && {\testupgreekii}\\
%  && {\testlowgreeki}\\
%  && {\testlowgreekii}\\
%  && {\testlowgreekiii}
%\end{eqnarray*}%

Math Bold (\texttt{\string\mathbf})
\def\test#1{|\mathbf{#1}|+{}}%
\begin{eqnarray*}
  && {\testupperi}\\
  && {\testupperii}\\
  && {\testloweri}\\
  && {\testlowerii}\\
  && {\testupgreeki}\\
  && {\testupgreekii}
\end{eqnarray*}%

Math Calligraphic (\texttt{\string\mathcal})
\def\test#1{|\mathcal{#1}|+{}}%
\begin{eqnarray*}
  && {\testupperi}\\
  && {\testupperii}
\end{eqnarray*}%


\subsection{Superscript Positioning \showfamily}

Default
\def\test#1{#1^{2}+{}}%
\begin{eqnarray*}
  && {\testupperi}\\
  && {\testupperii}\\
  && {\testloweri}\\
  && {\testlowerii}\\
  && {\testupgreeki}\\
  && {\testupgreekii}\\
  && {\testlowgreeki}\\
  && {\testlowgreekii}\\
  && {\testlowgreekiii}
\end{eqnarray*}%

Math Roman (\texttt{\string\mathrm})
\def\test#1{\mathrm{#1}^{2}+{}}%
\begin{eqnarray*}
  && {\testupperi}\\
  && {\testupperii}\\
  && {\testloweri}\\
  && {\testlowerii}\\
  && {\testupgreeki}\\
  && {\testupgreekii}
\end{eqnarray*}%

%Math Italic Bold (\texttt{\string\mathbm})
%\def\test#1{\mathbm{#1}^{2}+{}}%
%\begin{eqnarray*}
%  && {\testupperi}\\
%  && {\testupperii}\\
%  && {\testloweri}\\
%  && {\testlowerii}\\
%  && {\testupgreeki}\\
%  && {\testupgreekii}\\
%  && {\testlowgreeki}\\
%  && {\testlowgreekii}\\
%  && {\testlowgreekiii}
%\end{eqnarray*}%

Math Bold (\texttt{\string\mathbf})
\def\test#1{\mathbf{#1}^{2}+{}}%
\begin{eqnarray*}
  && {\testupperi}\\
  && {\testupperii}\\
  && {\testloweri}\\
  && {\testlowerii}\\
  && {\testupgreeki}\\
  && {\testupgreekii}
\end{eqnarray*}

Math Calligraphic (\texttt{\string\mathcal})
\def\test#1{\mathcal{#1}^{2}+{}}%
\begin{eqnarray*}
  && {\testupperi}\\
  && {\testupperii}
\end{eqnarray*}%


\subsection{Subscript Positioning \showfamily}

Default
\def\test#1{\mathnormal{#1}_{i}+{}}%
\begin{eqnarray*}
  && {\testupperi}\\
  && {\testupperii}\\
  && {\testloweri}\\
  && {\testlowerii}\\
  && {\testupgreeki}\\
  && {\testupgreekii}\\
  && {\testlowgreeki}\\
  && {\testlowgreekii}\\
  && {\testlowgreekiii}
\end{eqnarray*}%

Math Roman (\texttt{\string\mathrm})
\def\test#1{\mathrm{#1}_{i}+{}}%
\begin{eqnarray*}
  && {\testupperi}\\
  && {\testupperii}\\
  && {\testloweri}\\
  && {\testlowerii}\\
  && {\testupgreeki}\\
  && {\testupgreekii}
\end{eqnarray*}%

%Math Bold Italic (\texttt{\string\mathbm})
%\def\test#1{\mathbm{#1}_{i}+{}}%
%\begin{eqnarray*}
%  && {\testupperi}\\
%  && {\testupperii}\\
%  && {\testloweri}\\
%  && {\testlowerii}\\
%  && {\testupgreeki}\\
%  && {\testupgreekii}\\
%  && {\testlowgreeki}\\
%  && {\testlowgreekii}\\
%  && {\testlowgreekiii}
%\end{eqnarray*}

Math Bold (\texttt{\string\mathbf})
\def\test#1{\mathbf{#1}_{i}+{}}%
\begin{eqnarray*}
  && {\testupperi}\\
  && {\testupperii}\\
  && {\testloweri}\\
  && {\testlowerii}\\
  && {\testupgreeki}\\
  && {\testupgreekii}
\end{eqnarray*}%

Math Calligraphic (\texttt{\string\mathcal})
\def\test#1{\mathcal{#1}_{i}+{}}%
\begin{eqnarray*}
  && {\testupperi}\\
  && {\testupperii}
\end{eqnarray*}%


\subsection{Accent Positioning \showfamily}

Default
\def\test#1{\hat{#1}+{}}%
\begin{eqnarray*}
  && {\testnums}\\
  && {\testupperi}\\
  && {\testupperii}\\
  && {\testloweri}\\
  && {\testlowerii}\\
  && {\testupgreeki}\\
  && {\testupgreekii}\\
  && {\testlowgreeki}\\
  && {\testlowgreekii}\\
  && {\testlowgreekiii}
\end{eqnarray*}%

Math Italic (\texttt{\string\mathit})
\def\test#1{\hat{\mathit{#1}}+{}}%
\begin{eqnarray*}
  && {\testnums}\\
  && {\testupperi}\\
  && {\testupperii}\\
  && {\testloweri} \test\ell \test\wp \test\imath \test\jmath \tilde{i} \\
  && {\testlowerii}\\
  && {\testupgreeki}\\
  && {\testupgreekii}\\
  && {\testlowgreeki}\\
  && {\testlowgreekii}\\
  && {\testlowgreekiii}
\end{eqnarray*}%

Math Roman (\texttt{\string\mathrm})
\def\test#1{\hat{\mathrm{#1}}+{}}%
\begin{eqnarray*}
  && {\testnums}\\
  && {\testupperi}\\
  && {\testupperii}\\
  && {\testloweri}\\
  && {\testlowerii}\\
  && {\testupgreeki}\\
  && {\testupgreekii}
\end{eqnarray*}%

%Math Italic Bold (\texttt{\string\mathbm})
%\def\test#1{\hat{\mathbm{#1}}+{}}%
%\begin{eqnarray*}
%  && {\testnums}\\
%  && {\testupperi}\\
%  && {\testupperii}\\
%  && {\testloweri}\\
%  && {\testlowerii}\\
%  && {\testupgreeki}\\
%  && {\testupgreekii}\\
%  && {\testlowgreeki}\\
%  && {\testlowgreekii}\\
%  && {\testlowgreekiii}
%\end{eqnarray*}%

Math Bold (\texttt{\string\mathbf})
\def\test#1{\hat{\mathbf{#1}}+{}}%
\begin{eqnarray*}
  && {\testnums}\\
  && {\testupperi}\\
  && {\testupperii}\\
  && {\testloweri}\\
  && {\testlowerii}\\
  && {\testupgreeki}\\
  && {\testupgreekii}
\end{eqnarray*}

Math Calligraphic (\texttt{\string\mathcal})
\def\test#1{\hat{\mathcal{#1}}+{}}%
\begin{eqnarray*}
  && {\testupperi}\\
  && {\testupperii}
\end{eqnarray*}%


\subsection{Differentials \showfamily}

\begin{eqnarray*}
\gdef\test#1{\dit #1+{}}%
  && {\testupperi}\\
  && {\testupperii}\\
  && {\testloweri}\\
  && {\testlowerii}\\
  && {\testupgreeki}\\
  && {\testupgreekii}\\
  && {\testlowgreeki}\\
  && {\testlowgreekii}\\
  && {\testlowgreekiii}\\
\gdef\test#1{\dit \mathrm{#1}+{}}%
  && {\testupgreeki}\\
  && {\testupgreekii}
\end{eqnarray*}%

\begin{eqnarray*}
\gdef\test#1{\dup #1+{}}%
  && {\testupperi}\\
  && {\testupperii}\\
  && {\testloweri}\\
  && {\testlowerii}\\
  && {\testupgreeki}\\
  && {\testupgreekii}\\
  && {\testlowgreeki}\\
  && {\testlowgreekii}\\
  && {\testlowgreekiii}\\
\gdef\test#1{\dup \mathrm{#1}+{}}%
  && {\testupgreeki}\\
  && {\testupgreekii}
\end{eqnarray*}%

\begin{eqnarray*}
\gdef\test#1{\partial #1+{}}%
  && {\testupperi}\\
  && {\testupperii}\\
  && {\testloweri}\\
  && {\testlowerii}\\
  && {\testupgreeki}\\
  && {\testupgreekii}\\
  && {\testlowgreeki}\\
  && {\testlowgreekii}\\
  && {\testlowgreekiii}\\
\gdef\test#1{\partial \mathrm{#1}+{}}%
  && {\testupgreeki}\\
  && {\testupgreekii}
\end{eqnarray*}%


\subsection{Slash Kerning \showfamily}

\def\test#1{1/#1+{}}
\begin{eqnarray*}
  && {\testupperi}\\
  && {\testupperii}\\
  && {\testloweri}\\
  && {\testlowerii}\\
  && {\testupgreeki}\\
  && {\testupgreekii}\\
  && {\testlowgreeki}\\
  && {\testlowgreekii}\\
  && {\testlowgreekiii}
\end{eqnarray*}

\def\test#1{#1/2+{}}
\begin{eqnarray*}
  && {\testupperi}\\
  && {\testupperii}\\
  && {\testloweri}\\
  && {\testlowerii}\\
  && {\testupgreeki}\\
  && {\testupgreekii}\\
  && {\testlowgreeki}\\
  && {\testlowgreekii}\\
  && {\testlowgreekiii}
\end{eqnarray*}


\subsection{(Big) Operators \showfamily}

\def\testop#1{#1_{i=1}^{n} x^{n} \quad}
$
	\testop\sum
	\testop\prod
	\testop\coprod
	\testop\int
	\testop\oint
$

\noindent%
$
	\testop\bigotimes
	\testop\bigoplus
	\testop\bigodot
	\testop\bigwedge
	\testop\bigvee
	\testop\biguplus
	\testop\bigcup
	\testop\bigcap
	\testop\bigsqcup
	% \testop\bigsqcap
$

\begin{displaymath}
  \testop\sum
  \testop\prod
  \testop\coprod
  \testop\int
  \testop\oint
\end{displaymath}
\begin{displaymath}
  \testop\bigotimes
  \testop\bigoplus
  \testop\bigodot
  \testop\bigwedge
  \testop\bigvee
  \testop\biguplus
  \testop\bigcup
  \testop\bigcap
  \testop\bigsqcup
% \testop\bigsqcap
\end{displaymath}


\subsection{Radicals \showfamily}

\begin{displaymath}
  \sqrt{x+y} \qquad \sqrt{x^{2}+y^{2}} \qquad
  \sqrt{x_{i}^{2}+y_{j}^{2}} \qquad
  \sqrt{\left(\frac{\cos x}{2}\right)} \qquad
  \sqrt{\left(\frac{\sin x}{2}\right)}
\end{displaymath}

\begingroup
\delimitershortfall-1pt
\begin{displaymath}
  \sqrt{\sqrt{\sqrt{\sqrt{\sqrt{\sqrt{\sqrt{x+y}}}}}}}
\end{displaymath}
\endgroup % \delimitershortfall


\subsection{Over- and Underbraces \showfamily}

\begin{displaymath}
  \overbrace{x} \quad
  \overbrace{x+y} \quad
  \overbrace{x^{2}+y^{2}} \quad
  \overbrace{x_{i}^{2}+y_{j}^{2}} \quad
  \underbrace{x} \quad
  \underbrace{x+y} \quad
  \underbrace{x_{i}+y_{j}} \quad
  \underbrace{x_{i}^{2}+y_{j}^{2}} \quad
\end{displaymath}


\subsection{Normal and Wide Accents \showfamily}

\begin{displaymath}
  \dot{x} \quad
  \ddot{x} \quad
  \vec{x} \quad
  \bar{x} \quad
  \overline{x} \quad
  \overline{xx} \quad
  \tilde{x} \quad
  \widetilde{x} \quad
  \widetilde{xx} \quad
  \widetilde{xxx} \quad
  \hat{x} \quad
  \widehat{x} \quad
  \widehat{xx} \quad
  \widehat{xxx} \quad
\end{displaymath}

\begin{displaymath}
  \hat{x} \quad
  \check{x} \quad
  \tilde{x} \quad
  \acute{x} \quad
  \grave{x} \quad
  \dot{x} \quad
  \ddot{x} \quad
  \breve{x} \quad
  \bar{x} \quad
  \vec{x} \quad
\end{displaymath}


\subsection{Long Arrows \showfamily}

\begin{displaymath}
  \leftarrow \mathrel{-} \rightarrow \quad
  \leftrightarrow \quad
  \longleftarrow  \quad
  \longrightarrow \quad
  \longleftrightarrow \quad
  \Leftarrow = \Rightarrow \quad
  \Leftrightarrow \quad
  \Longleftarrow  \quad
  \Longrightarrow \quad
  \Longleftrightarrow \quad
\end{displaymath}


\subsection{Left and Right Delimiters \showfamily}

\def\testdelim#1#2{ - #1 f #2 - }
\begin{displaymath}
  \testdelim()
  \testdelim[]
  \testdelim\lfloor\rfloor
  \testdelim\lceil\rceil
  \testdelim\langle\rangle
  \testdelim\{\}
\end{displaymath}

Using {\tt\string\left} and {\tt\string\right}.
\def\testdelim#1#2{ - \left#1 f \right#2 - }
\begin{displaymath}
  \testdelim()
  \testdelim[]
  \testdelim\lfloor\rfloor
  \testdelim\lceil\rceil
  \testdelim\langle\rangle
  \testdelim\{\}
% \testdelim\lgroup\rgroup
% \testdelim\lmoustache\rmoustache
\end{displaymath}
\begin{displaymath}
  \testdelim)(
  \testdelim][
  \testdelim//
  \testdelim\backslash\backslash
  \testdelim/\backslash
  \testdelim\backslash/
\end{displaymath}


\subsection{Big-g-g Delimiters \showfamily}

\def\testdelim#1#2{%
  - \left#1\left#1\left#1\left#1\left#1\left#1\left#1\left#1 -
  \right#2\right#2\right#2\right#2\right#2\right#2\right#2\right#2 -}

\begingroup
\delimitershortfall-1pt
\begin{displaymath}
  \testdelim\lfloor\rfloor
  \qquad
  \testdelim()
\end{displaymath}
\begin{displaymath}
  \testdelim\lceil\rceil
  \qquad
  \testdelim\{\}
\end{displaymath}
\begin{displaymath}
  \testdelim[]
  \qquad
  \testdelim\lgroup\rgroup
\end{displaymath}
\begin{displaymath}
  \testdelim\langle\rangle
  \qquad
  \testdelim\lmoustache\rmoustache
\end{displaymath}
\begin{displaymath}
  \testdelim\uparrow\downarrow \quad
  \testdelim\Uparrow\Downarrow \quad
\end{displaymath}
\endgroup % \delimitershortfall

\def\X#1{$x #1 y$ &\tt\string#1}
\def\Y#1{$\big#1$ &\tt\string#1}
\def\Z#1{$x #1 y$}
\def\W#1#2{$#1{#2}$ &\tt\string#1\string{#2\string}}


\subsection{Binary Operators \showfamily}

\begin{tabular}{*8l}
\X\pm           &\X\cap         &\X\diamond             &\X\oplus     \\
\X\mp           &\X\cup         &\X\bigtriangleup       &\X\ominus    \\
\X\times        &\X\uplus       &\X\bigtriangledown     &\X\otimes    \\
\X\div          &\X\sqcap       &\X\triangleleft        &\X\oslash    \\
\X\ast          &\X\sqcup       &\X\triangleright       &\X\odot      \\
\X\star         &\X\vee         &\X\lhd                 &\X\bigcirc   \\
\X\circ         &\X\wedge       &\X\rhd                 &\X\dagger    \\
\X\bullet       &\X\setminus    &\X\unlhd               &\X\ddagger   \\
\X\cdot         &\X\wr          &\X\unrhd               &\X\S         \\
\X+             &\X-            &\X\amalg               &\X\P
\end{tabular}


\subsection{Relations \showfamily}

\begin{tabular}{*8l}
\X\leq          &\X\geq         &\X\equiv       &\X\models      \\
\X\prec         &\X\succ        &\X\sim         &\X\perp        \\
\X\preceq       &\X\succeq      &\X\simeq       &\X\mid         \\
\X\ll           &\X\gg          &\X\asymp       &\X\parallel    \\
\X\subset       &\X\supset      &\X\approx      &\X\bowtie      \\
\X\subseteq     &\X\supseteq    &\X\cong        &\X\Join        \\
\X\sqsubset     &\X\sqsupset    &\X\neq         &\X\smile       \\
\X\sqsubseteq   &\X\sqsupseteq  &\X\doteq       &\X\frown       \\
\X\in           &\X\ni          &\X\propto      &\X=            \\
\X\vdash        &\X\dashv       &\X<            &\X>            \\
\X:
\end{tabular}


\subsection{Punctuation \showfamily}

\begin{tabular}{*{5}{lp{3.2em}}}
\X,     &\X;    &\X\colon       &\X\ldotp       &\X\cdotp
\end{tabular}


\subsection{Arrows \showfamily}

\begin{tabular}{*6l}
\X\leftarrow            &\X\longleftarrow       &\X\uparrow     \\
\X\Leftarrow            &\X\Longleftarrow       &\X\Uparrow     \\
\X\rightarrow           &\X\longrightarrow      &\X\downarrow   \\
\X\Rightarrow           &\X\Longrightarrow      &\X\Downarrow   \\
\X\leftrightarrow       &\X\longleftrightarrow  &\X\updownarrow \\
\X\Leftrightarrow       &\X\Longleftrightarrow  &\X\Updownarrow \\
\X\mapsto               &\X\longmapsto          &\X\nearrow     \\
\X\hookleftarrow        &\X\hookrightarrow      &\X\searrow     \\
\X\leftharpoonup        &\X\rightharpoonup      &\X\swarrow     \\
\X\leftharpoondown      &\X\rightharpoondown    &\X\nwarrow     \\
\X\rightleftharpoons    &\X\leadsto
\end{tabular}


\subsection{Miscellaneous Symbols \showfamily}

\begin{tabular}{*8l}
\X\ldots        &\X\cdots       &\X\vdots       &\X\ddots       \\
\X\aleph        &\X\prime       &\X\forall      &\X\infty       \\
\X\hbar         &\X\emptyset    &\X\exists      &\X\Box         \\
\X\imath        &\X\nabla       &\X\neg         &\X\Diamond     \\
\X\jmath        &\X\surd        &\X\flat        &\X\triangle    \\
\X\ell          &\X\top         &\X\natural     &\X\clubsuit    \\
\X\wp           &\X\bot         &\X\sharp       &\X\diamondsuit \\
\X\Re           &\X\|           &\X\backslash   &\X\heartsuit   \\
\X\Im           &\X\angle       &\X\partial     &\X\spadesuit   \\
\X\mho          &\X.            &\X|            &\X!
\end{tabular}


\subsection{Variable-Sized Operators \showfamily}

\begin{tabular}{*6l}
\X\sum          &\X\bigcap      &\X\bigodot     \\
\X\prod         &\X\bigcup      &\X\bigotimes   \\
\X\coprod       &\X\bigsqcup    &\X\bigoplus    \\
\X\int          &\X\bigvee      &\X\biguplus    \\
\X\oint         &\X\bigwedge
\end{tabular}


\subsection{Log-Like Operators \showfamily}

\begin{tabular}{*8l}
\Z\arccos &\Z\cos  &\Z\csc &\Z\exp &
           \Z\ker    &\Z\limsup &\Z\min &\Z\sinh \\
\Z\arcsin &\Z\cosh &\Z\deg &\Z\gcd &
           \Z\lg     &\Z\ln     &\Z\Pr  &\Z\sup  \\
\Z\arctan &\Z\cot  &\Z\det &\Z\hom &
           \Z\lim    &\Z\log    &\Z\sec &\Z\tan  \\
\Z\arg    &\Z\coth &\Z\dim &\Z\inf &
           \Z\liminf &\Z\max    &\Z\sin &\Z\tanh
\end{tabular}


\subsection{Delimiters \showfamily}

\begin{tabular}{*8l}
\X(             &\X)            &\X\uparrow     &\X\Uparrow     \\
\X[             &\X]            &\X\downarrow   &\X\Downarrow   \\
\X\{            &\X\}           &\X\updownarrow &\X\Updownarrow \\
\X\lfloor       &\X\rfloor      &\X\lceil       &\X\rceil       \\
\X\langle       &\X\rangle      &\X/            &\X\backslash   \\
\X|             &\X\|
\end{tabular}


\subsection{Large Delimiters \showfamily}

\begin{tabular}{*8l}
\Y\rmoustache&  \Y\lmoustache&  \Y\rgroup&      \Y\lgroup\\[5pt]
\Y\arrowvert&   \Y\Arrowvert&   \Y\bracevert
\end{tabular}


\subsection{Math Mode Accents \showfamily}

\begin{tabular}{*{10}l}
\W\hat{a}     &\W\acute{a}  &\W\bar{a}    &\W\dot{a}    &\W\breve{a}\\
\W\check{a}   &\W\grave{a}  &\W\vec{a}    &\W\ddot{a}   &\W\tilde{a}\\
\end{tabular}


\subsection{Miscellaneous Constructions \showfamily}

\begin{tabular}{*4l}
\W\widetilde{abc}       &\W\widehat{abc}                        \\
\W\overleftarrow{abc}   &\W\overrightarrow{abc}                 \\
\W\overline{abc}        &\W\underline{abc}                      \\
\W\overbrace{abc}       &\W\underbrace{abc}                     \\[5pt]
\W\sqrt{abc}            &$\sqrt[n]{abc}$&\verb|\sqrt[n]{abc}|   \\
$f'$&\verb|f'|          &$\frac{abc}{xyz}$&\verb|\frac{abc}{xyz}|
\end{tabular}


\subsection{AMS Delimiters \showfamily}

\begin{tabular}{*8l}
\X\ulcorner&\X\urcorner&\X\llcorner&\X\lrcorner
\end{tabular}


\subsection{AMS Arrows \showfamily}

\begin{tabular}{*8l}
\X\dashrightarrow       &\X\dashleftarrow
        \\ \X\leftleftarrows      &\X\leftrightarrows     \\
\X\Lleftarrow           &\X\twoheadleftarrow
        \\ \X\leftarrowtail       &\X\looparrowleft       \\
\X\leftrightharpoons    &\X\curvearrowleft
        \\ \X\circlearrowleft     &\X\Lsh                 \\
\X\upuparrows           &\X\upharpoonleft
        \\ \X\downharpoonleft     &\X\multimap            \\
\X\leftrightsquigarrow  &\X\rightrightarrows
        \\ \X\rightleftarrows     &\X\rightrightarrows    \\
\X\rightleftarrows      &\X\twoheadrightarrow
        \\ \X\rightarrowtail      &\X\looparrowright      \\
\X\rightleftharpoons    &\X\curvearrowright
        \\ \X\circlearrowright    &\X\Rsh                 \\
\X\downdownarrows       &\X\upharpoonright
        \\ \X\downharpoonright    &\X\rightsquigarrow
\end{tabular}


\subsection{AMS Negated Arrows \showfamily}

\begin{tabular}{*8l}
\X\nleftarrow   &\X\nrightarrow \\ \X\nLeftarrow  &\X\nRightarrow \\
\X\nleftrightarrow&\X\nLeftrightarrow
\end{tabular}


\subsection{AMS Greek \showfamily}

\begin{tabular}{*4l}
\X\digamma      &\X\varkappa
\end{tabular}


\subsection{AMS Hebrew \showfamily}

\begin{tabular}{*6l}
\X\beth &\X\daleth      &\X\gimel
\end{tabular}


\subsection{AMS Miscellaneous \showfamily}

\begin{tabular}{*8l}
\X\hbar         &\X\hslash      \\ \X\vartriangle &\X\triangledown      \\
\X\square       &\X\lozenge     \\ \X\circledS    &\X\angle             \\
\X\measuredangle&\X\nexists     \\ \X\mho         &\X\Finv$^u$          \\
\X\Game$^u$     &\X\Bbbk$^u$    \\ \X\backprime   &\X\varnothing        \\
\X\blacktriangle&\X\blacktriangledown \\ \X\blacksquare&\X\blacklozenge  \\
\X\bigstar      &\X\sphericalangle     \\ \X\complement  &\X\eth       \\
\X\diagup$^u$   &\X\diagdown$^u$
\end{tabular}

$^u$ Not defined in {\tt amssymb.sty}, define using the
\verb|\newsymbol|  command.


\subsection{AMS Binary Operators \showfamily}

\begin{tabular}{*8l}
\X\dotplus      &\X\smallsetminus \\ \X\Cap        &\X\Cup               \\
\X\barwedge     &\X\veebar      \\ \X\doublebarwedge&\X\boxminus        \\
\X\boxtimes     &\X\boxdot      \\ \X\boxplus     &\X\divideontimes     \\
\X\ltimes       &\X\rtimes      \\ \X\leftthreetimes&\X\rightthreetimes \\
\X\curlywedge   &\X\curlyvee    \\ \X\circleddash &\X\circledast        \\
\X\circledcirc  &\X\centerdot   \\ \X\intercal
\end{tabular}


\subsection{AMS Relations \showfamily}

\begin{tabular}{*2l}
\X\leqslant    \\\X\lesssim    \\
\X\approxeq    \\\X\lll        \\
\X\lesseqgtr   \\\X\doteqdot   \\
\X\fallingdotseq\\\X\backsimeq  \\
\X\Subset      \\\X\preccurlyeq\\
\X\precsim     \\\X\vartriangleleft\\
\X\vDash      \\\X\smallsmile \\
\X\bumpeq      \\\X\geqq       \\
\X\eqslantgtr  \\\X\gtrapprox  \\
\X\ggg         \\\X\gtreqless  \\
\X\eqcirc      \\\X\triangleq  \\
\X\thickapprox \\\X\Supset     \\
\X\succcurlyeq \\\X\succsim    \\
\X\vartriangleright\\\X\Vdash      \\
\X\shortparallel\\\X\pitchfork  \\
\X\blacktriangleleft \\\X\backepsilon\\
\X\because
\end{tabular}


\subsection{AMS Negated Relations \showfamily}

\begin{tabular}{*8l}
\X\nless        &\X\nleq        \\ \X\nleqslant   &\X\nleqq       \\
\X\lneq         &\X\lneqq       \\ \X\lvertneqq   &\X\lnsim       \\
\X\lnapprox     &\X\nprec       \\ \X\npreceq     &\X\precnsim    \\
\X\precnapprox  &\X\nsim        \\ \X\nshortmid   &\X\nmid        \\
\X\nvdash       &\X\nvDash      \\ \X\ntriangleleft&\X\ntrianglelefteq\\
\X\nsubseteq    &\X\subsetneq   \\ \X\varsubsetneq&\X\subsetneqq  \\
\X\varsubsetneqq&\X\ngtr        \\ \X\ngeq        &\X\ngeqslant   \\
\X\ngeqq        &\X\gneq        \\ \X\gneqq       &\X\gvertneqq   \\
\X\gnsim        &\X\gnapprox    \\ \X\nsucc       &\X\nsucceq     \\
\X\nsucceqq     &\X\succnsim    \\ \X\succnapprox &\X\ncong       \\
\X\nshortparallel&\X\nparallel  \\ \X\nvDash      &\X\nVDash      \\
\X\ntriangleright&\X\ntrianglerighteq \\ \X\nsupseteq&\X\nsupseteqq\\
\X\supsetneq    &\X\varsupsetneq \\ \X\supsetneqq  &\X\varsupsetneqq
\end{tabular}%
	 \subsection{Math ``Torture'' Test \showfamily}
	 Most of the following examples are taken from \textit{The \TeX book} \citep[][see \url{https://ctan.org/pkg/texbook}]{Knuth1984} and were adapted for \LaTeX\ from Karl Berry's torture test for plain \TeX\ math fonts.

\noindent $x + y - z$, \quad $x + y * z$, \quad $z * y / z$, \quad 
$(x+y)(x-y) = x^2 - y^2$, 

\noindent $x \times y \cdot z = [x\, y\, z]$, \quad $x\circ y \bullet z$, \quad
$x\cup y \cap z$, \quad $x\sqcup y \sqcap z$, \quad

\noindent $x \vee y \wedge z$, \quad $x\pm y\mp z$, \quad
$x=y/z$, \;\; $x:=y$, \;\; $x\le y \ne z$, \;\; $x \sim y \simeq z$
$x \equiv y \nequiv z$, \;\; $x\subset y \subseteq z$

\noindent $\sin2\theta=2\sin\theta\cos\theta$, \quad
$\hbox{O}(n\log n\log n)$, \quad
$\Pr(X>x)=\exp(-x/\mu)$,

\noindent $\bigl(x\in A(n)\bigm|x\in B(n)\bigr)$, \quad
$\bigcup_n X_n\bigm\|\bigcap_n Y_n$

% page 178

\noindent In-text matrices $\binom{1\,1}{0\,1}$ and $\bigl(\genfrac{}{}{0pt}{}{a}{1}\genfrac{}{}{0pt}{}{b}{m}\genfrac{}{}{0pt}{}{c}{n}\bigr)$.

% page 142

$$a_0+\frac1{\displaystyle a_1 +
	{\strut \frac1{\displaystyle a_2 +
			{\strut \frac1{\displaystyle a_3 +
					{\strut \frac1{\displaystyle a_4}}}}}}}$$

% page 143

$$\binom{p}{2}x^2y^{p-2} - \frac1{1 - x}\frac{1}{1 - x^2}
=
\frac{a+1}{b}\bigg/\frac{c+1}{d}.$$

%% page 145

$$\sqrt{1+\sqrt{1+\sqrt{1+\sqrt{1+\sqrt{1+x}}}}}$$

$$\sqrt[n]{1+\sqrt[k]{1+\sqrt[5]{1+\sqrt[4]{1+\sqrt[3]{1+x}}}}}$$

%% page 147

$$\left(\frac{\partial^2}{\partial x^2} + \frac{\partial^2}{\partial y^2}\right)
\bigl|\varphi(x+\mathup{i}y)\bigr|^2=0$$

%% page 149

% $$\pi(n)=\sum_{m=2}^n\left\lfloor\biggl(\sum_{k=1}^{m-1}\bigl
% \lfloor(m/k)\big/\lceil m/k\rceil\bigr\rfloor\biggr)^{-1}\right\rfloor.$$

$$\pi(n)=\sum_{m=2}^n\left\lfloor\Biggl(\sum_{k=1}^{m-1}\bigl
\lfloor(m/k)\big/\lceil m/k\rceil\bigr\rfloor\Biggr)^{-1}\right\rfloor.$$

% page 168

$$\int_0^\infty \frac{t - \mathup{i} b}{t^2 + b^2}e^{\mathup{i}at}\,\mathup{d}t=e^{ab}E_1(ab), \quad
a,b > 0.$$

% page 176

$$\mathbf{A} \coloneqq \begin{pmatrix}x-\lambda&1&0\\
0&x-\lambda&1\\
0&0&x-\lambda\end{pmatrix}.$$

$$\left\lgroup\begin{matrix}a&b&c\\ d&e&f\\\end{matrix}\right\rgroup
\left\lgroup\begin{matrix}u&x\cr v&y\cr w&z\end{matrix}\right\rgroup$$

% page 177

$$\mathbf{A} = \begin{pmatrix}a_{11}&a_{12}&\ldots&a_{1n}\\
a_{21}&a_{22}&\ldots&a_{2n}\\
\vdots&\vdots&\ddots&\vdots\\
a_{m1}&a_{m2}&\ldots&a_{mn}\end{pmatrix}$$

$$\mathbf{M}=\bordermatrix{&C&I&C'\cr
	C&1&0&0\cr I&b&1-b&0\cr C'&0&a&1-a}$$

%% page 186

$$\sum_{n=0}^\infty a_nz^n\quad\hbox{converges if}\quad
|z|<\Bigl(\limsup_{n\to\infty}\root n\of{|a_n|}\,\Bigr)^{-1}.$$

$$\frac{f(x+\mathup{\Delta} x)-f(x)}{\mathup{\Delta} x}\to f'(x)
\qquad \hbox{as $\mathup{\Delta} x\to0$.}$$

$$\|u_i\|=1,\qquad u_i\cdot u_j=0\quad\hbox{if $i\ne j$.}$$

%% page 191

$$\hbox{The confluent image of}\quad
\begin{Bmatrix}\hbox{an arc}\hfill\\\hbox{a circle}\hfill\\
\hbox{a fan}\hfill\\\end{Bmatrix}
\quad\hbox{is}\quad
\begin{Bmatrix}\hbox{an arc}\hfill\\
\hbox{an arc or a circle}\hfill\\
\hbox{a fan or an arc}\hfill\end{Bmatrix}.$$

%% page 191

\begin{align*}
T(n)\le T(2^{\lceil\lg n\rceil})
&\le c(3^{\lceil\lg n\rceil}-2^{\lceil\lg n\rceil})\\
&<3c\cdot3^{\lg n}\\
&=3c\,n^{\lg3}.
\end{align*}

%\begin{align*}
%\left\{%
%\begin{gathered}\alpha&=f(z)\\ \beta&=f(z^2)\\ \gamma&=f(z^3)
%\end{gathered}
%\right\}
%\qquad
%\left\{%
%\begin{gathered}
%x&=\alpha^2-\beta\\ y&=2\gamma
%\end{gathered}
%\right\}%
%\end{align*}

%$$\left\{
%\begin{align}
%\alpha&=f(z)\cr \beta&=f(z^2)\cr \gamma&=f(z^3)\\
%%\end{align}
%\right\}
%\qquad
%\left\{
%%\begin{align}
%x&=\alpha^2-\beta\cr y&=2\gamma\\
%\end{align}
%\right\}.$$
%%% page 192

\begin{align*}
\begin{aligned}
(x+y)(x-y)&=x^2-xy+yx-y^2\\
&=x^2-y^2\\
(x+y)^2&=x^2+2xy+y^2.
\end{aligned}
\end{align*}

%% page 192

\begin{align*}
\begin{aligned}
\left( \int\limits_{-\infty}^\infty \mathup{e}^{-x^2}\,\mathup{d}x \right)^2
&=\int_{-\infty}^\infty\int_{-\infty}^\infty \mathup{e}^{-(x^2+y^2)}\,\mathup{d}x\,\mathup{d}y\\
&=\int_0^{2\piup}\int_0^\infty \mathup{e}^{-r^2}\,\mathup{d}r\,\mathup{d}\theta\\
&=\int_0^{2\piup}\biggl(\mathup{e}^{-\frac{r^2}{2}}\biggl|_{r=0}^{r=\infty}\,\biggr)\,\mathup{d}\theta\\
&=\piup.
\end{aligned}
\end{align*}


%% page 197

$$\prod_{k\ge0}\frac{1}{(1-q^kz)}=
\sum_{n\ge0}z^n\bigg/\!\!\prod_{1\le k\le n}(1-q^k).$$

$$\sum_{\substack{\scriptstyle 0< i\le m\\\scriptstyle0<j\le n}}p(i,j) \,\ne
%
% $$\sum_{i=1}^p \sum_{j=1}^q \sum_{k=1}^r a_{ij} b_{jk} c_{ki}$$
%
\sum_{i=1}^p \sum_{j=1}^q \sum_{k=1}^r a_{ij} b_{jk} c_{ki} \,\ne
%
\sum_{\substack{\scriptstyle 1\le i\le p \\ \scriptstyle 1\le j\le q\\
		\scriptstyle 1\le k\le r}} a_{ij} b_{jk} c_{ki}$$

$$\max_{1\le n\le m}\log_2P_n \quad \hbox{and} \quad
\lim_{x\to0}\frac{\sin x}{x}=1$$
Inline math:
$\max_{1\le n\le m}\log_2P_n \quad \hbox{and} \quad
\lim_{x\to0}\frac{\sin x}{x}=1$
$$p_1(n)=\lim_{m\to\infty}\sum_{\nu=0}^\infty\bigl(1-\cos^{2m}(\nu!^n\piup/n)\bigr)$$
Inline math:
$p_1(n)=\lim_{m\to\infty}\sum_{\nu=0}^\infty\bigl(1-\cos^{2m}(\nu!^n\piup/n)\bigr)$%
 	}

	\renewcommand{\showfamily}{{\color{magenta}%
		Sans Serif%
	}}
	{\sffamily\mdseries%
	 \section{Math Test \showfamily}

\subsection{Overview \showfamily}

{\parindent 0pt
Default: $a \alpha \alphaup b \beta G \Gamma \upGamma \epsilon \varepsilon \theta \vartheta P \Pi \Sigma \sigma$; $\sigma_\epsilon, c^\alpha$

mathnormal: $\mathnormal{a \alpha \alphaup b \beta G \Gamma \upGamma \epsilon \varepsilon \theta \vartheta P \Pi \Sigma \sigma}$

mathrm: $\mathrm{a \alpha \alphaup b \beta G \Gamma \upGamma \epsilon \varepsilon \theta \vartheta P \Pi \Sigma \sigma}$

mathup: $\mathup{a \alpha \alphaup b \beta G \Gamma \upGamma \epsilon \varepsilon \theta \vartheta P \Pi \Sigma \sigma}$

mathit: $\mathit{a \alpha \alphaup b \beta G \Gamma \upGamma \epsilon \varepsilon \theta \vartheta P \Pi \Sigma \sigma}$

mathbf: $\mathbf{a \alphaup b \beta G \Gamma \upGamma \epsilon \varepsilon \theta \vartheta P \Pi \Sigma \sigma}$

mathbfit: $\mathbfit{a \alpha b \beta G \Gamma \upGamma \epsilon \varepsilon \theta \vartheta P \Pi \Sigma \sigma}$

mathbfup: $\mathbfup{a \alpha b \beta G \Gamma \upGamma \epsilon \varepsilon \theta \vartheta P \Pi \Sigma \sigma}$

\bigskip

{\bfseries
Default: $a \alpha \alphaup b \beta G \Gamma \upGamma \epsilon \varepsilon \theta \vartheta P \Pi \Sigma \sigma$; $\sigma_\epsilon, c^\alpha$

mathnormal: $\mathnormal{a \alpha \alphaup b \beta G \Gamma \upGamma \epsilon \varepsilon \theta \vartheta P \Pi \Sigma \sigma}$

mathrm: $\mathrm{a \alpha \alphaup b \beta G \Gamma \upGamma \epsilon \varepsilon \theta \vartheta P \Pi \Sigma \sigma}$

mathup: $\mathup{a \alpha \alphaup b \beta G \Gamma \upGamma \epsilon \varepsilon \theta \vartheta P \Pi \Sigma \sigma}$

mathit: $\mathit{a \alpha \alphaup b \beta G \Gamma \upGamma \epsilon \varepsilon \theta \vartheta P \Pi \Sigma \sigma}$

mathbf: $\mathbf{a \alpha \alphaup b \beta G \Gamma \upGamma \epsilon \varepsilon \theta \vartheta P \Pi \Sigma \sigma}$

mathbfit: $\mathbfit{a \alpha \alphaup b \beta G \Gamma \upGamma \epsilon \varepsilon \theta \vartheta P \Pi \Sigma \sigma}$

mathbfup: $\mathbfup{a \alpha \alphaup b \beta G \Gamma \upGamma \epsilon \varepsilon \theta \vartheta P \Pi \Sigma \sigma}$
}

\bigskip

{\sffamily\mdseries
Default: $a \alpha \alphaup b \beta G \Gamma \upGamma \epsilon \varepsilon \theta \vartheta P \Pi \Sigma \sigma$; $\sigma_\epsilon, c^\alpha$

mathnormal: $\mathnormal{a \alpha \alphaup b \beta G \Gamma \upGamma \epsilon \varepsilon \theta \vartheta P \Pi \Sigma \sigma}$

mathrm: $\mathrm{a \alpha \alphaup b \beta G \Gamma \upGamma \epsilon \varepsilon \theta \vartheta P \Pi \Sigma \sigma}$

mathup: $\mathup{a \alpha \alphaup b \beta G \Gamma \upGamma \epsilon \varepsilon \theta \vartheta P \Pi \Sigma \sigma}$

mathit: $\mathit{a \alpha \alphaup b \beta G \Gamma \upGamma \epsilon \varepsilon \theta \vartheta P \Pi \Sigma \sigma}$

mathbf: $\mathbf{a \alpha \alphaup b \beta G \Gamma \upGamma \epsilon \varepsilon \theta \vartheta P \Pi \Sigma \sigma}$

mathbfit: $\mathbfit{a \alpha \alphaup b \beta G \Gamma \upGamma \epsilon \varepsilon \theta \vartheta P \Pi \Sigma \sigma}$

mathbfup: $\mathbfup{a \alpha \alphaup b \beta G \Gamma \upGamma \epsilon \varepsilon \theta \vartheta P \Pi \Sigma \sigma}$
}

\bigskip

{\sffamily\bfseries

Default: $a \alpha \alphaup b \beta G \Gamma \upGamma \epsilon \varepsilon \theta \vartheta P \Pi \Sigma \sigma$; $\sigma_\epsilon, c^\alpha$

mathnormal: $\mathnormal{a \alpha \alphaup b \beta G \Gamma \upGamma \epsilon \varepsilon \theta \vartheta P \Pi \Sigma \sigma}$

mathrm: $\mathrm{a \alpha \alphaup b \beta G \Gamma \upGamma \epsilon \varepsilon \theta \vartheta P \Pi \Sigma \sigma}$

mathup: $\mathup{a \alpha \alphaup b \beta G \Gamma \upGamma \epsilon \varepsilon \theta \vartheta P \Pi \Sigma \sigma}$

mathit: $\mathit{a \alpha \alphaup b \beta G \Gamma \upGamma \epsilon \varepsilon \theta \vartheta P \Pi \Sigma \sigma}$

mathbf: $\mathbf{a \alpha \alphaup b \beta G \Gamma \upGamma \epsilon \varepsilon \theta \vartheta P \Pi \Sigma \sigma}$

mathbfit: $\mathbfit{a \alpha \alphaup b \beta G \Gamma \upGamma \epsilon \varepsilon \theta \vartheta P \Pi \Sigma \sigma}$

mathbfup: $\mathbfup{a \alpha \alphaup b \beta G \Gamma \upGamma \epsilon \varepsilon \theta \vartheta P \Pi \Sigma \sigma}$
}
}


\subsection{Formulas \showfamily}

\noindent%
\checkgreekletters

\noindent%
{\boldmath\checkgreekletters}

\noindent%
{\sffamily\selectfont \checkgreekletters}

\noindent%
{\sffamily\bfseries\selectfont \checkgreekletters}

\noindent%
{\sffamily $\alpha a > 0, \beta b + (3 \times 27), \Gamma G = 7 < 8, \lambda$}

\noindent%
$\alpha a > 0, \beta b + (3 \times 27), \Gamma G = 7 < 8, \lambda$

$\lim_{\nu \to \infty} v(\nu) = \max_{s \in S} \{s \pm 3 \gamma + y - 1\} = 4 \times 7$

$\hat{\beta} = (X'X)^{-1}X'y$

$$\lim_{N \to \infty} \sum_{i=0}^{N} x^i = \min_{x \in \mathbb{R}} S(x)$$

$$\int_{-\infty}^{\infty} x\,f(x)\,\mathup{d}x = \left( \frac{27}{2} \right)$$

Disambiguation: $0$~O~$O$, $1$~l~I~$|$~$l$~$I$~$/$, $i$~$j$, $rn$~$m$, $\theta$~$\Theta$, $\phi$~$\psi$, --~$-$

Latin vs. Greek: $a$~$\alpha$, $d$~$\delta$, $e$~$\epsilon$, $i$~$\iota$, $k$~$\kappa$, $n$~$\eta$, $o$~$\sigma$, $p$~$\rho$, \textit{\ss} $\beta$, $u$~$\upsilon$, $v$~$\nu$, $w$~$\omega$, $x$~$\chi$, $y$~$\gamma$, $A$~$\Delta$~$\Lambda$, $O$~$\Theta$~$\Omega$, $T$~$\Gamma$, $Y$~$\Upsilon$.

\noindent%
{\bfseries%
$\alpha a > 0, \beta b + (3 \times 27), \Gamma G = 7 < 8, \lambda$

$\lim_{\nu \to \infty} v(\nu) = \max_{s \in S} \{s \pm 3 \gamma + y - 1\} = 4 \times 7$

$\hat{\beta} = (X'X)^{-1}X'y$

$$\lim_{N \to \infty} \sum_{i=0}^{N} x^i = \min_{x \in \mathbb{R}} S(x)$$

$$\int_{-\infty}^{\infty} x\,f(x)\,\mathup{d}x = \left( \frac{27}{2} \right)$$

Disambiguation: $0$~O~$O$, $1$~l~I~$|$~$l$~$I$~$/$, $i$~$j$, $rn$~$m$, $\theta$~$\Theta$, $\phi$~$\psi$, --~$-$

Latin vs. Greek: $a$~$\alpha$, $d$~$\delta$, $e$~$\epsilon$, $i$~$\iota$, $k$~$\kappa$, $n$~$\eta$, $o$~$\sigma$, $p$~$\rho$, \textit{\ss} $\beta$, $u$~$\upsilon$, $v$~$\nu$, $w$~$\omega$, $x$~$\chi$, $y$~$\gamma$, $A$~$\Delta$~$\Lambda$, $O$~$\Theta$~$\Omega$, $T$~$\Gamma$, $Y$~$\Upsilon$.
}

\noindent%
{\sffamily%
$\alpha a > 0, \beta b + (3 \times 27), \Gamma G = 7 < 8, \lambda$

$\lim_{\nu \to \infty} v(\nu) = \max_{s \in S} \{s \pm 3 \gamma + y - 1\} = 4 \times 7$

$\hat{\beta} = (X'X)^{-1}X'y$

$$\lim_{N \to \infty} \sum_{i=0}^{N} x^i = \min_{x \in \mathbb{R}} S(x)$$

$$\int_{-\infty}^{\infty} x\,f(x)\,\mathup{d}x = \left( \frac{27}{2} \right)$$

Disambiguation: $0$~O~$O$, $1$~l~I~$|$~$l$~$I$~$/$, $i$~$j$, $rn$~$m$, $\theta$~$\Theta$, $\phi$~$\psi$, --~$-$

Latin vs. Greek: $a$~$\alpha$, $d$~$\delta$, $e$~$\epsilon$, $i$~$\iota$, $k$~$\kappa$, $n$~$\eta$, $o$~$\sigma$, $p$~$\rho$, \textit{\ss} $\beta$, $u$~$\upsilon$, $v$~$\nu$, $w$~$\omega$, $x$~$\chi$, $y$~$\gamma$, $A$~$\Delta$~$\Lambda$, $O$~$\Theta$~$\Omega$, $T$~$\Gamma$, $Y$~$\Upsilon$.
}

\noindent%
{\sffamily\bfseries%
$\alpha a > 0, \beta b + (3 \times 27), \Gamma G = 7 < 8, \lambda$

$\lim_{\nu \to \infty} v(\nu) = \max_{s \in S} \{s \pm 3 \gamma + y - 1\} = 4 \times 7$

$\hat{\beta} = (X'X)^{-1}X'y$

$$\lim_{N \to \infty} \sum_{i=0}^{N} x^i = \min_{x \in \mathbb{R}} S(x)$$

$$\int_{-\infty}^{\infty} x\,f(x)\,\mathup{d}x = \left( \frac{27}{2} \right)$$

Disambiguation: $0$~O~$O$, $1$~l~I~$|$~$l$~$I$~$/$, $i$~$j$, $rn$~$m$, $\theta$~$\Theta$, $\phi$~$\psi$, --~$-$

Latin vs. Greek: $a$~$\alpha$, $d$~$\delta$, $e$~$\epsilon$, $i$~$\iota$, $k$~$\kappa$, $n$~$\eta$, $o$~$\sigma$, $p$~$\rho$, \textit{\ss} $\beta$, $u$~$\upsilon$, $v$~$\nu$, $w$~$\omega$, $x$~$\chi$, $y$~$\gamma$, $A$~$\Delta$~$\Lambda$, $O$~$\Theta$~$\Omega$, $T$~$\Gamma$, $Y$~$\Upsilon$.
}


\subsection{Math Alphabets \showfamily}

%\sffamily\selectfont

Default
\def\test#1{#1,}
\begin{eqnarray*}
  && {\testnums}\\
  && {\testupper}\\
  && {\testlower}\\
  && {\testupgreek}\\
  && {\testlowgreek}
\end{eqnarray*}%

Math Normal (\texttt{\string\mathnormal})
\def\test#1{\mathnormal{#1},}
\begin{eqnarray*}
  && {\testnums}\\
  && {\testupper}\\
  && {\testlower}\\
  && {\testupgreek}\\
  && {\testlowgreek}
\end{eqnarray*}%

Math Italic (\texttt{\string\mathit})
\def\test#1{\mathit{#1},}
\begin{eqnarray*}
  && {\testnums}\\
  && {\testupper}\\
  && {\testlower}\\
  && {\testupgreek}\\
  && {\testlowgreek}
\end{eqnarray*}%

Math Roman (\texttt{\string\mathrm})
\def\test#1{\mathrm{#1},}
\begin{eqnarray*}
  && {\testnums}\\
  && {\testupper}\\
  && {\testlower}\\
  && {\testupgreek}\\
  && {\testlowgreek}
\end{eqnarray*}%

%Math Italic Bold (\texttt{\string\mathbm})
%\def\test#1{\mathbm{#1},}
%\begin{eqnarray*}
%  && {\testnums}\\
%  && {\testupper}\\
%  && {\testlower}\\
%  && {\testupgreek}\\
%  && {\testlowgreek}
%\end{eqnarray*}%

Math Bold (\texttt{\string\mathbf})
\def\test#1{\mathbf{#1},}
\begin{eqnarray*}
  && {\testnums}\\
  && {\testupper}\\
  && {\testlower}\\
  && {\testupgreek}\\
  && {\testlowgreek}
\end{eqnarray*}%

Caligraphic (\texttt{\string\mathcal})
\def\test#1{\mathcal{#1},}
\begin{eqnarray*}
  && {\testupper}
\end{eqnarray*}%

Script (\texttt{\string\mathscr})
\def\test#1{\mathscr{#1},}
\begin{eqnarray*}
  && {\testupper}
\end{eqnarray*}%

Fraktur (\texttt{\string\mathfrak})
\def\test#1{\mathfrak{#1},}
\begin{eqnarray*}
  && {\testupper}\\
  && {\testlower}
\end{eqnarray*}%

Blackboard Bold (\texttt{\string\mathbb})
\def\test#1{\mathbb{#1},}
\begin{eqnarray*}
  && {\testupper}
\end{eqnarray*}%

\subsection{Character Sidebearings \showfamily}

Default
\def\test#1{|#1|+{}}
\begin{eqnarray*}
  && {\testupperi}\\
  && {\testupperii}\\
  && {\testloweri}\\
  && {\testlowerii}\\
  && {\testupgreeki}\\
  && {\testupgreekii}\\
  && {\testlowgreeki}\\
  && {\testlowgreekii}\\
  && {\testlowgreekiii}
\end{eqnarray*}%

Math Roman (\texttt{\string\mathrm})
\def\test#1{|\mathrm{#1}|+{}}%
\begin{eqnarray*}
  && {\testupperi}\\
  && {\testupperii}\\
  && {\testloweri}\\
  && {\testlowerii}\\
  && {\testupgreeki}\\
  && {\testupgreekii}
\end{eqnarray*}%

%Math Italic Bold (\texttt{\string\mathbm})
%\def\test#1{|\mathbm{#1}|+{}}%
%\begin{eqnarray*}
%  && {\testupperi}\\
%  && {\testupperii}\\
%  && {\testloweri}\\
%  && {\testlowerii}\\
%  && {\testupgreeki}\\
%  && {\testupgreekii}\\
%  && {\testlowgreeki}\\
%  && {\testlowgreekii}\\
%  && {\testlowgreekiii}
%\end{eqnarray*}%

Math Bold (\texttt{\string\mathbf})
\def\test#1{|\mathbf{#1}|+{}}%
\begin{eqnarray*}
  && {\testupperi}\\
  && {\testupperii}\\
  && {\testloweri}\\
  && {\testlowerii}\\
  && {\testupgreeki}\\
  && {\testupgreekii}
\end{eqnarray*}%

Math Calligraphic (\texttt{\string\mathcal})
\def\test#1{|\mathcal{#1}|+{}}%
\begin{eqnarray*}
  && {\testupperi}\\
  && {\testupperii}
\end{eqnarray*}%


\subsection{Superscript Positioning \showfamily}

Default
\def\test#1{#1^{2}+{}}%
\begin{eqnarray*}
  && {\testupperi}\\
  && {\testupperii}\\
  && {\testloweri}\\
  && {\testlowerii}\\
  && {\testupgreeki}\\
  && {\testupgreekii}\\
  && {\testlowgreeki}\\
  && {\testlowgreekii}\\
  && {\testlowgreekiii}
\end{eqnarray*}%

Math Roman (\texttt{\string\mathrm})
\def\test#1{\mathrm{#1}^{2}+{}}%
\begin{eqnarray*}
  && {\testupperi}\\
  && {\testupperii}\\
  && {\testloweri}\\
  && {\testlowerii}\\
  && {\testupgreeki}\\
  && {\testupgreekii}
\end{eqnarray*}%

%Math Italic Bold (\texttt{\string\mathbm})
%\def\test#1{\mathbm{#1}^{2}+{}}%
%\begin{eqnarray*}
%  && {\testupperi}\\
%  && {\testupperii}\\
%  && {\testloweri}\\
%  && {\testlowerii}\\
%  && {\testupgreeki}\\
%  && {\testupgreekii}\\
%  && {\testlowgreeki}\\
%  && {\testlowgreekii}\\
%  && {\testlowgreekiii}
%\end{eqnarray*}%

Math Bold (\texttt{\string\mathbf})
\def\test#1{\mathbf{#1}^{2}+{}}%
\begin{eqnarray*}
  && {\testupperi}\\
  && {\testupperii}\\
  && {\testloweri}\\
  && {\testlowerii}\\
  && {\testupgreeki}\\
  && {\testupgreekii}
\end{eqnarray*}

Math Calligraphic (\texttt{\string\mathcal})
\def\test#1{\mathcal{#1}^{2}+{}}%
\begin{eqnarray*}
  && {\testupperi}\\
  && {\testupperii}
\end{eqnarray*}%


\subsection{Subscript Positioning \showfamily}

Default
\def\test#1{\mathnormal{#1}_{i}+{}}%
\begin{eqnarray*}
  && {\testupperi}\\
  && {\testupperii}\\
  && {\testloweri}\\
  && {\testlowerii}\\
  && {\testupgreeki}\\
  && {\testupgreekii}\\
  && {\testlowgreeki}\\
  && {\testlowgreekii}\\
  && {\testlowgreekiii}
\end{eqnarray*}%

Math Roman (\texttt{\string\mathrm})
\def\test#1{\mathrm{#1}_{i}+{}}%
\begin{eqnarray*}
  && {\testupperi}\\
  && {\testupperii}\\
  && {\testloweri}\\
  && {\testlowerii}\\
  && {\testupgreeki}\\
  && {\testupgreekii}
\end{eqnarray*}%

%Math Bold Italic (\texttt{\string\mathbm})
%\def\test#1{\mathbm{#1}_{i}+{}}%
%\begin{eqnarray*}
%  && {\testupperi}\\
%  && {\testupperii}\\
%  && {\testloweri}\\
%  && {\testlowerii}\\
%  && {\testupgreeki}\\
%  && {\testupgreekii}\\
%  && {\testlowgreeki}\\
%  && {\testlowgreekii}\\
%  && {\testlowgreekiii}
%\end{eqnarray*}

Math Bold (\texttt{\string\mathbf})
\def\test#1{\mathbf{#1}_{i}+{}}%
\begin{eqnarray*}
  && {\testupperi}\\
  && {\testupperii}\\
  && {\testloweri}\\
  && {\testlowerii}\\
  && {\testupgreeki}\\
  && {\testupgreekii}
\end{eqnarray*}%

Math Calligraphic (\texttt{\string\mathcal})
\def\test#1{\mathcal{#1}_{i}+{}}%
\begin{eqnarray*}
  && {\testupperi}\\
  && {\testupperii}
\end{eqnarray*}%


\subsection{Accent Positioning \showfamily}

Default
\def\test#1{\hat{#1}+{}}%
\begin{eqnarray*}
  && {\testnums}\\
  && {\testupperi}\\
  && {\testupperii}\\
  && {\testloweri}\\
  && {\testlowerii}\\
  && {\testupgreeki}\\
  && {\testupgreekii}\\
  && {\testlowgreeki}\\
  && {\testlowgreekii}\\
  && {\testlowgreekiii}
\end{eqnarray*}%

Math Italic (\texttt{\string\mathit})
\def\test#1{\hat{\mathit{#1}}+{}}%
\begin{eqnarray*}
  && {\testnums}\\
  && {\testupperi}\\
  && {\testupperii}\\
  && {\testloweri} \test\ell \test\wp \test\imath \test\jmath \tilde{i} \\
  && {\testlowerii}\\
  && {\testupgreeki}\\
  && {\testupgreekii}\\
  && {\testlowgreeki}\\
  && {\testlowgreekii}\\
  && {\testlowgreekiii}
\end{eqnarray*}%

Math Roman (\texttt{\string\mathrm})
\def\test#1{\hat{\mathrm{#1}}+{}}%
\begin{eqnarray*}
  && {\testnums}\\
  && {\testupperi}\\
  && {\testupperii}\\
  && {\testloweri}\\
  && {\testlowerii}\\
  && {\testupgreeki}\\
  && {\testupgreekii}
\end{eqnarray*}%

%Math Italic Bold (\texttt{\string\mathbm})
%\def\test#1{\hat{\mathbm{#1}}+{}}%
%\begin{eqnarray*}
%  && {\testnums}\\
%  && {\testupperi}\\
%  && {\testupperii}\\
%  && {\testloweri}\\
%  && {\testlowerii}\\
%  && {\testupgreeki}\\
%  && {\testupgreekii}\\
%  && {\testlowgreeki}\\
%  && {\testlowgreekii}\\
%  && {\testlowgreekiii}
%\end{eqnarray*}%

Math Bold (\texttt{\string\mathbf})
\def\test#1{\hat{\mathbf{#1}}+{}}%
\begin{eqnarray*}
  && {\testnums}\\
  && {\testupperi}\\
  && {\testupperii}\\
  && {\testloweri}\\
  && {\testlowerii}\\
  && {\testupgreeki}\\
  && {\testupgreekii}
\end{eqnarray*}

Math Calligraphic (\texttt{\string\mathcal})
\def\test#1{\hat{\mathcal{#1}}+{}}%
\begin{eqnarray*}
  && {\testupperi}\\
  && {\testupperii}
\end{eqnarray*}%


\subsection{Differentials \showfamily}

\begin{eqnarray*}
\gdef\test#1{\dit #1+{}}%
  && {\testupperi}\\
  && {\testupperii}\\
  && {\testloweri}\\
  && {\testlowerii}\\
  && {\testupgreeki}\\
  && {\testupgreekii}\\
  && {\testlowgreeki}\\
  && {\testlowgreekii}\\
  && {\testlowgreekiii}\\
\gdef\test#1{\dit \mathrm{#1}+{}}%
  && {\testupgreeki}\\
  && {\testupgreekii}
\end{eqnarray*}%

\begin{eqnarray*}
\gdef\test#1{\dup #1+{}}%
  && {\testupperi}\\
  && {\testupperii}\\
  && {\testloweri}\\
  && {\testlowerii}\\
  && {\testupgreeki}\\
  && {\testupgreekii}\\
  && {\testlowgreeki}\\
  && {\testlowgreekii}\\
  && {\testlowgreekiii}\\
\gdef\test#1{\dup \mathrm{#1}+{}}%
  && {\testupgreeki}\\
  && {\testupgreekii}
\end{eqnarray*}%

\begin{eqnarray*}
\gdef\test#1{\partial #1+{}}%
  && {\testupperi}\\
  && {\testupperii}\\
  && {\testloweri}\\
  && {\testlowerii}\\
  && {\testupgreeki}\\
  && {\testupgreekii}\\
  && {\testlowgreeki}\\
  && {\testlowgreekii}\\
  && {\testlowgreekiii}\\
\gdef\test#1{\partial \mathrm{#1}+{}}%
  && {\testupgreeki}\\
  && {\testupgreekii}
\end{eqnarray*}%


\subsection{Slash Kerning \showfamily}

\def\test#1{1/#1+{}}
\begin{eqnarray*}
  && {\testupperi}\\
  && {\testupperii}\\
  && {\testloweri}\\
  && {\testlowerii}\\
  && {\testupgreeki}\\
  && {\testupgreekii}\\
  && {\testlowgreeki}\\
  && {\testlowgreekii}\\
  && {\testlowgreekiii}
\end{eqnarray*}

\def\test#1{#1/2+{}}
\begin{eqnarray*}
  && {\testupperi}\\
  && {\testupperii}\\
  && {\testloweri}\\
  && {\testlowerii}\\
  && {\testupgreeki}\\
  && {\testupgreekii}\\
  && {\testlowgreeki}\\
  && {\testlowgreekii}\\
  && {\testlowgreekiii}
\end{eqnarray*}


\subsection{(Big) Operators \showfamily}

\def\testop#1{#1_{i=1}^{n} x^{n} \quad}
$
	\testop\sum
	\testop\prod
	\testop\coprod
	\testop\int
	\testop\oint
$

\noindent%
$
	\testop\bigotimes
	\testop\bigoplus
	\testop\bigodot
	\testop\bigwedge
	\testop\bigvee
	\testop\biguplus
	\testop\bigcup
	\testop\bigcap
	\testop\bigsqcup
	% \testop\bigsqcap
$

\begin{displaymath}
  \testop\sum
  \testop\prod
  \testop\coprod
  \testop\int
  \testop\oint
\end{displaymath}
\begin{displaymath}
  \testop\bigotimes
  \testop\bigoplus
  \testop\bigodot
  \testop\bigwedge
  \testop\bigvee
  \testop\biguplus
  \testop\bigcup
  \testop\bigcap
  \testop\bigsqcup
% \testop\bigsqcap
\end{displaymath}


\subsection{Radicals \showfamily}

\begin{displaymath}
  \sqrt{x+y} \qquad \sqrt{x^{2}+y^{2}} \qquad
  \sqrt{x_{i}^{2}+y_{j}^{2}} \qquad
  \sqrt{\left(\frac{\cos x}{2}\right)} \qquad
  \sqrt{\left(\frac{\sin x}{2}\right)}
\end{displaymath}

\begingroup
\delimitershortfall-1pt
\begin{displaymath}
  \sqrt{\sqrt{\sqrt{\sqrt{\sqrt{\sqrt{\sqrt{x+y}}}}}}}
\end{displaymath}
\endgroup % \delimitershortfall


\subsection{Over- and Underbraces \showfamily}

\begin{displaymath}
  \overbrace{x} \quad
  \overbrace{x+y} \quad
  \overbrace{x^{2}+y^{2}} \quad
  \overbrace{x_{i}^{2}+y_{j}^{2}} \quad
  \underbrace{x} \quad
  \underbrace{x+y} \quad
  \underbrace{x_{i}+y_{j}} \quad
  \underbrace{x_{i}^{2}+y_{j}^{2}} \quad
\end{displaymath}


\subsection{Normal and Wide Accents \showfamily}

\begin{displaymath}
  \dot{x} \quad
  \ddot{x} \quad
  \vec{x} \quad
  \bar{x} \quad
  \overline{x} \quad
  \overline{xx} \quad
  \tilde{x} \quad
  \widetilde{x} \quad
  \widetilde{xx} \quad
  \widetilde{xxx} \quad
  \hat{x} \quad
  \widehat{x} \quad
  \widehat{xx} \quad
  \widehat{xxx} \quad
\end{displaymath}

\begin{displaymath}
  \hat{x} \quad
  \check{x} \quad
  \tilde{x} \quad
  \acute{x} \quad
  \grave{x} \quad
  \dot{x} \quad
  \ddot{x} \quad
  \breve{x} \quad
  \bar{x} \quad
  \vec{x} \quad
\end{displaymath}


\subsection{Long Arrows \showfamily}

\begin{displaymath}
  \leftarrow \mathrel{-} \rightarrow \quad
  \leftrightarrow \quad
  \longleftarrow  \quad
  \longrightarrow \quad
  \longleftrightarrow \quad
  \Leftarrow = \Rightarrow \quad
  \Leftrightarrow \quad
  \Longleftarrow  \quad
  \Longrightarrow \quad
  \Longleftrightarrow \quad
\end{displaymath}


\subsection{Left and Right Delimiters \showfamily}

\def\testdelim#1#2{ - #1 f #2 - }
\begin{displaymath}
  \testdelim()
  \testdelim[]
  \testdelim\lfloor\rfloor
  \testdelim\lceil\rceil
  \testdelim\langle\rangle
  \testdelim\{\}
\end{displaymath}

Using {\tt\string\left} and {\tt\string\right}.
\def\testdelim#1#2{ - \left#1 f \right#2 - }
\begin{displaymath}
  \testdelim()
  \testdelim[]
  \testdelim\lfloor\rfloor
  \testdelim\lceil\rceil
  \testdelim\langle\rangle
  \testdelim\{\}
% \testdelim\lgroup\rgroup
% \testdelim\lmoustache\rmoustache
\end{displaymath}
\begin{displaymath}
  \testdelim)(
  \testdelim][
  \testdelim//
  \testdelim\backslash\backslash
  \testdelim/\backslash
  \testdelim\backslash/
\end{displaymath}


\subsection{Big-g-g Delimiters \showfamily}

\def\testdelim#1#2{%
  - \left#1\left#1\left#1\left#1\left#1\left#1\left#1\left#1 -
  \right#2\right#2\right#2\right#2\right#2\right#2\right#2\right#2 -}

\begingroup
\delimitershortfall-1pt
\begin{displaymath}
  \testdelim\lfloor\rfloor
  \qquad
  \testdelim()
\end{displaymath}
\begin{displaymath}
  \testdelim\lceil\rceil
  \qquad
  \testdelim\{\}
\end{displaymath}
\begin{displaymath}
  \testdelim[]
  \qquad
  \testdelim\lgroup\rgroup
\end{displaymath}
\begin{displaymath}
  \testdelim\langle\rangle
  \qquad
  \testdelim\lmoustache\rmoustache
\end{displaymath}
\begin{displaymath}
  \testdelim\uparrow\downarrow \quad
  \testdelim\Uparrow\Downarrow \quad
\end{displaymath}
\endgroup % \delimitershortfall

\def\X#1{$x #1 y$ &\tt\string#1}
\def\Y#1{$\big#1$ &\tt\string#1}
\def\Z#1{$x #1 y$}
\def\W#1#2{$#1{#2}$ &\tt\string#1\string{#2\string}}


\subsection{Binary Operators \showfamily}

\begin{tabular}{*8l}
\X\pm           &\X\cap         &\X\diamond             &\X\oplus     \\
\X\mp           &\X\cup         &\X\bigtriangleup       &\X\ominus    \\
\X\times        &\X\uplus       &\X\bigtriangledown     &\X\otimes    \\
\X\div          &\X\sqcap       &\X\triangleleft        &\X\oslash    \\
\X\ast          &\X\sqcup       &\X\triangleright       &\X\odot      \\
\X\star         &\X\vee         &\X\lhd                 &\X\bigcirc   \\
\X\circ         &\X\wedge       &\X\rhd                 &\X\dagger    \\
\X\bullet       &\X\setminus    &\X\unlhd               &\X\ddagger   \\
\X\cdot         &\X\wr          &\X\unrhd               &\X\S         \\
\X+             &\X-            &\X\amalg               &\X\P
\end{tabular}


\subsection{Relations \showfamily}

\begin{tabular}{*8l}
\X\leq          &\X\geq         &\X\equiv       &\X\models      \\
\X\prec         &\X\succ        &\X\sim         &\X\perp        \\
\X\preceq       &\X\succeq      &\X\simeq       &\X\mid         \\
\X\ll           &\X\gg          &\X\asymp       &\X\parallel    \\
\X\subset       &\X\supset      &\X\approx      &\X\bowtie      \\
\X\subseteq     &\X\supseteq    &\X\cong        &\X\Join        \\
\X\sqsubset     &\X\sqsupset    &\X\neq         &\X\smile       \\
\X\sqsubseteq   &\X\sqsupseteq  &\X\doteq       &\X\frown       \\
\X\in           &\X\ni          &\X\propto      &\X=            \\
\X\vdash        &\X\dashv       &\X<            &\X>            \\
\X:
\end{tabular}


\subsection{Punctuation \showfamily}

\begin{tabular}{*{5}{lp{3.2em}}}
\X,     &\X;    &\X\colon       &\X\ldotp       &\X\cdotp
\end{tabular}


\subsection{Arrows \showfamily}

\begin{tabular}{*6l}
\X\leftarrow            &\X\longleftarrow       &\X\uparrow     \\
\X\Leftarrow            &\X\Longleftarrow       &\X\Uparrow     \\
\X\rightarrow           &\X\longrightarrow      &\X\downarrow   \\
\X\Rightarrow           &\X\Longrightarrow      &\X\Downarrow   \\
\X\leftrightarrow       &\X\longleftrightarrow  &\X\updownarrow \\
\X\Leftrightarrow       &\X\Longleftrightarrow  &\X\Updownarrow \\
\X\mapsto               &\X\longmapsto          &\X\nearrow     \\
\X\hookleftarrow        &\X\hookrightarrow      &\X\searrow     \\
\X\leftharpoonup        &\X\rightharpoonup      &\X\swarrow     \\
\X\leftharpoondown      &\X\rightharpoondown    &\X\nwarrow     \\
\X\rightleftharpoons    &\X\leadsto
\end{tabular}


\subsection{Miscellaneous Symbols \showfamily}

\begin{tabular}{*8l}
\X\ldots        &\X\cdots       &\X\vdots       &\X\ddots       \\
\X\aleph        &\X\prime       &\X\forall      &\X\infty       \\
\X\hbar         &\X\emptyset    &\X\exists      &\X\Box         \\
\X\imath        &\X\nabla       &\X\neg         &\X\Diamond     \\
\X\jmath        &\X\surd        &\X\flat        &\X\triangle    \\
\X\ell          &\X\top         &\X\natural     &\X\clubsuit    \\
\X\wp           &\X\bot         &\X\sharp       &\X\diamondsuit \\
\X\Re           &\X\|           &\X\backslash   &\X\heartsuit   \\
\X\Im           &\X\angle       &\X\partial     &\X\spadesuit   \\
\X\mho          &\X.            &\X|            &\X!
\end{tabular}


\subsection{Variable-Sized Operators \showfamily}

\begin{tabular}{*6l}
\X\sum          &\X\bigcap      &\X\bigodot     \\
\X\prod         &\X\bigcup      &\X\bigotimes   \\
\X\coprod       &\X\bigsqcup    &\X\bigoplus    \\
\X\int          &\X\bigvee      &\X\biguplus    \\
\X\oint         &\X\bigwedge
\end{tabular}


\subsection{Log-Like Operators \showfamily}

\begin{tabular}{*8l}
\Z\arccos &\Z\cos  &\Z\csc &\Z\exp &
           \Z\ker    &\Z\limsup &\Z\min &\Z\sinh \\
\Z\arcsin &\Z\cosh &\Z\deg &\Z\gcd &
           \Z\lg     &\Z\ln     &\Z\Pr  &\Z\sup  \\
\Z\arctan &\Z\cot  &\Z\det &\Z\hom &
           \Z\lim    &\Z\log    &\Z\sec &\Z\tan  \\
\Z\arg    &\Z\coth &\Z\dim &\Z\inf &
           \Z\liminf &\Z\max    &\Z\sin &\Z\tanh
\end{tabular}


\subsection{Delimiters \showfamily}

\begin{tabular}{*8l}
\X(             &\X)            &\X\uparrow     &\X\Uparrow     \\
\X[             &\X]            &\X\downarrow   &\X\Downarrow   \\
\X\{            &\X\}           &\X\updownarrow &\X\Updownarrow \\
\X\lfloor       &\X\rfloor      &\X\lceil       &\X\rceil       \\
\X\langle       &\X\rangle      &\X/            &\X\backslash   \\
\X|             &\X\|
\end{tabular}


\subsection{Large Delimiters \showfamily}

\begin{tabular}{*8l}
\Y\rmoustache&  \Y\lmoustache&  \Y\rgroup&      \Y\lgroup\\[5pt]
\Y\arrowvert&   \Y\Arrowvert&   \Y\bracevert
\end{tabular}


\subsection{Math Mode Accents \showfamily}

\begin{tabular}{*{10}l}
\W\hat{a}     &\W\acute{a}  &\W\bar{a}    &\W\dot{a}    &\W\breve{a}\\
\W\check{a}   &\W\grave{a}  &\W\vec{a}    &\W\ddot{a}   &\W\tilde{a}\\
\end{tabular}


\subsection{Miscellaneous Constructions \showfamily}

\begin{tabular}{*4l}
\W\widetilde{abc}       &\W\widehat{abc}                        \\
\W\overleftarrow{abc}   &\W\overrightarrow{abc}                 \\
\W\overline{abc}        &\W\underline{abc}                      \\
\W\overbrace{abc}       &\W\underbrace{abc}                     \\[5pt]
\W\sqrt{abc}            &$\sqrt[n]{abc}$&\verb|\sqrt[n]{abc}|   \\
$f'$&\verb|f'|          &$\frac{abc}{xyz}$&\verb|\frac{abc}{xyz}|
\end{tabular}


\subsection{AMS Delimiters \showfamily}

\begin{tabular}{*8l}
\X\ulcorner&\X\urcorner&\X\llcorner&\X\lrcorner
\end{tabular}


\subsection{AMS Arrows \showfamily}

\begin{tabular}{*8l}
\X\dashrightarrow       &\X\dashleftarrow
        \\ \X\leftleftarrows      &\X\leftrightarrows     \\
\X\Lleftarrow           &\X\twoheadleftarrow
        \\ \X\leftarrowtail       &\X\looparrowleft       \\
\X\leftrightharpoons    &\X\curvearrowleft
        \\ \X\circlearrowleft     &\X\Lsh                 \\
\X\upuparrows           &\X\upharpoonleft
        \\ \X\downharpoonleft     &\X\multimap            \\
\X\leftrightsquigarrow  &\X\rightrightarrows
        \\ \X\rightleftarrows     &\X\rightrightarrows    \\
\X\rightleftarrows      &\X\twoheadrightarrow
        \\ \X\rightarrowtail      &\X\looparrowright      \\
\X\rightleftharpoons    &\X\curvearrowright
        \\ \X\circlearrowright    &\X\Rsh                 \\
\X\downdownarrows       &\X\upharpoonright
        \\ \X\downharpoonright    &\X\rightsquigarrow
\end{tabular}


\subsection{AMS Negated Arrows \showfamily}

\begin{tabular}{*8l}
\X\nleftarrow   &\X\nrightarrow \\ \X\nLeftarrow  &\X\nRightarrow \\
\X\nleftrightarrow&\X\nLeftrightarrow
\end{tabular}


\subsection{AMS Greek \showfamily}

\begin{tabular}{*4l}
\X\digamma      &\X\varkappa
\end{tabular}


\subsection{AMS Hebrew \showfamily}

\begin{tabular}{*6l}
\X\beth &\X\daleth      &\X\gimel
\end{tabular}


\subsection{AMS Miscellaneous \showfamily}

\begin{tabular}{*8l}
\X\hbar         &\X\hslash      \\ \X\vartriangle &\X\triangledown      \\
\X\square       &\X\lozenge     \\ \X\circledS    &\X\angle             \\
\X\measuredangle&\X\nexists     \\ \X\mho         &\X\Finv$^u$          \\
\X\Game$^u$     &\X\Bbbk$^u$    \\ \X\backprime   &\X\varnothing        \\
\X\blacktriangle&\X\blacktriangledown \\ \X\blacksquare&\X\blacklozenge  \\
\X\bigstar      &\X\sphericalangle     \\ \X\complement  &\X\eth       \\
\X\diagup$^u$   &\X\diagdown$^u$
\end{tabular}

$^u$ Not defined in {\tt amssymb.sty}, define using the
\verb|\newsymbol|  command.


\subsection{AMS Binary Operators \showfamily}

\begin{tabular}{*8l}
\X\dotplus      &\X\smallsetminus \\ \X\Cap        &\X\Cup               \\
\X\barwedge     &\X\veebar      \\ \X\doublebarwedge&\X\boxminus        \\
\X\boxtimes     &\X\boxdot      \\ \X\boxplus     &\X\divideontimes     \\
\X\ltimes       &\X\rtimes      \\ \X\leftthreetimes&\X\rightthreetimes \\
\X\curlywedge   &\X\curlyvee    \\ \X\circleddash &\X\circledast        \\
\X\circledcirc  &\X\centerdot   \\ \X\intercal
\end{tabular}


\subsection{AMS Relations \showfamily}

\begin{tabular}{*2l}
\X\leqslant    \\\X\lesssim    \\
\X\approxeq    \\\X\lll        \\
\X\lesseqgtr   \\\X\doteqdot   \\
\X\fallingdotseq\\\X\backsimeq  \\
\X\Subset      \\\X\preccurlyeq\\
\X\precsim     \\\X\vartriangleleft\\
\X\vDash      \\\X\smallsmile \\
\X\bumpeq      \\\X\geqq       \\
\X\eqslantgtr  \\\X\gtrapprox  \\
\X\ggg         \\\X\gtreqless  \\
\X\eqcirc      \\\X\triangleq  \\
\X\thickapprox \\\X\Supset     \\
\X\succcurlyeq \\\X\succsim    \\
\X\vartriangleright\\\X\Vdash      \\
\X\shortparallel\\\X\pitchfork  \\
\X\blacktriangleleft \\\X\backepsilon\\
\X\because
\end{tabular}


\subsection{AMS Negated Relations \showfamily}

\begin{tabular}{*8l}
\X\nless        &\X\nleq        \\ \X\nleqslant   &\X\nleqq       \\
\X\lneq         &\X\lneqq       \\ \X\lvertneqq   &\X\lnsim       \\
\X\lnapprox     &\X\nprec       \\ \X\npreceq     &\X\precnsim    \\
\X\precnapprox  &\X\nsim        \\ \X\nshortmid   &\X\nmid        \\
\X\nvdash       &\X\nvDash      \\ \X\ntriangleleft&\X\ntrianglelefteq\\
\X\nsubseteq    &\X\subsetneq   \\ \X\varsubsetneq&\X\subsetneqq  \\
\X\varsubsetneqq&\X\ngtr        \\ \X\ngeq        &\X\ngeqslant   \\
\X\ngeqq        &\X\gneq        \\ \X\gneqq       &\X\gvertneqq   \\
\X\gnsim        &\X\gnapprox    \\ \X\nsucc       &\X\nsucceq     \\
\X\nsucceqq     &\X\succnsim    \\ \X\succnapprox &\X\ncong       \\
\X\nshortparallel&\X\nparallel  \\ \X\nvDash      &\X\nVDash      \\
\X\ntriangleright&\X\ntrianglerighteq \\ \X\nsupseteq&\X\nsupseteqq\\
\X\supsetneq    &\X\varsupsetneq \\ \X\supsetneqq  &\X\varsupsetneqq
\end{tabular}%
	 \subsection{Math ``Torture'' Test \showfamily}
	 Most of the following examples are taken from \textit{The \TeX book} \citep[][see \url{https://ctan.org/pkg/texbook}]{Knuth1984} and were adapted for \LaTeX\ from Karl Berry's torture test for plain \TeX\ math fonts.

\noindent $x + y - z$, \quad $x + y * z$, \quad $z * y / z$, \quad 
$(x+y)(x-y) = x^2 - y^2$, 

\noindent $x \times y \cdot z = [x\, y\, z]$, \quad $x\circ y \bullet z$, \quad
$x\cup y \cap z$, \quad $x\sqcup y \sqcap z$, \quad

\noindent $x \vee y \wedge z$, \quad $x\pm y\mp z$, \quad
$x=y/z$, \;\; $x:=y$, \;\; $x\le y \ne z$, \;\; $x \sim y \simeq z$
$x \equiv y \nequiv z$, \;\; $x\subset y \subseteq z$

\noindent $\sin2\theta=2\sin\theta\cos\theta$, \quad
$\hbox{O}(n\log n\log n)$, \quad
$\Pr(X>x)=\exp(-x/\mu)$,

\noindent $\bigl(x\in A(n)\bigm|x\in B(n)\bigr)$, \quad
$\bigcup_n X_n\bigm\|\bigcap_n Y_n$

% page 178

\noindent In-text matrices $\binom{1\,1}{0\,1}$ and $\bigl(\genfrac{}{}{0pt}{}{a}{1}\genfrac{}{}{0pt}{}{b}{m}\genfrac{}{}{0pt}{}{c}{n}\bigr)$.

% page 142

$$a_0+\frac1{\displaystyle a_1 +
	{\strut \frac1{\displaystyle a_2 +
			{\strut \frac1{\displaystyle a_3 +
					{\strut \frac1{\displaystyle a_4}}}}}}}$$

% page 143

$$\binom{p}{2}x^2y^{p-2} - \frac1{1 - x}\frac{1}{1 - x^2}
=
\frac{a+1}{b}\bigg/\frac{c+1}{d}.$$

%% page 145

$$\sqrt{1+\sqrt{1+\sqrt{1+\sqrt{1+\sqrt{1+x}}}}}$$

$$\sqrt[n]{1+\sqrt[k]{1+\sqrt[5]{1+\sqrt[4]{1+\sqrt[3]{1+x}}}}}$$

%% page 147

$$\left(\frac{\partial^2}{\partial x^2} + \frac{\partial^2}{\partial y^2}\right)
\bigl|\varphi(x+\mathup{i}y)\bigr|^2=0$$

%% page 149

% $$\pi(n)=\sum_{m=2}^n\left\lfloor\biggl(\sum_{k=1}^{m-1}\bigl
% \lfloor(m/k)\big/\lceil m/k\rceil\bigr\rfloor\biggr)^{-1}\right\rfloor.$$

$$\pi(n)=\sum_{m=2}^n\left\lfloor\Biggl(\sum_{k=1}^{m-1}\bigl
\lfloor(m/k)\big/\lceil m/k\rceil\bigr\rfloor\Biggr)^{-1}\right\rfloor.$$

% page 168

$$\int_0^\infty \frac{t - \mathup{i} b}{t^2 + b^2}e^{\mathup{i}at}\,\mathup{d}t=e^{ab}E_1(ab), \quad
a,b > 0.$$

% page 176

$$\mathbf{A} \coloneqq \begin{pmatrix}x-\lambda&1&0\\
0&x-\lambda&1\\
0&0&x-\lambda\end{pmatrix}.$$

$$\left\lgroup\begin{matrix}a&b&c\\ d&e&f\\\end{matrix}\right\rgroup
\left\lgroup\begin{matrix}u&x\cr v&y\cr w&z\end{matrix}\right\rgroup$$

% page 177

$$\mathbf{A} = \begin{pmatrix}a_{11}&a_{12}&\ldots&a_{1n}\\
a_{21}&a_{22}&\ldots&a_{2n}\\
\vdots&\vdots&\ddots&\vdots\\
a_{m1}&a_{m2}&\ldots&a_{mn}\end{pmatrix}$$

$$\mathbf{M}=\bordermatrix{&C&I&C'\cr
	C&1&0&0\cr I&b&1-b&0\cr C'&0&a&1-a}$$

%% page 186

$$\sum_{n=0}^\infty a_nz^n\quad\hbox{converges if}\quad
|z|<\Bigl(\limsup_{n\to\infty}\root n\of{|a_n|}\,\Bigr)^{-1}.$$

$$\frac{f(x+\mathup{\Delta} x)-f(x)}{\mathup{\Delta} x}\to f'(x)
\qquad \hbox{as $\mathup{\Delta} x\to0$.}$$

$$\|u_i\|=1,\qquad u_i\cdot u_j=0\quad\hbox{if $i\ne j$.}$$

%% page 191

$$\hbox{The confluent image of}\quad
\begin{Bmatrix}\hbox{an arc}\hfill\\\hbox{a circle}\hfill\\
\hbox{a fan}\hfill\\\end{Bmatrix}
\quad\hbox{is}\quad
\begin{Bmatrix}\hbox{an arc}\hfill\\
\hbox{an arc or a circle}\hfill\\
\hbox{a fan or an arc}\hfill\end{Bmatrix}.$$

%% page 191

\begin{align*}
T(n)\le T(2^{\lceil\lg n\rceil})
&\le c(3^{\lceil\lg n\rceil}-2^{\lceil\lg n\rceil})\\
&<3c\cdot3^{\lg n}\\
&=3c\,n^{\lg3}.
\end{align*}

%\begin{align*}
%\left\{%
%\begin{gathered}\alpha&=f(z)\\ \beta&=f(z^2)\\ \gamma&=f(z^3)
%\end{gathered}
%\right\}
%\qquad
%\left\{%
%\begin{gathered}
%x&=\alpha^2-\beta\\ y&=2\gamma
%\end{gathered}
%\right\}%
%\end{align*}

%$$\left\{
%\begin{align}
%\alpha&=f(z)\cr \beta&=f(z^2)\cr \gamma&=f(z^3)\\
%%\end{align}
%\right\}
%\qquad
%\left\{
%%\begin{align}
%x&=\alpha^2-\beta\cr y&=2\gamma\\
%\end{align}
%\right\}.$$
%%% page 192

\begin{align*}
\begin{aligned}
(x+y)(x-y)&=x^2-xy+yx-y^2\\
&=x^2-y^2\\
(x+y)^2&=x^2+2xy+y^2.
\end{aligned}
\end{align*}

%% page 192

\begin{align*}
\begin{aligned}
\left( \int\limits_{-\infty}^\infty \mathup{e}^{-x^2}\,\mathup{d}x \right)^2
&=\int_{-\infty}^\infty\int_{-\infty}^\infty \mathup{e}^{-(x^2+y^2)}\,\mathup{d}x\,\mathup{d}y\\
&=\int_0^{2\piup}\int_0^\infty \mathup{e}^{-r^2}\,\mathup{d}r\,\mathup{d}\theta\\
&=\int_0^{2\piup}\biggl(\mathup{e}^{-\frac{r^2}{2}}\biggl|_{r=0}^{r=\infty}\,\biggr)\,\mathup{d}\theta\\
&=\piup.
\end{aligned}
\end{align*}


%% page 197

$$\prod_{k\ge0}\frac{1}{(1-q^kz)}=
\sum_{n\ge0}z^n\bigg/\!\!\prod_{1\le k\le n}(1-q^k).$$

$$\sum_{\substack{\scriptstyle 0< i\le m\\\scriptstyle0<j\le n}}p(i,j) \,\ne
%
% $$\sum_{i=1}^p \sum_{j=1}^q \sum_{k=1}^r a_{ij} b_{jk} c_{ki}$$
%
\sum_{i=1}^p \sum_{j=1}^q \sum_{k=1}^r a_{ij} b_{jk} c_{ki} \,\ne
%
\sum_{\substack{\scriptstyle 1\le i\le p \\ \scriptstyle 1\le j\le q\\
		\scriptstyle 1\le k\le r}} a_{ij} b_{jk} c_{ki}$$

$$\max_{1\le n\le m}\log_2P_n \quad \hbox{and} \quad
\lim_{x\to0}\frac{\sin x}{x}=1$$
Inline math:
$\max_{1\le n\le m}\log_2P_n \quad \hbox{and} \quad
\lim_{x\to0}\frac{\sin x}{x}=1$
$$p_1(n)=\lim_{m\to\infty}\sum_{\nu=0}^\infty\bigl(1-\cos^{2m}(\nu!^n\piup/n)\bigr)$$
Inline math:
$p_1(n)=\lim_{m\to\infty}\sum_{\nu=0}^\infty\bigl(1-\cos^{2m}(\nu!^n\piup/n)\bigr)$%
	}

	\renewcommand{\showfamily}{{\color{magenta}%
		Sans Serif Bold%
	}}
	{\sffamily\bfseries%
	 \section{Math Test \showfamily}

\subsection{Overview \showfamily}

{\parindent 0pt
Default: $a \alpha \alphaup b \beta G \Gamma \upGamma \epsilon \varepsilon \theta \vartheta P \Pi \Sigma \sigma$; $\sigma_\epsilon, c^\alpha$

mathnormal: $\mathnormal{a \alpha \alphaup b \beta G \Gamma \upGamma \epsilon \varepsilon \theta \vartheta P \Pi \Sigma \sigma}$

mathrm: $\mathrm{a \alpha \alphaup b \beta G \Gamma \upGamma \epsilon \varepsilon \theta \vartheta P \Pi \Sigma \sigma}$

mathup: $\mathup{a \alpha \alphaup b \beta G \Gamma \upGamma \epsilon \varepsilon \theta \vartheta P \Pi \Sigma \sigma}$

mathit: $\mathit{a \alpha \alphaup b \beta G \Gamma \upGamma \epsilon \varepsilon \theta \vartheta P \Pi \Sigma \sigma}$

mathbf: $\mathbf{a \alphaup b \beta G \Gamma \upGamma \epsilon \varepsilon \theta \vartheta P \Pi \Sigma \sigma}$

mathbfit: $\mathbfit{a \alpha b \beta G \Gamma \upGamma \epsilon \varepsilon \theta \vartheta P \Pi \Sigma \sigma}$

mathbfup: $\mathbfup{a \alpha b \beta G \Gamma \upGamma \epsilon \varepsilon \theta \vartheta P \Pi \Sigma \sigma}$

\bigskip

{\bfseries
Default: $a \alpha \alphaup b \beta G \Gamma \upGamma \epsilon \varepsilon \theta \vartheta P \Pi \Sigma \sigma$; $\sigma_\epsilon, c^\alpha$

mathnormal: $\mathnormal{a \alpha \alphaup b \beta G \Gamma \upGamma \epsilon \varepsilon \theta \vartheta P \Pi \Sigma \sigma}$

mathrm: $\mathrm{a \alpha \alphaup b \beta G \Gamma \upGamma \epsilon \varepsilon \theta \vartheta P \Pi \Sigma \sigma}$

mathup: $\mathup{a \alpha \alphaup b \beta G \Gamma \upGamma \epsilon \varepsilon \theta \vartheta P \Pi \Sigma \sigma}$

mathit: $\mathit{a \alpha \alphaup b \beta G \Gamma \upGamma \epsilon \varepsilon \theta \vartheta P \Pi \Sigma \sigma}$

mathbf: $\mathbf{a \alpha \alphaup b \beta G \Gamma \upGamma \epsilon \varepsilon \theta \vartheta P \Pi \Sigma \sigma}$

mathbfit: $\mathbfit{a \alpha \alphaup b \beta G \Gamma \upGamma \epsilon \varepsilon \theta \vartheta P \Pi \Sigma \sigma}$

mathbfup: $\mathbfup{a \alpha \alphaup b \beta G \Gamma \upGamma \epsilon \varepsilon \theta \vartheta P \Pi \Sigma \sigma}$
}

\bigskip

{\sffamily\mdseries
Default: $a \alpha \alphaup b \beta G \Gamma \upGamma \epsilon \varepsilon \theta \vartheta P \Pi \Sigma \sigma$; $\sigma_\epsilon, c^\alpha$

mathnormal: $\mathnormal{a \alpha \alphaup b \beta G \Gamma \upGamma \epsilon \varepsilon \theta \vartheta P \Pi \Sigma \sigma}$

mathrm: $\mathrm{a \alpha \alphaup b \beta G \Gamma \upGamma \epsilon \varepsilon \theta \vartheta P \Pi \Sigma \sigma}$

mathup: $\mathup{a \alpha \alphaup b \beta G \Gamma \upGamma \epsilon \varepsilon \theta \vartheta P \Pi \Sigma \sigma}$

mathit: $\mathit{a \alpha \alphaup b \beta G \Gamma \upGamma \epsilon \varepsilon \theta \vartheta P \Pi \Sigma \sigma}$

mathbf: $\mathbf{a \alpha \alphaup b \beta G \Gamma \upGamma \epsilon \varepsilon \theta \vartheta P \Pi \Sigma \sigma}$

mathbfit: $\mathbfit{a \alpha \alphaup b \beta G \Gamma \upGamma \epsilon \varepsilon \theta \vartheta P \Pi \Sigma \sigma}$

mathbfup: $\mathbfup{a \alpha \alphaup b \beta G \Gamma \upGamma \epsilon \varepsilon \theta \vartheta P \Pi \Sigma \sigma}$
}

\bigskip

{\sffamily\bfseries

Default: $a \alpha \alphaup b \beta G \Gamma \upGamma \epsilon \varepsilon \theta \vartheta P \Pi \Sigma \sigma$; $\sigma_\epsilon, c^\alpha$

mathnormal: $\mathnormal{a \alpha \alphaup b \beta G \Gamma \upGamma \epsilon \varepsilon \theta \vartheta P \Pi \Sigma \sigma}$

mathrm: $\mathrm{a \alpha \alphaup b \beta G \Gamma \upGamma \epsilon \varepsilon \theta \vartheta P \Pi \Sigma \sigma}$

mathup: $\mathup{a \alpha \alphaup b \beta G \Gamma \upGamma \epsilon \varepsilon \theta \vartheta P \Pi \Sigma \sigma}$

mathit: $\mathit{a \alpha \alphaup b \beta G \Gamma \upGamma \epsilon \varepsilon \theta \vartheta P \Pi \Sigma \sigma}$

mathbf: $\mathbf{a \alpha \alphaup b \beta G \Gamma \upGamma \epsilon \varepsilon \theta \vartheta P \Pi \Sigma \sigma}$

mathbfit: $\mathbfit{a \alpha \alphaup b \beta G \Gamma \upGamma \epsilon \varepsilon \theta \vartheta P \Pi \Sigma \sigma}$

mathbfup: $\mathbfup{a \alpha \alphaup b \beta G \Gamma \upGamma \epsilon \varepsilon \theta \vartheta P \Pi \Sigma \sigma}$
}
}


\subsection{Formulas \showfamily}

\noindent%
\checkgreekletters

\noindent%
{\boldmath\checkgreekletters}

\noindent%
{\sffamily\selectfont \checkgreekletters}

\noindent%
{\sffamily\bfseries\selectfont \checkgreekletters}

\noindent%
{\sffamily $\alpha a > 0, \beta b + (3 \times 27), \Gamma G = 7 < 8, \lambda$}

\noindent%
$\alpha a > 0, \beta b + (3 \times 27), \Gamma G = 7 < 8, \lambda$

$\lim_{\nu \to \infty} v(\nu) = \max_{s \in S} \{s \pm 3 \gamma + y - 1\} = 4 \times 7$

$\hat{\beta} = (X'X)^{-1}X'y$

$$\lim_{N \to \infty} \sum_{i=0}^{N} x^i = \min_{x \in \mathbb{R}} S(x)$$

$$\int_{-\infty}^{\infty} x\,f(x)\,\mathup{d}x = \left( \frac{27}{2} \right)$$

Disambiguation: $0$~O~$O$, $1$~l~I~$|$~$l$~$I$~$/$, $i$~$j$, $rn$~$m$, $\theta$~$\Theta$, $\phi$~$\psi$, --~$-$

Latin vs. Greek: $a$~$\alpha$, $d$~$\delta$, $e$~$\epsilon$, $i$~$\iota$, $k$~$\kappa$, $n$~$\eta$, $o$~$\sigma$, $p$~$\rho$, \textit{\ss} $\beta$, $u$~$\upsilon$, $v$~$\nu$, $w$~$\omega$, $x$~$\chi$, $y$~$\gamma$, $A$~$\Delta$~$\Lambda$, $O$~$\Theta$~$\Omega$, $T$~$\Gamma$, $Y$~$\Upsilon$.

\noindent%
{\bfseries%
$\alpha a > 0, \beta b + (3 \times 27), \Gamma G = 7 < 8, \lambda$

$\lim_{\nu \to \infty} v(\nu) = \max_{s \in S} \{s \pm 3 \gamma + y - 1\} = 4 \times 7$

$\hat{\beta} = (X'X)^{-1}X'y$

$$\lim_{N \to \infty} \sum_{i=0}^{N} x^i = \min_{x \in \mathbb{R}} S(x)$$

$$\int_{-\infty}^{\infty} x\,f(x)\,\mathup{d}x = \left( \frac{27}{2} \right)$$

Disambiguation: $0$~O~$O$, $1$~l~I~$|$~$l$~$I$~$/$, $i$~$j$, $rn$~$m$, $\theta$~$\Theta$, $\phi$~$\psi$, --~$-$

Latin vs. Greek: $a$~$\alpha$, $d$~$\delta$, $e$~$\epsilon$, $i$~$\iota$, $k$~$\kappa$, $n$~$\eta$, $o$~$\sigma$, $p$~$\rho$, \textit{\ss} $\beta$, $u$~$\upsilon$, $v$~$\nu$, $w$~$\omega$, $x$~$\chi$, $y$~$\gamma$, $A$~$\Delta$~$\Lambda$, $O$~$\Theta$~$\Omega$, $T$~$\Gamma$, $Y$~$\Upsilon$.
}

\noindent%
{\sffamily%
$\alpha a > 0, \beta b + (3 \times 27), \Gamma G = 7 < 8, \lambda$

$\lim_{\nu \to \infty} v(\nu) = \max_{s \in S} \{s \pm 3 \gamma + y - 1\} = 4 \times 7$

$\hat{\beta} = (X'X)^{-1}X'y$

$$\lim_{N \to \infty} \sum_{i=0}^{N} x^i = \min_{x \in \mathbb{R}} S(x)$$

$$\int_{-\infty}^{\infty} x\,f(x)\,\mathup{d}x = \left( \frac{27}{2} \right)$$

Disambiguation: $0$~O~$O$, $1$~l~I~$|$~$l$~$I$~$/$, $i$~$j$, $rn$~$m$, $\theta$~$\Theta$, $\phi$~$\psi$, --~$-$

Latin vs. Greek: $a$~$\alpha$, $d$~$\delta$, $e$~$\epsilon$, $i$~$\iota$, $k$~$\kappa$, $n$~$\eta$, $o$~$\sigma$, $p$~$\rho$, \textit{\ss} $\beta$, $u$~$\upsilon$, $v$~$\nu$, $w$~$\omega$, $x$~$\chi$, $y$~$\gamma$, $A$~$\Delta$~$\Lambda$, $O$~$\Theta$~$\Omega$, $T$~$\Gamma$, $Y$~$\Upsilon$.
}

\noindent%
{\sffamily\bfseries%
$\alpha a > 0, \beta b + (3 \times 27), \Gamma G = 7 < 8, \lambda$

$\lim_{\nu \to \infty} v(\nu) = \max_{s \in S} \{s \pm 3 \gamma + y - 1\} = 4 \times 7$

$\hat{\beta} = (X'X)^{-1}X'y$

$$\lim_{N \to \infty} \sum_{i=0}^{N} x^i = \min_{x \in \mathbb{R}} S(x)$$

$$\int_{-\infty}^{\infty} x\,f(x)\,\mathup{d}x = \left( \frac{27}{2} \right)$$

Disambiguation: $0$~O~$O$, $1$~l~I~$|$~$l$~$I$~$/$, $i$~$j$, $rn$~$m$, $\theta$~$\Theta$, $\phi$~$\psi$, --~$-$

Latin vs. Greek: $a$~$\alpha$, $d$~$\delta$, $e$~$\epsilon$, $i$~$\iota$, $k$~$\kappa$, $n$~$\eta$, $o$~$\sigma$, $p$~$\rho$, \textit{\ss} $\beta$, $u$~$\upsilon$, $v$~$\nu$, $w$~$\omega$, $x$~$\chi$, $y$~$\gamma$, $A$~$\Delta$~$\Lambda$, $O$~$\Theta$~$\Omega$, $T$~$\Gamma$, $Y$~$\Upsilon$.
}


\subsection{Math Alphabets \showfamily}

%\sffamily\selectfont

Default
\def\test#1{#1,}
\begin{eqnarray*}
  && {\testnums}\\
  && {\testupper}\\
  && {\testlower}\\
  && {\testupgreek}\\
  && {\testlowgreek}
\end{eqnarray*}%

Math Normal (\texttt{\string\mathnormal})
\def\test#1{\mathnormal{#1},}
\begin{eqnarray*}
  && {\testnums}\\
  && {\testupper}\\
  && {\testlower}\\
  && {\testupgreek}\\
  && {\testlowgreek}
\end{eqnarray*}%

Math Italic (\texttt{\string\mathit})
\def\test#1{\mathit{#1},}
\begin{eqnarray*}
  && {\testnums}\\
  && {\testupper}\\
  && {\testlower}\\
  && {\testupgreek}\\
  && {\testlowgreek}
\end{eqnarray*}%

Math Roman (\texttt{\string\mathrm})
\def\test#1{\mathrm{#1},}
\begin{eqnarray*}
  && {\testnums}\\
  && {\testupper}\\
  && {\testlower}\\
  && {\testupgreek}\\
  && {\testlowgreek}
\end{eqnarray*}%

%Math Italic Bold (\texttt{\string\mathbm})
%\def\test#1{\mathbm{#1},}
%\begin{eqnarray*}
%  && {\testnums}\\
%  && {\testupper}\\
%  && {\testlower}\\
%  && {\testupgreek}\\
%  && {\testlowgreek}
%\end{eqnarray*}%

Math Bold (\texttt{\string\mathbf})
\def\test#1{\mathbf{#1},}
\begin{eqnarray*}
  && {\testnums}\\
  && {\testupper}\\
  && {\testlower}\\
  && {\testupgreek}\\
  && {\testlowgreek}
\end{eqnarray*}%

Caligraphic (\texttt{\string\mathcal})
\def\test#1{\mathcal{#1},}
\begin{eqnarray*}
  && {\testupper}
\end{eqnarray*}%

Script (\texttt{\string\mathscr})
\def\test#1{\mathscr{#1},}
\begin{eqnarray*}
  && {\testupper}
\end{eqnarray*}%

Fraktur (\texttt{\string\mathfrak})
\def\test#1{\mathfrak{#1},}
\begin{eqnarray*}
  && {\testupper}\\
  && {\testlower}
\end{eqnarray*}%

Blackboard Bold (\texttt{\string\mathbb})
\def\test#1{\mathbb{#1},}
\begin{eqnarray*}
  && {\testupper}
\end{eqnarray*}%

\subsection{Character Sidebearings \showfamily}

Default
\def\test#1{|#1|+{}}
\begin{eqnarray*}
  && {\testupperi}\\
  && {\testupperii}\\
  && {\testloweri}\\
  && {\testlowerii}\\
  && {\testupgreeki}\\
  && {\testupgreekii}\\
  && {\testlowgreeki}\\
  && {\testlowgreekii}\\
  && {\testlowgreekiii}
\end{eqnarray*}%

Math Roman (\texttt{\string\mathrm})
\def\test#1{|\mathrm{#1}|+{}}%
\begin{eqnarray*}
  && {\testupperi}\\
  && {\testupperii}\\
  && {\testloweri}\\
  && {\testlowerii}\\
  && {\testupgreeki}\\
  && {\testupgreekii}
\end{eqnarray*}%

%Math Italic Bold (\texttt{\string\mathbm})
%\def\test#1{|\mathbm{#1}|+{}}%
%\begin{eqnarray*}
%  && {\testupperi}\\
%  && {\testupperii}\\
%  && {\testloweri}\\
%  && {\testlowerii}\\
%  && {\testupgreeki}\\
%  && {\testupgreekii}\\
%  && {\testlowgreeki}\\
%  && {\testlowgreekii}\\
%  && {\testlowgreekiii}
%\end{eqnarray*}%

Math Bold (\texttt{\string\mathbf})
\def\test#1{|\mathbf{#1}|+{}}%
\begin{eqnarray*}
  && {\testupperi}\\
  && {\testupperii}\\
  && {\testloweri}\\
  && {\testlowerii}\\
  && {\testupgreeki}\\
  && {\testupgreekii}
\end{eqnarray*}%

Math Calligraphic (\texttt{\string\mathcal})
\def\test#1{|\mathcal{#1}|+{}}%
\begin{eqnarray*}
  && {\testupperi}\\
  && {\testupperii}
\end{eqnarray*}%


\subsection{Superscript Positioning \showfamily}

Default
\def\test#1{#1^{2}+{}}%
\begin{eqnarray*}
  && {\testupperi}\\
  && {\testupperii}\\
  && {\testloweri}\\
  && {\testlowerii}\\
  && {\testupgreeki}\\
  && {\testupgreekii}\\
  && {\testlowgreeki}\\
  && {\testlowgreekii}\\
  && {\testlowgreekiii}
\end{eqnarray*}%

Math Roman (\texttt{\string\mathrm})
\def\test#1{\mathrm{#1}^{2}+{}}%
\begin{eqnarray*}
  && {\testupperi}\\
  && {\testupperii}\\
  && {\testloweri}\\
  && {\testlowerii}\\
  && {\testupgreeki}\\
  && {\testupgreekii}
\end{eqnarray*}%

%Math Italic Bold (\texttt{\string\mathbm})
%\def\test#1{\mathbm{#1}^{2}+{}}%
%\begin{eqnarray*}
%  && {\testupperi}\\
%  && {\testupperii}\\
%  && {\testloweri}\\
%  && {\testlowerii}\\
%  && {\testupgreeki}\\
%  && {\testupgreekii}\\
%  && {\testlowgreeki}\\
%  && {\testlowgreekii}\\
%  && {\testlowgreekiii}
%\end{eqnarray*}%

Math Bold (\texttt{\string\mathbf})
\def\test#1{\mathbf{#1}^{2}+{}}%
\begin{eqnarray*}
  && {\testupperi}\\
  && {\testupperii}\\
  && {\testloweri}\\
  && {\testlowerii}\\
  && {\testupgreeki}\\
  && {\testupgreekii}
\end{eqnarray*}

Math Calligraphic (\texttt{\string\mathcal})
\def\test#1{\mathcal{#1}^{2}+{}}%
\begin{eqnarray*}
  && {\testupperi}\\
  && {\testupperii}
\end{eqnarray*}%


\subsection{Subscript Positioning \showfamily}

Default
\def\test#1{\mathnormal{#1}_{i}+{}}%
\begin{eqnarray*}
  && {\testupperi}\\
  && {\testupperii}\\
  && {\testloweri}\\
  && {\testlowerii}\\
  && {\testupgreeki}\\
  && {\testupgreekii}\\
  && {\testlowgreeki}\\
  && {\testlowgreekii}\\
  && {\testlowgreekiii}
\end{eqnarray*}%

Math Roman (\texttt{\string\mathrm})
\def\test#1{\mathrm{#1}_{i}+{}}%
\begin{eqnarray*}
  && {\testupperi}\\
  && {\testupperii}\\
  && {\testloweri}\\
  && {\testlowerii}\\
  && {\testupgreeki}\\
  && {\testupgreekii}
\end{eqnarray*}%

%Math Bold Italic (\texttt{\string\mathbm})
%\def\test#1{\mathbm{#1}_{i}+{}}%
%\begin{eqnarray*}
%  && {\testupperi}\\
%  && {\testupperii}\\
%  && {\testloweri}\\
%  && {\testlowerii}\\
%  && {\testupgreeki}\\
%  && {\testupgreekii}\\
%  && {\testlowgreeki}\\
%  && {\testlowgreekii}\\
%  && {\testlowgreekiii}
%\end{eqnarray*}

Math Bold (\texttt{\string\mathbf})
\def\test#1{\mathbf{#1}_{i}+{}}%
\begin{eqnarray*}
  && {\testupperi}\\
  && {\testupperii}\\
  && {\testloweri}\\
  && {\testlowerii}\\
  && {\testupgreeki}\\
  && {\testupgreekii}
\end{eqnarray*}%

Math Calligraphic (\texttt{\string\mathcal})
\def\test#1{\mathcal{#1}_{i}+{}}%
\begin{eqnarray*}
  && {\testupperi}\\
  && {\testupperii}
\end{eqnarray*}%


\subsection{Accent Positioning \showfamily}

Default
\def\test#1{\hat{#1}+{}}%
\begin{eqnarray*}
  && {\testnums}\\
  && {\testupperi}\\
  && {\testupperii}\\
  && {\testloweri}\\
  && {\testlowerii}\\
  && {\testupgreeki}\\
  && {\testupgreekii}\\
  && {\testlowgreeki}\\
  && {\testlowgreekii}\\
  && {\testlowgreekiii}
\end{eqnarray*}%

Math Italic (\texttt{\string\mathit})
\def\test#1{\hat{\mathit{#1}}+{}}%
\begin{eqnarray*}
  && {\testnums}\\
  && {\testupperi}\\
  && {\testupperii}\\
  && {\testloweri} \test\ell \test\wp \test\imath \test\jmath \tilde{i} \\
  && {\testlowerii}\\
  && {\testupgreeki}\\
  && {\testupgreekii}\\
  && {\testlowgreeki}\\
  && {\testlowgreekii}\\
  && {\testlowgreekiii}
\end{eqnarray*}%

Math Roman (\texttt{\string\mathrm})
\def\test#1{\hat{\mathrm{#1}}+{}}%
\begin{eqnarray*}
  && {\testnums}\\
  && {\testupperi}\\
  && {\testupperii}\\
  && {\testloweri}\\
  && {\testlowerii}\\
  && {\testupgreeki}\\
  && {\testupgreekii}
\end{eqnarray*}%

%Math Italic Bold (\texttt{\string\mathbm})
%\def\test#1{\hat{\mathbm{#1}}+{}}%
%\begin{eqnarray*}
%  && {\testnums}\\
%  && {\testupperi}\\
%  && {\testupperii}\\
%  && {\testloweri}\\
%  && {\testlowerii}\\
%  && {\testupgreeki}\\
%  && {\testupgreekii}\\
%  && {\testlowgreeki}\\
%  && {\testlowgreekii}\\
%  && {\testlowgreekiii}
%\end{eqnarray*}%

Math Bold (\texttt{\string\mathbf})
\def\test#1{\hat{\mathbf{#1}}+{}}%
\begin{eqnarray*}
  && {\testnums}\\
  && {\testupperi}\\
  && {\testupperii}\\
  && {\testloweri}\\
  && {\testlowerii}\\
  && {\testupgreeki}\\
  && {\testupgreekii}
\end{eqnarray*}

Math Calligraphic (\texttt{\string\mathcal})
\def\test#1{\hat{\mathcal{#1}}+{}}%
\begin{eqnarray*}
  && {\testupperi}\\
  && {\testupperii}
\end{eqnarray*}%


\subsection{Differentials \showfamily}

\begin{eqnarray*}
\gdef\test#1{\dit #1+{}}%
  && {\testupperi}\\
  && {\testupperii}\\
  && {\testloweri}\\
  && {\testlowerii}\\
  && {\testupgreeki}\\
  && {\testupgreekii}\\
  && {\testlowgreeki}\\
  && {\testlowgreekii}\\
  && {\testlowgreekiii}\\
\gdef\test#1{\dit \mathrm{#1}+{}}%
  && {\testupgreeki}\\
  && {\testupgreekii}
\end{eqnarray*}%

\begin{eqnarray*}
\gdef\test#1{\dup #1+{}}%
  && {\testupperi}\\
  && {\testupperii}\\
  && {\testloweri}\\
  && {\testlowerii}\\
  && {\testupgreeki}\\
  && {\testupgreekii}\\
  && {\testlowgreeki}\\
  && {\testlowgreekii}\\
  && {\testlowgreekiii}\\
\gdef\test#1{\dup \mathrm{#1}+{}}%
  && {\testupgreeki}\\
  && {\testupgreekii}
\end{eqnarray*}%

\begin{eqnarray*}
\gdef\test#1{\partial #1+{}}%
  && {\testupperi}\\
  && {\testupperii}\\
  && {\testloweri}\\
  && {\testlowerii}\\
  && {\testupgreeki}\\
  && {\testupgreekii}\\
  && {\testlowgreeki}\\
  && {\testlowgreekii}\\
  && {\testlowgreekiii}\\
\gdef\test#1{\partial \mathrm{#1}+{}}%
  && {\testupgreeki}\\
  && {\testupgreekii}
\end{eqnarray*}%


\subsection{Slash Kerning \showfamily}

\def\test#1{1/#1+{}}
\begin{eqnarray*}
  && {\testupperi}\\
  && {\testupperii}\\
  && {\testloweri}\\
  && {\testlowerii}\\
  && {\testupgreeki}\\
  && {\testupgreekii}\\
  && {\testlowgreeki}\\
  && {\testlowgreekii}\\
  && {\testlowgreekiii}
\end{eqnarray*}

\def\test#1{#1/2+{}}
\begin{eqnarray*}
  && {\testupperi}\\
  && {\testupperii}\\
  && {\testloweri}\\
  && {\testlowerii}\\
  && {\testupgreeki}\\
  && {\testupgreekii}\\
  && {\testlowgreeki}\\
  && {\testlowgreekii}\\
  && {\testlowgreekiii}
\end{eqnarray*}


\subsection{(Big) Operators \showfamily}

\def\testop#1{#1_{i=1}^{n} x^{n} \quad}
$
	\testop\sum
	\testop\prod
	\testop\coprod
	\testop\int
	\testop\oint
$

\noindent%
$
	\testop\bigotimes
	\testop\bigoplus
	\testop\bigodot
	\testop\bigwedge
	\testop\bigvee
	\testop\biguplus
	\testop\bigcup
	\testop\bigcap
	\testop\bigsqcup
	% \testop\bigsqcap
$

\begin{displaymath}
  \testop\sum
  \testop\prod
  \testop\coprod
  \testop\int
  \testop\oint
\end{displaymath}
\begin{displaymath}
  \testop\bigotimes
  \testop\bigoplus
  \testop\bigodot
  \testop\bigwedge
  \testop\bigvee
  \testop\biguplus
  \testop\bigcup
  \testop\bigcap
  \testop\bigsqcup
% \testop\bigsqcap
\end{displaymath}


\subsection{Radicals \showfamily}

\begin{displaymath}
  \sqrt{x+y} \qquad \sqrt{x^{2}+y^{2}} \qquad
  \sqrt{x_{i}^{2}+y_{j}^{2}} \qquad
  \sqrt{\left(\frac{\cos x}{2}\right)} \qquad
  \sqrt{\left(\frac{\sin x}{2}\right)}
\end{displaymath}

\begingroup
\delimitershortfall-1pt
\begin{displaymath}
  \sqrt{\sqrt{\sqrt{\sqrt{\sqrt{\sqrt{\sqrt{x+y}}}}}}}
\end{displaymath}
\endgroup % \delimitershortfall


\subsection{Over- and Underbraces \showfamily}

\begin{displaymath}
  \overbrace{x} \quad
  \overbrace{x+y} \quad
  \overbrace{x^{2}+y^{2}} \quad
  \overbrace{x_{i}^{2}+y_{j}^{2}} \quad
  \underbrace{x} \quad
  \underbrace{x+y} \quad
  \underbrace{x_{i}+y_{j}} \quad
  \underbrace{x_{i}^{2}+y_{j}^{2}} \quad
\end{displaymath}


\subsection{Normal and Wide Accents \showfamily}

\begin{displaymath}
  \dot{x} \quad
  \ddot{x} \quad
  \vec{x} \quad
  \bar{x} \quad
  \overline{x} \quad
  \overline{xx} \quad
  \tilde{x} \quad
  \widetilde{x} \quad
  \widetilde{xx} \quad
  \widetilde{xxx} \quad
  \hat{x} \quad
  \widehat{x} \quad
  \widehat{xx} \quad
  \widehat{xxx} \quad
\end{displaymath}

\begin{displaymath}
  \hat{x} \quad
  \check{x} \quad
  \tilde{x} \quad
  \acute{x} \quad
  \grave{x} \quad
  \dot{x} \quad
  \ddot{x} \quad
  \breve{x} \quad
  \bar{x} \quad
  \vec{x} \quad
\end{displaymath}


\subsection{Long Arrows \showfamily}

\begin{displaymath}
  \leftarrow \mathrel{-} \rightarrow \quad
  \leftrightarrow \quad
  \longleftarrow  \quad
  \longrightarrow \quad
  \longleftrightarrow \quad
  \Leftarrow = \Rightarrow \quad
  \Leftrightarrow \quad
  \Longleftarrow  \quad
  \Longrightarrow \quad
  \Longleftrightarrow \quad
\end{displaymath}


\subsection{Left and Right Delimiters \showfamily}

\def\testdelim#1#2{ - #1 f #2 - }
\begin{displaymath}
  \testdelim()
  \testdelim[]
  \testdelim\lfloor\rfloor
  \testdelim\lceil\rceil
  \testdelim\langle\rangle
  \testdelim\{\}
\end{displaymath}

Using {\tt\string\left} and {\tt\string\right}.
\def\testdelim#1#2{ - \left#1 f \right#2 - }
\begin{displaymath}
  \testdelim()
  \testdelim[]
  \testdelim\lfloor\rfloor
  \testdelim\lceil\rceil
  \testdelim\langle\rangle
  \testdelim\{\}
% \testdelim\lgroup\rgroup
% \testdelim\lmoustache\rmoustache
\end{displaymath}
\begin{displaymath}
  \testdelim)(
  \testdelim][
  \testdelim//
  \testdelim\backslash\backslash
  \testdelim/\backslash
  \testdelim\backslash/
\end{displaymath}


\subsection{Big-g-g Delimiters \showfamily}

\def\testdelim#1#2{%
  - \left#1\left#1\left#1\left#1\left#1\left#1\left#1\left#1 -
  \right#2\right#2\right#2\right#2\right#2\right#2\right#2\right#2 -}

\begingroup
\delimitershortfall-1pt
\begin{displaymath}
  \testdelim\lfloor\rfloor
  \qquad
  \testdelim()
\end{displaymath}
\begin{displaymath}
  \testdelim\lceil\rceil
  \qquad
  \testdelim\{\}
\end{displaymath}
\begin{displaymath}
  \testdelim[]
  \qquad
  \testdelim\lgroup\rgroup
\end{displaymath}
\begin{displaymath}
  \testdelim\langle\rangle
  \qquad
  \testdelim\lmoustache\rmoustache
\end{displaymath}
\begin{displaymath}
  \testdelim\uparrow\downarrow \quad
  \testdelim\Uparrow\Downarrow \quad
\end{displaymath}
\endgroup % \delimitershortfall

\def\X#1{$x #1 y$ &\tt\string#1}
\def\Y#1{$\big#1$ &\tt\string#1}
\def\Z#1{$x #1 y$}
\def\W#1#2{$#1{#2}$ &\tt\string#1\string{#2\string}}


\subsection{Binary Operators \showfamily}

\begin{tabular}{*8l}
\X\pm           &\X\cap         &\X\diamond             &\X\oplus     \\
\X\mp           &\X\cup         &\X\bigtriangleup       &\X\ominus    \\
\X\times        &\X\uplus       &\X\bigtriangledown     &\X\otimes    \\
\X\div          &\X\sqcap       &\X\triangleleft        &\X\oslash    \\
\X\ast          &\X\sqcup       &\X\triangleright       &\X\odot      \\
\X\star         &\X\vee         &\X\lhd                 &\X\bigcirc   \\
\X\circ         &\X\wedge       &\X\rhd                 &\X\dagger    \\
\X\bullet       &\X\setminus    &\X\unlhd               &\X\ddagger   \\
\X\cdot         &\X\wr          &\X\unrhd               &\X\S         \\
\X+             &\X-            &\X\amalg               &\X\P
\end{tabular}


\subsection{Relations \showfamily}

\begin{tabular}{*8l}
\X\leq          &\X\geq         &\X\equiv       &\X\models      \\
\X\prec         &\X\succ        &\X\sim         &\X\perp        \\
\X\preceq       &\X\succeq      &\X\simeq       &\X\mid         \\
\X\ll           &\X\gg          &\X\asymp       &\X\parallel    \\
\X\subset       &\X\supset      &\X\approx      &\X\bowtie      \\
\X\subseteq     &\X\supseteq    &\X\cong        &\X\Join        \\
\X\sqsubset     &\X\sqsupset    &\X\neq         &\X\smile       \\
\X\sqsubseteq   &\X\sqsupseteq  &\X\doteq       &\X\frown       \\
\X\in           &\X\ni          &\X\propto      &\X=            \\
\X\vdash        &\X\dashv       &\X<            &\X>            \\
\X:
\end{tabular}


\subsection{Punctuation \showfamily}

\begin{tabular}{*{5}{lp{3.2em}}}
\X,     &\X;    &\X\colon       &\X\ldotp       &\X\cdotp
\end{tabular}


\subsection{Arrows \showfamily}

\begin{tabular}{*6l}
\X\leftarrow            &\X\longleftarrow       &\X\uparrow     \\
\X\Leftarrow            &\X\Longleftarrow       &\X\Uparrow     \\
\X\rightarrow           &\X\longrightarrow      &\X\downarrow   \\
\X\Rightarrow           &\X\Longrightarrow      &\X\Downarrow   \\
\X\leftrightarrow       &\X\longleftrightarrow  &\X\updownarrow \\
\X\Leftrightarrow       &\X\Longleftrightarrow  &\X\Updownarrow \\
\X\mapsto               &\X\longmapsto          &\X\nearrow     \\
\X\hookleftarrow        &\X\hookrightarrow      &\X\searrow     \\
\X\leftharpoonup        &\X\rightharpoonup      &\X\swarrow     \\
\X\leftharpoondown      &\X\rightharpoondown    &\X\nwarrow     \\
\X\rightleftharpoons    &\X\leadsto
\end{tabular}


\subsection{Miscellaneous Symbols \showfamily}

\begin{tabular}{*8l}
\X\ldots        &\X\cdots       &\X\vdots       &\X\ddots       \\
\X\aleph        &\X\prime       &\X\forall      &\X\infty       \\
\X\hbar         &\X\emptyset    &\X\exists      &\X\Box         \\
\X\imath        &\X\nabla       &\X\neg         &\X\Diamond     \\
\X\jmath        &\X\surd        &\X\flat        &\X\triangle    \\
\X\ell          &\X\top         &\X\natural     &\X\clubsuit    \\
\X\wp           &\X\bot         &\X\sharp       &\X\diamondsuit \\
\X\Re           &\X\|           &\X\backslash   &\X\heartsuit   \\
\X\Im           &\X\angle       &\X\partial     &\X\spadesuit   \\
\X\mho          &\X.            &\X|            &\X!
\end{tabular}


\subsection{Variable-Sized Operators \showfamily}

\begin{tabular}{*6l}
\X\sum          &\X\bigcap      &\X\bigodot     \\
\X\prod         &\X\bigcup      &\X\bigotimes   \\
\X\coprod       &\X\bigsqcup    &\X\bigoplus    \\
\X\int          &\X\bigvee      &\X\biguplus    \\
\X\oint         &\X\bigwedge
\end{tabular}


\subsection{Log-Like Operators \showfamily}

\begin{tabular}{*8l}
\Z\arccos &\Z\cos  &\Z\csc &\Z\exp &
           \Z\ker    &\Z\limsup &\Z\min &\Z\sinh \\
\Z\arcsin &\Z\cosh &\Z\deg &\Z\gcd &
           \Z\lg     &\Z\ln     &\Z\Pr  &\Z\sup  \\
\Z\arctan &\Z\cot  &\Z\det &\Z\hom &
           \Z\lim    &\Z\log    &\Z\sec &\Z\tan  \\
\Z\arg    &\Z\coth &\Z\dim &\Z\inf &
           \Z\liminf &\Z\max    &\Z\sin &\Z\tanh
\end{tabular}


\subsection{Delimiters \showfamily}

\begin{tabular}{*8l}
\X(             &\X)            &\X\uparrow     &\X\Uparrow     \\
\X[             &\X]            &\X\downarrow   &\X\Downarrow   \\
\X\{            &\X\}           &\X\updownarrow &\X\Updownarrow \\
\X\lfloor       &\X\rfloor      &\X\lceil       &\X\rceil       \\
\X\langle       &\X\rangle      &\X/            &\X\backslash   \\
\X|             &\X\|
\end{tabular}


\subsection{Large Delimiters \showfamily}

\begin{tabular}{*8l}
\Y\rmoustache&  \Y\lmoustache&  \Y\rgroup&      \Y\lgroup\\[5pt]
\Y\arrowvert&   \Y\Arrowvert&   \Y\bracevert
\end{tabular}


\subsection{Math Mode Accents \showfamily}

\begin{tabular}{*{10}l}
\W\hat{a}     &\W\acute{a}  &\W\bar{a}    &\W\dot{a}    &\W\breve{a}\\
\W\check{a}   &\W\grave{a}  &\W\vec{a}    &\W\ddot{a}   &\W\tilde{a}\\
\end{tabular}


\subsection{Miscellaneous Constructions \showfamily}

\begin{tabular}{*4l}
\W\widetilde{abc}       &\W\widehat{abc}                        \\
\W\overleftarrow{abc}   &\W\overrightarrow{abc}                 \\
\W\overline{abc}        &\W\underline{abc}                      \\
\W\overbrace{abc}       &\W\underbrace{abc}                     \\[5pt]
\W\sqrt{abc}            &$\sqrt[n]{abc}$&\verb|\sqrt[n]{abc}|   \\
$f'$&\verb|f'|          &$\frac{abc}{xyz}$&\verb|\frac{abc}{xyz}|
\end{tabular}


\subsection{AMS Delimiters \showfamily}

\begin{tabular}{*8l}
\X\ulcorner&\X\urcorner&\X\llcorner&\X\lrcorner
\end{tabular}


\subsection{AMS Arrows \showfamily}

\begin{tabular}{*8l}
\X\dashrightarrow       &\X\dashleftarrow
        \\ \X\leftleftarrows      &\X\leftrightarrows     \\
\X\Lleftarrow           &\X\twoheadleftarrow
        \\ \X\leftarrowtail       &\X\looparrowleft       \\
\X\leftrightharpoons    &\X\curvearrowleft
        \\ \X\circlearrowleft     &\X\Lsh                 \\
\X\upuparrows           &\X\upharpoonleft
        \\ \X\downharpoonleft     &\X\multimap            \\
\X\leftrightsquigarrow  &\X\rightrightarrows
        \\ \X\rightleftarrows     &\X\rightrightarrows    \\
\X\rightleftarrows      &\X\twoheadrightarrow
        \\ \X\rightarrowtail      &\X\looparrowright      \\
\X\rightleftharpoons    &\X\curvearrowright
        \\ \X\circlearrowright    &\X\Rsh                 \\
\X\downdownarrows       &\X\upharpoonright
        \\ \X\downharpoonright    &\X\rightsquigarrow
\end{tabular}


\subsection{AMS Negated Arrows \showfamily}

\begin{tabular}{*8l}
\X\nleftarrow   &\X\nrightarrow \\ \X\nLeftarrow  &\X\nRightarrow \\
\X\nleftrightarrow&\X\nLeftrightarrow
\end{tabular}


\subsection{AMS Greek \showfamily}

\begin{tabular}{*4l}
\X\digamma      &\X\varkappa
\end{tabular}


\subsection{AMS Hebrew \showfamily}

\begin{tabular}{*6l}
\X\beth &\X\daleth      &\X\gimel
\end{tabular}


\subsection{AMS Miscellaneous \showfamily}

\begin{tabular}{*8l}
\X\hbar         &\X\hslash      \\ \X\vartriangle &\X\triangledown      \\
\X\square       &\X\lozenge     \\ \X\circledS    &\X\angle             \\
\X\measuredangle&\X\nexists     \\ \X\mho         &\X\Finv$^u$          \\
\X\Game$^u$     &\X\Bbbk$^u$    \\ \X\backprime   &\X\varnothing        \\
\X\blacktriangle&\X\blacktriangledown \\ \X\blacksquare&\X\blacklozenge  \\
\X\bigstar      &\X\sphericalangle     \\ \X\complement  &\X\eth       \\
\X\diagup$^u$   &\X\diagdown$^u$
\end{tabular}

$^u$ Not defined in {\tt amssymb.sty}, define using the
\verb|\newsymbol|  command.


\subsection{AMS Binary Operators \showfamily}

\begin{tabular}{*8l}
\X\dotplus      &\X\smallsetminus \\ \X\Cap        &\X\Cup               \\
\X\barwedge     &\X\veebar      \\ \X\doublebarwedge&\X\boxminus        \\
\X\boxtimes     &\X\boxdot      \\ \X\boxplus     &\X\divideontimes     \\
\X\ltimes       &\X\rtimes      \\ \X\leftthreetimes&\X\rightthreetimes \\
\X\curlywedge   &\X\curlyvee    \\ \X\circleddash &\X\circledast        \\
\X\circledcirc  &\X\centerdot   \\ \X\intercal
\end{tabular}


\subsection{AMS Relations \showfamily}

\begin{tabular}{*2l}
\X\leqslant    \\\X\lesssim    \\
\X\approxeq    \\\X\lll        \\
\X\lesseqgtr   \\\X\doteqdot   \\
\X\fallingdotseq\\\X\backsimeq  \\
\X\Subset      \\\X\preccurlyeq\\
\X\precsim     \\\X\vartriangleleft\\
\X\vDash      \\\X\smallsmile \\
\X\bumpeq      \\\X\geqq       \\
\X\eqslantgtr  \\\X\gtrapprox  \\
\X\ggg         \\\X\gtreqless  \\
\X\eqcirc      \\\X\triangleq  \\
\X\thickapprox \\\X\Supset     \\
\X\succcurlyeq \\\X\succsim    \\
\X\vartriangleright\\\X\Vdash      \\
\X\shortparallel\\\X\pitchfork  \\
\X\blacktriangleleft \\\X\backepsilon\\
\X\because
\end{tabular}


\subsection{AMS Negated Relations \showfamily}

\begin{tabular}{*8l}
\X\nless        &\X\nleq        \\ \X\nleqslant   &\X\nleqq       \\
\X\lneq         &\X\lneqq       \\ \X\lvertneqq   &\X\lnsim       \\
\X\lnapprox     &\X\nprec       \\ \X\npreceq     &\X\precnsim    \\
\X\precnapprox  &\X\nsim        \\ \X\nshortmid   &\X\nmid        \\
\X\nvdash       &\X\nvDash      \\ \X\ntriangleleft&\X\ntrianglelefteq\\
\X\nsubseteq    &\X\subsetneq   \\ \X\varsubsetneq&\X\subsetneqq  \\
\X\varsubsetneqq&\X\ngtr        \\ \X\ngeq        &\X\ngeqslant   \\
\X\ngeqq        &\X\gneq        \\ \X\gneqq       &\X\gvertneqq   \\
\X\gnsim        &\X\gnapprox    \\ \X\nsucc       &\X\nsucceq     \\
\X\nsucceqq     &\X\succnsim    \\ \X\succnapprox &\X\ncong       \\
\X\nshortparallel&\X\nparallel  \\ \X\nvDash      &\X\nVDash      \\
\X\ntriangleright&\X\ntrianglerighteq \\ \X\nsupseteq&\X\nsupseteqq\\
\X\supsetneq    &\X\varsupsetneq \\ \X\supsetneqq  &\X\varsupsetneqq
\end{tabular}%
	 \subsection{Math ``Torture'' Test \showfamily}
	 Most of the following examples are taken from \textit{The \TeX book} \citep[][see \url{https://ctan.org/pkg/texbook}]{Knuth1984} and were adapted for \LaTeX\ from Karl Berry's torture test for plain \TeX\ math fonts.

\noindent $x + y - z$, \quad $x + y * z$, \quad $z * y / z$, \quad 
$(x+y)(x-y) = x^2 - y^2$, 

\noindent $x \times y \cdot z = [x\, y\, z]$, \quad $x\circ y \bullet z$, \quad
$x\cup y \cap z$, \quad $x\sqcup y \sqcap z$, \quad

\noindent $x \vee y \wedge z$, \quad $x\pm y\mp z$, \quad
$x=y/z$, \;\; $x:=y$, \;\; $x\le y \ne z$, \;\; $x \sim y \simeq z$
$x \equiv y \nequiv z$, \;\; $x\subset y \subseteq z$

\noindent $\sin2\theta=2\sin\theta\cos\theta$, \quad
$\hbox{O}(n\log n\log n)$, \quad
$\Pr(X>x)=\exp(-x/\mu)$,

\noindent $\bigl(x\in A(n)\bigm|x\in B(n)\bigr)$, \quad
$\bigcup_n X_n\bigm\|\bigcap_n Y_n$

% page 178

\noindent In-text matrices $\binom{1\,1}{0\,1}$ and $\bigl(\genfrac{}{}{0pt}{}{a}{1}\genfrac{}{}{0pt}{}{b}{m}\genfrac{}{}{0pt}{}{c}{n}\bigr)$.

% page 142

$$a_0+\frac1{\displaystyle a_1 +
	{\strut \frac1{\displaystyle a_2 +
			{\strut \frac1{\displaystyle a_3 +
					{\strut \frac1{\displaystyle a_4}}}}}}}$$

% page 143

$$\binom{p}{2}x^2y^{p-2} - \frac1{1 - x}\frac{1}{1 - x^2}
=
\frac{a+1}{b}\bigg/\frac{c+1}{d}.$$

%% page 145

$$\sqrt{1+\sqrt{1+\sqrt{1+\sqrt{1+\sqrt{1+x}}}}}$$

$$\sqrt[n]{1+\sqrt[k]{1+\sqrt[5]{1+\sqrt[4]{1+\sqrt[3]{1+x}}}}}$$

%% page 147

$$\left(\frac{\partial^2}{\partial x^2} + \frac{\partial^2}{\partial y^2}\right)
\bigl|\varphi(x+\mathup{i}y)\bigr|^2=0$$

%% page 149

% $$\pi(n)=\sum_{m=2}^n\left\lfloor\biggl(\sum_{k=1}^{m-1}\bigl
% \lfloor(m/k)\big/\lceil m/k\rceil\bigr\rfloor\biggr)^{-1}\right\rfloor.$$

$$\pi(n)=\sum_{m=2}^n\left\lfloor\Biggl(\sum_{k=1}^{m-1}\bigl
\lfloor(m/k)\big/\lceil m/k\rceil\bigr\rfloor\Biggr)^{-1}\right\rfloor.$$

% page 168

$$\int_0^\infty \frac{t - \mathup{i} b}{t^2 + b^2}e^{\mathup{i}at}\,\mathup{d}t=e^{ab}E_1(ab), \quad
a,b > 0.$$

% page 176

$$\mathbf{A} \coloneqq \begin{pmatrix}x-\lambda&1&0\\
0&x-\lambda&1\\
0&0&x-\lambda\end{pmatrix}.$$

$$\left\lgroup\begin{matrix}a&b&c\\ d&e&f\\\end{matrix}\right\rgroup
\left\lgroup\begin{matrix}u&x\cr v&y\cr w&z\end{matrix}\right\rgroup$$

% page 177

$$\mathbf{A} = \begin{pmatrix}a_{11}&a_{12}&\ldots&a_{1n}\\
a_{21}&a_{22}&\ldots&a_{2n}\\
\vdots&\vdots&\ddots&\vdots\\
a_{m1}&a_{m2}&\ldots&a_{mn}\end{pmatrix}$$

$$\mathbf{M}=\bordermatrix{&C&I&C'\cr
	C&1&0&0\cr I&b&1-b&0\cr C'&0&a&1-a}$$

%% page 186

$$\sum_{n=0}^\infty a_nz^n\quad\hbox{converges if}\quad
|z|<\Bigl(\limsup_{n\to\infty}\root n\of{|a_n|}\,\Bigr)^{-1}.$$

$$\frac{f(x+\mathup{\Delta} x)-f(x)}{\mathup{\Delta} x}\to f'(x)
\qquad \hbox{as $\mathup{\Delta} x\to0$.}$$

$$\|u_i\|=1,\qquad u_i\cdot u_j=0\quad\hbox{if $i\ne j$.}$$

%% page 191

$$\hbox{The confluent image of}\quad
\begin{Bmatrix}\hbox{an arc}\hfill\\\hbox{a circle}\hfill\\
\hbox{a fan}\hfill\\\end{Bmatrix}
\quad\hbox{is}\quad
\begin{Bmatrix}\hbox{an arc}\hfill\\
\hbox{an arc or a circle}\hfill\\
\hbox{a fan or an arc}\hfill\end{Bmatrix}.$$

%% page 191

\begin{align*}
T(n)\le T(2^{\lceil\lg n\rceil})
&\le c(3^{\lceil\lg n\rceil}-2^{\lceil\lg n\rceil})\\
&<3c\cdot3^{\lg n}\\
&=3c\,n^{\lg3}.
\end{align*}

%\begin{align*}
%\left\{%
%\begin{gathered}\alpha&=f(z)\\ \beta&=f(z^2)\\ \gamma&=f(z^3)
%\end{gathered}
%\right\}
%\qquad
%\left\{%
%\begin{gathered}
%x&=\alpha^2-\beta\\ y&=2\gamma
%\end{gathered}
%\right\}%
%\end{align*}

%$$\left\{
%\begin{align}
%\alpha&=f(z)\cr \beta&=f(z^2)\cr \gamma&=f(z^3)\\
%%\end{align}
%\right\}
%\qquad
%\left\{
%%\begin{align}
%x&=\alpha^2-\beta\cr y&=2\gamma\\
%\end{align}
%\right\}.$$
%%% page 192

\begin{align*}
\begin{aligned}
(x+y)(x-y)&=x^2-xy+yx-y^2\\
&=x^2-y^2\\
(x+y)^2&=x^2+2xy+y^2.
\end{aligned}
\end{align*}

%% page 192

\begin{align*}
\begin{aligned}
\left( \int\limits_{-\infty}^\infty \mathup{e}^{-x^2}\,\mathup{d}x \right)^2
&=\int_{-\infty}^\infty\int_{-\infty}^\infty \mathup{e}^{-(x^2+y^2)}\,\mathup{d}x\,\mathup{d}y\\
&=\int_0^{2\piup}\int_0^\infty \mathup{e}^{-r^2}\,\mathup{d}r\,\mathup{d}\theta\\
&=\int_0^{2\piup}\biggl(\mathup{e}^{-\frac{r^2}{2}}\biggl|_{r=0}^{r=\infty}\,\biggr)\,\mathup{d}\theta\\
&=\piup.
\end{aligned}
\end{align*}


%% page 197

$$\prod_{k\ge0}\frac{1}{(1-q^kz)}=
\sum_{n\ge0}z^n\bigg/\!\!\prod_{1\le k\le n}(1-q^k).$$

$$\sum_{\substack{\scriptstyle 0< i\le m\\\scriptstyle0<j\le n}}p(i,j) \,\ne
%
% $$\sum_{i=1}^p \sum_{j=1}^q \sum_{k=1}^r a_{ij} b_{jk} c_{ki}$$
%
\sum_{i=1}^p \sum_{j=1}^q \sum_{k=1}^r a_{ij} b_{jk} c_{ki} \,\ne
%
\sum_{\substack{\scriptstyle 1\le i\le p \\ \scriptstyle 1\le j\le q\\
		\scriptstyle 1\le k\le r}} a_{ij} b_{jk} c_{ki}$$

$$\max_{1\le n\le m}\log_2P_n \quad \hbox{and} \quad
\lim_{x\to0}\frac{\sin x}{x}=1$$
Inline math:
$\max_{1\le n\le m}\log_2P_n \quad \hbox{and} \quad
\lim_{x\to0}\frac{\sin x}{x}=1$
$$p_1(n)=\lim_{m\to\infty}\sum_{\nu=0}^\infty\bigl(1-\cos^{2m}(\nu!^n\piup/n)\bigr)$$
Inline math:
$p_1(n)=\lim_{m\to\infty}\sum_{\nu=0}^\infty\bigl(1-\cos^{2m}(\nu!^n\piup/n)\bigr)$%
	}
}

\end{refsection}

%\end{comment}


%%%%%%%%%%%%%%%%%%%%%%%%%%%%%  FORMAL STUFF  %%%%%%%%%%%%%%%%%%%%%%%%%%%%


\ifnum \dissafterdefense=1
	\begin{refsection}
		\selectlanguage{ngerman}
		\chapter*{%
			Lebenslauf%
			%	\setcounter{footnote}{1}\textsuperscript{\fnsymbol{footnote}}
		}
		% !TeX spellcheck = de_DE


Geboren am 24.~Januar~1995 in \dissbornin, wuchs ich in Neustadt (Nord\-rhein-West\-falen) sowie in Newcastle (Nova Landia, Neufundland) auf. Im Jahr~2013 erlangte ich am Gymnasium Neustadt die allgemeine Hoch\-schul\-reife. Im Winter\-semester 2013/2014 habe ich zun\"achst das Studium der Kunst\-geschichte an der Rheinischen Friedrich-Wilhelms-Universit\"at Bonn begonnen. Im Sommersemester 2014 nahm ich dann das Studium der Volks\-wirt\-schafts\-lehre auf, das ich im August 2018 mit dem Abschluss Master of Science (M.\,Sc.) beendete (Gesamtnote: 1,3). Meine Masterarbeit \glqq\selectlanguage{american}The Influence of Stress on the Performance of \caps{BGSE} Graduate Students\selectlanguage{ngerman}\grqq\ wurde von Prof.~Dr. Lorem Ipsum betreut. W\"ahrend des Masterstudiums besuchte ich im Herbst 2016 die Universit\"{a}t Tel Aviv in Israel als Austauschstudent. Im Oktober~2018 habe ich das Promo\-tions\-studium an der Bonn \selectlanguage{american}Graduate School of Economics\selectlanguage{ngerman} aufgenommen.
	\end{refsection}
\fi


\end{document}