% Preamble file to include all required packages and 

% Basic packages for mathematics
\usepackage{etex, amsmath, amssymb, amsthm, latexsym, mathtools}


% Formatting packages
\usepackage{a4wide, fancyhdr, titlesec, lipsum, ragged2e, microtype, thmtools, ccicons, needspace, setspace, framed, lipsum, mdframed}
% a4wide	- Sets the geometry to use a reasonable amount of width
% fancyhdr	- Gives us interesting headers and footers
% titlesec	- Define different formatting and fonts for titles
% lipsum	- Automatically generate Lorem Ipsum text
% ragged2e	- Improve ragged right tex
% microtype	- Letter spacing
% thmtools	- Theorem styling
% ccicons	- Text icons for Creative Commons
% needspace	- Prevent page breaks between example heads and bodies
% setspace	- Define singlespace areas
% framed	- For shaded quotes
% lipsum	- For lorem ipsum
% mdframed	- For framing the Algorithm style


% Packages to make tables useful and pretty
\usepackage{booktabs, array, multicol, multirow, longtable, lscape}
% booktabs	- Make nice tables by banishing \hline
% array		- Set widths on table columns
% multicol	- Allow cells to span columns
% multirow	- Allow cells to span rows
% longtable	- Enable pagebreaks in a table
% lscape	- Enable landscape mode

\usepackage[table]{xcolor}

% Relevant graphics packages
\usepackage{graphicx, wrapfig, titlepic, textpos, tikz, pgfplots}
% graphicx	- Include graphics
% wrapfig	- Allow text to wrap around figures
% titlepic	- Add a picture to the title page (requires titlepage option for articles)
% textpos	- Absolute positioning on the page
% tikz		- For tikzpictures
% pgfplots	- For charts

% Packages for dual thesis/paper use
\usepackage{xspace, etoolbox, todonotes, appendix}
% xspace	- Clever spacing when defining new commands
% etoolbox	- Conditional logic
% todonotes	- Add notes for things that need doing
% appendix	- Add appendix title

% Other packages required
\usepackage{listings, enumitem}
% listings	- Automatically colour code
% enumitem	- Customise lists

% Natbib, for the Author Year citation styles
\usepackage[round]{natbib}

% Fonts
\usepackage{fontspec, unicode-math}
\defaultfontfeatures{Scale=MatchLowercase}

\setmainfont[Ligatures=TeX, Numbers=OldStyle]{Linux Libertine O}
%\setmainfont{Hoefler Text}
\setsansfont[Mapping=tex-text]{Helvetica}
\setmonofont{Consolas}

\newcommand{\dispnums}{\fontspec[Ligatures=TeX]{Linux Libertine O}}

\setmathfont[Numbers=OldStyle]{Cambria Math}
% use Libertine for the letters
\setmathfont[range=\mathit/{latin,Latin,num,Greek,greek}, Numbers=OldStyle]{Linux Libertine O Italic}
\setmathfont[range=\mathup/{latin,Latin,num,Greek,greek}, Numbers=OldStyle]{Linux Libertine O}
\setmathfont[range=\mathbfup/{latin,Latin,num,Greek,greek}, Numbers=OldStyle]{Linux Libertine O Bold Italic}
%\setmathfont[range={"2202}]{Linux Libertine O}% "02202 = \partial % doesn't work
\setmathfont[range={"221E}]{Linux Libertine O}% "0221E = \infty
% etc. (list should be completed depending on needs)
\setmathfont[range={"025BE}]{XITS Math} % \blacktriangledown

\usepackage{tocloft}
% tocloft	- TOC formatting

% Change the space between page numbers and dots in the TOC, and typeset everything in sans serif
\cftsetpnumwidth{2em}
\renewcommand{\cfttoctitlefont}{\Huge \sffamily \bfseries}

\renewcommand{\cftpartfont}{\sffamily \bfseries \Large}
\renewcommand{\cftpartpagefont}{\sffamily \bfseries \Large}

\renewcommand{\cftchappagefont}{\sffamily \bfseries \large}
\renewcommand{\cftchapfont}{\sffamily \bfseries \large}

\renewcommand{\cftsecfont}{\sffamily}
\renewcommand{\cftsecpagefont}{\sffamily}

\renewcommand{\cftsubsecfont}{\sffamily}
\renewcommand{\cftsubsecpagefont}{\sffamily}


% The pdf version will have hyperlinks and extra details, so set \ispdfversion to 'true' to get those (defined in Guide.tex).
\newbool{pdfversion}
\setbool{pdfversion}{\ispdfversion}

\ifbool{pdfversion}{
	\usepackage{hyperref}
	% \hyperref	- For hyperlinks
	\hypersetup{
	    pdftitle	 = {Problem-Solving in Undergraduate Mathematics and Computer-aided Assessment},    % title
	    pdfauthor	 = {M.~S.~Badger},     % author
	    pdfsubject	 = {Problem-solving},   % subject of the document
	    pdfcreator	 = {Matthew Badger},   % creator of the document
	    pdfproducer	 = {Matthew Badger}, % producer of the document
	    pdfkeywords	 = {Mathematics} {Problem-solving} {Moore Method} {Computer-aided Assessment}, % list of keywords
	    pdfnewwindow = true,
		colorlinks	 = true,
		linkcolor	 = red!40!black,
		citecolor	 = blue!60!black,
		urlcolor	 = green!60!black,
		unicode      = true
	}
}

%%%% Page formatting %%%%

% Allow LaTeX leeway to get rid of widows and orphans
% \setlength{\parskip}{2ex plus 1ex minus 1ex}
\widowpenalty=1000
\clubpenalty=1000

% Tweak the margins to match requirements, vary depending on whether it's the pdf or print version.
\setlength{\topmargin}{-0.5cm}
\setlength{\footskip}{1cm}
\setlength{\evensidemargin}{0cm}
\setlength{\textwidth}{16cm}
\setlength{\textheight}{22cm}
\setlength{\vfuzz}{1pc}
\setlength{\hfuzz}{1pc}

\ifbool{pdfversion}{
	\linespread{1.0}
	\setlength{\oddsidemargin}{0cm}
}{
	\linespread{2.0}
	\setlength{\oddsidemargin}{0.5cm}
}

% Set the titles in sans font

\titleformat{\part}                                     % Level begin redefined
	[display]                                           % Paragraph shape (display, block, etc.)
	{\thispagestyle{empty} \Large \bfseries \sffamily}	% Formatting of the whole title; label + text
	{\centering Part \thepart}                          % Label, usually \Xtitlename\ \theX
	{0pt}                                               % Horizontal separation between label and body
	{\centering \Huge}                                  % Code preceeding the title body

\titleformat{\chapter}									% Level begin redefined
	[display]											% Paragraph shape (display, block, etc.)
	{\thispagestyle{empty} \sffamily\large\bfseries}	% Formatting of the whole title; label + text
	{\centering \chaptertitlename\ \thechapter}			% Label, usually \Xtitlename\ \theX
	{0pt}												% Horizontal separation between label and body
	{\centering \huge}									% Code preceeding the title body
	[]													% Code after the title body
  
\titleformat*{\section}{\sffamily\Large\bfseries}
\titleformat*{\subsection}{\sffamily\large\bfseries}
\titleformat*{\subsubsection}{\sffamily\bfseries}

% Define the different types of theorem
\newtheorem{thm}{Theorem}[section]
\newtheorem{theorem}{Theorem}[section]
\newtheorem{lemma}[theorem]{Lemma}
\newtheorem{proposition}[theorem]{Proposition}
\newtheorem{corollary}[theorem]{Corollary}
\newtheorem{definition}[theorem]{Definition}

\newcommand{\exrule}{\rule{\textwidth-4em}{1pt}}
	
\declaretheoremstyle[
	headfont		= \sffamily\bfseries,			% Head font
	headformat		= $\blacktriangledown$ \NAME~\NUMBER\NOTE\vspace{-3mm},	% LaTeX for the header; \NAME, \NUMBER and \NOTE supplied by thmtools, be careful with spacing
	postheadhook	= ~\newline\exrule\newline,		% Punctuation after the head
	notefont		= \itshape,						% Fond of the note (example name in this case)
	notebraces		= {--- }{},						% Braces before and after note; again be careful with spaces.
	bodyfont		= \itshape \rmfamily \linespread{1.05}, % Body font and formatting
	preheadhook		= \quote,						% Hook before the head
	postheadspace	= 0pt,							% Space after the head; we want none as we have a new line already
	prefoothook		= \ifhmode \hspace*{\fill}\newline \else \vspace{-5mm} \fi \exrule,	% Hook for LaTeX code before the foot of the example (the space below it). If we're in horizontal mode, add a new line, otherwise just draw the ruled line. This makes 90% of the examples require no manual spacing.
	postfoothook	= \endquote
	]{examplestyle}
	
\declaretheorem[style=examplestyle, numberwithin=section]{example}


\declaretheoremstyle[
	spaceabove		= 15pt, spacebelow=15pt,		% Space above and below the entire example
	headfont		= \sffamily\bfseries,			% Head font
	headformat		= \NAME~\NUMBER\NOTE\vspace{-3mm},	% LaTeX for the header; \NAME, \NUMBER and \NOTE supplied by thmtools, be careful with spacing
	headpunct		= \newline\newline,				% Punctuation after the head
	notefont		= \itshape,						% Fond of the note (example name in this case)
	notebraces		= {--- }{},						% Braces before and after note; again be careful with spaces.
	bodyfont		= \sffamily\bfseries,			% Body font and formatting
	postheadspace	= 10pt							% Space after the head
	]{algorstyle}

\theoremstyle{algorstyle}
\newmdtheoremenv[linecolor=black, leftmargin=30, rightmargin=30, skipabove=20pt, skipbelow=20pt]{algorithm}%
	{Algorithm}[section]

% For quote shading
\ifbool{pdfversion}{
	\definecolor{shadecolor}{rgb}{0.93,0.95,1}
}{
	\definecolor{shadecolor}{rgb}{0.95,0.95,0.95}
}


% Basic mathematics commands

\newcommand{\R}{{\mathbb R}}
\newcommand{\N}{{\mathbb N}}
\newcommand{\Z}{{\mathbb Z}}
\newcommand{\Q}{{\mathbb Q}}
\newcommand{\F}{{\mathbb F}}
\renewcommand{\C}{{\mathbb C}}
\renewcommand{\T}{{\mathbb T}}

\newcommand{\<}{\langle}
\renewcommand{\>}{\rangle}

\newcommand{\ftt}{\footnotesize\texttt}

\newcommand{\ie}{\emph{i.e.\@} }
\newcommand{\etc}{\emph{etc.\@} }

% The following command adds space to either side of the symbol above an xrightarrow
\newcommand*{\myrightarrow}[1]{\xrightarrow{\mathmakebox[2em]{#1}}}

% Create the @ symbol

\makeatletter
\def\imod#1{\allowbreak\mkern10mu({\operator@font mod}\,\,#1)}
\makeatother

\newfontfamily\quotefont[Ligatures=TeX]{Linux Libertine O}

% Command for the open quotes
\newcommand*{\openquote}{\tikz[remember picture,overlay,xshift=-15pt,yshift=-13pt]
     \node (OQ) {\quotefont\fontsize{60}{60}\selectfont``};\kern0pt}

% Wrap everything in its own environment
\newenvironment{shadequote}%
{\begin{snugshade}\begin{quote}\openquote}
{\hfill\end{quote}\end{snugshade}}