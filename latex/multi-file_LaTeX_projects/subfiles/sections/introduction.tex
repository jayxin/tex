\documentclass[../main.tex]{subfiles}
\graphicspath{{\subfix{../images/}}}
\begin{document}
%The actual contents is typed inside \begin{document}
%and \end{document}. Everything outside this environment
%will be ignored, or more specifically, will be considered as
%part of the preamble. Avoid leaving blank lines at the top
%and bottom of the file.
\textbf{Hello world!}

\begin{figure}[bh]
\centering
\includegraphics[width=3cm]{overleaf-logo}

\label{fig:img1}
\caption{Overleaf logo}
\end{figure}

Hello, here is some text without a meaning.
This text should show what a printed text will look like at this place.  If you read this text, you will get no information.  Really?  Is there no information?  Is there a diference between this text and some nonsense like Huardest gefburn"?  Kjift  not at all!  A blindtext like this gives you information about the selected font, how the letters a rewritten and an impression of the look.  This text should contain all letters of the alphabet and it should be written in of the original language.  There is no need for special content, but the length of words should match the language.
\end{document}
